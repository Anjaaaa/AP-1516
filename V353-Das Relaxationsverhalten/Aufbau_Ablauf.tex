Zunächst soll die \textbf{Zeitkonstante} ermittelt werden. Dazu werden mit einer Schaltung nach Abbildung VERWEIS die Auf- und Entladungskurven auf dem Oszilloskop visualisiert. Mit Hilfe der \textsc{Cursor}-Funktion des Oszilloskops wird dann bei einer Aufladekurve der Endwert bestimmt und für verschiedene Zeitpunkte $t_i$ der Wert $U(t_i)$ abgelesen. \\
Danach wird mit Schaltung VERWEIS die \textbf{Amplitude $\bm{A}$} bestimmt. Hierfür werden am Sinus-Generator nacheinander verschiedene Frequenzen eingestellt, die einen großen Frequenzbereich überstreichen. Über die \textsc{Measure}-Funktion des Oszilloskops kann dann die jeweilige Amplitude angezeigt werden. \\
Die \textbf{Phasenverschiebung $\bm{\varphi}$} zwischen der Erreger- und der Kondensatorfrequenz wird mit Schaltung VERWEIS gemessen. So können beide Spannungen am Oszilloskop angezeigt werden. Am Sinus-Generator werden dieselben Frequenzen, die zur Messung der Amplituden verwendet werden, eingestellt und mit der \textsc{Cursor}-Funktion wird direkt die Phasenverschiebung als $\Delta t$ zwischen den Nulldurchgängen gemessen. \\
Im letzten Versuchsteil soll die \textbf{Integratorfunktion} des RC-Kreises verifiziert werden. Dafür wird wiederum Schaltung VERWEIS verwendet. Der Spannungsgenerator generiert eine Rechteck-, eine Sinus- und eine Dreieckspannung. Am Oszilloskop können dann die Spannung aus dem Generator und die integrierte Spannung aus dem RC-Kreis angezeigt werden.