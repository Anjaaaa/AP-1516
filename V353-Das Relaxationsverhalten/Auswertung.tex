\todo[color=red]{Das große Problem ist: Wir haben das $U_0$ nicht bestimmt. Ich habe es einfach auf 20 Volt geschätzt. Es scheint jedoch etwas kleiner gewesen zu sein.}

\todo[inline, color = green]{Ich bin mir ziemlich sicher, dass das immer 19.4 Volt waren. Erinnerst du dich an die Messung der Amplituden? Bei den ersten Werten haben wir doch immer nachgeschaut, ob die Generatorspannung auch wirklich konstant bleibt. Und das waren 19.4.}


\subsection{Berechnung der Zeitkonstanten durch}
\subsubsection{... die Aufladekurve des Kondensators}
Um die Gleichung für die linerade Regression zu erhalten, wird Formel \eqref{eq:aufladen} umgeformt zu:
\begin{equation}
\ln(U_C - U_0) = -\frac{1}{RC} t + \ln(U_0)
\end{equation}
Die angelegte Spannung $U_0$ entspricht der Peak to Peak Amplitude der Rechteckspannung.
\begin{align*}
	U_0 = \SI{20}{\volt}
\end{align*}
In Abbildung \ref{fig:spannung1} ist die Differenz der angelegten Spannung und der Kondensatorspannung $U_0 - U_C$ halblogarithmisch über die Zeit aufgetragen. Eine lineare Ausgleichsrechnung der Form
\begin{equation}
\ln(U_C - U_0) = m \cdot t + b
\end{equation} an die in Tabelle \ref{tab:aufladekurve} dargestellten Werte mittels Python liefert:
\begin{align}
	m = \SI{-964.0(229)}{\second} \\
	b = \num{2.84(4)} \\
	\text{Zeitkonstante:} \quad RC = - \frac{1}{m} = \SI{1,037(25)e-3}{\second} \\
	\text{Berechnete Ausgangsspannung:} \quad U_{0\text{reg}} = e ^b \, \si{\volt} = \SI{17,1(6)}{\volt}
\end{align}
	
	
	
	
	

\begin{figure}[h!]
	\centering
	\includegraphics[width=0.7\textwidth]{aufladekurve.png}
	\caption{Aufladekurve des Kondensators bei angelegter Rechteckspannung}
	\label{fig:aufladekurve}
\end{figure} 

\begin{figure}[h!]
	\centering
	\includegraphics[width=0.7\textwidth]{Spannung1.png}
	\caption{Spannungsdifferenzen halblogarithmisch aufgetragen}
	\label{fig:spannung1}
\end{figure} 

\begin{figure}[h!]
	\centering
	\includegraphics[width=0.7\textwidth]{Spannung2.png}
	\caption{Ausgleichsgerade zur Bestimmung der Zeitkonstanten}
	\label{fig:Spannung2}
\end{figure} 

\begin{figure}[h!]
	\centering
	\captionof{table}{Werte der Aufladekurve}
	\begin{tabular}{c|c}
		Zeit in \si{\milli\second}& $\ln(U_0-U)$ \\
		\hline
		0.1 &  2.88 \\
		0.2 &  2.75 \\
		0.3 &  2.64 \\
		0.4 &  2.52 \\
		0.5 &  2.4  \\
		0.6 &  2.28 \\
		0.7 &  2.17 \\
		0.8 &  2.05 \\
		0.9 &  1.95 \\
		1   &  1.82 \\
		1.1 &  1.72 \\
		1.2 &  1.61 \\
		1.3 &  1.53 \\
		1.4 &  1.39 \\
		1.5 &  1.28 \\
		1.6 &  1.22 \\
		1.9 &  0.88 \\
		2.2 &  0.59 \\
		2.5 &  0.34 \\
		2.8 &  0.18 \\
		3.1 &  0    \\
		3.4 & -0.22 \\
	\end{tabular}
	\label{tab:aufladekurve}
\end{figure}



\clearpage
\subsubsection{... die Ampitude bei periodischer Anregung}
\begin{figure}[h!]
\centering
\includegraphics[width=0.7\textwidth]{Amplitude.png}
\caption{Amplitude in Abhängigkeit der Frequenz}
\label{fig:amplitude}
\end{figure} 

\todo[color=red]{Das große Problem ist: Wir haben das $U_0$ nicht bestimmt. Ich habe es einfach auf 20 Volt geschätzt. Es scheint jedoch etwas kleiner gewesen zu sein.}

\todo[inline, color = green]{Ich bin mir ziemlich sicher, dass das immer 19.4 Volt waren. Erinnerst du dich an die Messung der Amplituden? Bei den ersten Werten haben wir doch immer nachgeschaut, ob die Generatorspannung auch wirklich konstant bleibt. Und das waren 19.4.}

