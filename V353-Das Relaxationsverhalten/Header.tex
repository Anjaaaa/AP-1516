\usepackage{geometry}
\geometry{a4paper, top=40mm, left=30mm, right=30mm, bottom=40mm} % bottom und top 40

\usepackage{multicol}
\usepackage{color}

\setlength\parindent{0pt}
\usepackage[german]{babel}
\usepackage[utf8]{inputenc}

\usepackage{amsmath, amsthm, amssymb}
\usepackage{bm}
\usepackage{url}
\usepackage{physics}
\usepackage{enumitem}

\usepackage{caption}
\usepackage{graphicx, wrapfig}
\usepackage{subcaption} %z.B. in figure umgebungen um einzelne bilder zu benennen

\usepackage[version=4]{mhchem}
\usepackage{tikz}
%\usetikzlibrary{matrix}
%\usetikzlibrary{circuits.ee.IEC}
%%Volt- und Amperemeter festlegen:
%\tikzset{circuit declare symbol = ammeter}
%\tikzset{set ammeter graphic ={draw,generic circle IEC, minimum size=5mm,info=center:A}}
%\tikzset{circuit declare symbol = voltmeter}
%\tikzset{set voltmeter graphic ={draw,generic circle IEC, minimum size=5mm,info=center:V}}
\usepackage[european]{circuitikz}
\usetikzlibrary{arrows}
\newcommand{\mymeter}[2] 		% Option um Schaltsymbole zu drehen
{  % #1 = name , #2 = rotation angle
	\begin{scope}[transform shape,rotate=#2]
		\draw[thick] (#1)node(){$\mathbf V$} circle (11pt);
		\draw[rotate=45,-latex] (#1)  +(-17pt,0) --+(17pt,0);
	\end{scope}
}
\usepackage[
	disable,
	colorinlistoftodos,
	linecolor=yellow,
	backgroundcolor=yellow,
	textwidth=0.15\textwidth,
	textsize=footnotesize]{todonotes}

\usepackage[
locale=DE,
separate-uncertainty=true, % Immer Fehler mit ±
per-mode=symbol-or-fraction, % m/s im Text, sonst \frac
% alternativ:
% per-mode=reciprocal,
% m s^{-1}
output-decimal-marker=., % . statt , für Dezimalzahlen
]{siunitx}


\usepackage{fancyhdr}	%Kopf- und Fußzeile gestalten
\pagestyle{fancy}
\renewcommand{\sectionmark}[1]{\markright{#1}}
\renewcommand{\subsectionmark}[1]{\markright{#1}}
\fancyhead{}				%Default-Einstellungen im Header löschen
%\fancyhead[L]{\sc{Relaxationsverhalten eines RC-Kreises}}
\fancyhead[R]{\sc{\rightmark}}
% \leftmark sind die Sections \rightmark sind dei Subsections



