\subsection{Relaxationsphänomene allgemein}
\todo[inline]{Hier bin ich noch überhaupt nicht zufrieden mit den Variablennamen.}
Der Vorgang, dass ein System aus seinem Ausgangszustand ausgelenkt wird und dann ohne Oszillation zu eben diesem zurückkehrt, wird Relaxation genannt. Je weiter die veränderte Größe $A$ dabei von ihrem Ausgangs- bzw. Endzustand $A_0$
\todo[color=red]{Zum Lesen finde ich es angenehmer, wenn der Endzustand $A_\infty $ heißt} entfernt ist, desto schneller ändert sie sich. Häufig ist dieser Zusammenhang sogar direkt proportional
\begin{align}\label{Rel_allgemein_DGL}
	\dv{A}{t} (t) = b\left(A(t)-A_0\right) \ .
\end{align}
Die Lösung dieser Gleichung ist
\begin{align}\label{Rel_allgemein}
	A(t) = A_0 + \left(A(0) - A_0\right)\mathrm{e}^{bt} \ ,
\end{align}
sie ist beschränkt, da sich aus dem Lösungsweg ergibt, dass $b<0$. \footnote{Nach: Versuchsanleitung zu V353: \glqq Das Relaxationsverhalten eines RC-Kreises\grqq, Anfängerpraktikum TU Dortmund, \url{http://129.217.224.2/HOMEPAGE/PHYSIKER/BACHELOR/AP/SKRIPT/V353.pdf}, abgerufen am 20.01.2016 um 20:23 Uhr} \\
Im Versuch wird ein RC-Kreis stellvertretend für einen beliebigen mechanischen Relaxationsvorgang betrachtet.
\subsection{RC-Kreis mit Gleichspannungsquelle \label{sec:Gleichspannung}}
Zunächst wird ein RC-Kreis mit einer Spannungsquelle $U_0 = const$ betrachtet. Mit der Maschenregel folgt
\begin{align}
	U_0 &= U_\text{C} + U_\text{R} = \frac{Q}{C} + \dot{Q}R \\
	\Leftrightarrow\quad \dot{Q}(t) &= \underbrace{-\frac{1}{RC}}_{\sim b}(Q(t) - \underbrace{CU_0}_{\sim A_0}) \ .
\end{align}
Die zweite Gleichung ist äquivalent zu \eqref{Rel_allgemein_DGL}. Die Lösung einer solchen Differenzialgleichung ist (siehe \eqref{Rel_allgemein})
\begin{align}
	Q(t) = CU_0 + \left( Q(0) - CU_0 \right)\mathrm{e}^{-\frac{t}{RC}} \ .
\end{align}
\todo[color=red]{Die Zeitkonstante ist nur RC, nicht der Bruch}
Der als Zeitkonstante bezeichnete Term $\frac{1}{RC}$ ist ein Maß dafür, wie stark das System danach strebt in seine Endposition zu gelangen. Je kleiner die Zeitkonstante ist, desto schneller konvergiert $Q(t)$ gegen seinen Endwert. \glqq Konvergenz\grqq\ ist hier ein wichtiges Stichwort, denn theoretisch ist es unmöglich den Endwert zu erreichen. \\
\ \\
Wird nun ein \textbf{Aufladevorgang} betrachtet, so ist $U_0 \not= 0$, da sonst keine Ladungen auf den Kondensator fließen könnten, und $Q(0) = 0$, der Kondensator soll also ungeladen sein. Daraus folgt, dass die Aufladung durch
\begin{align}\label{Aufladung}
	Q(t) = CU_0 \left( 1 - \mathrm{e}^{-\frac{t}{RC}} \right)
\end{align}
beschrieben werden kann. \\
Bei einem \textbf{Entladevorgang} wird die Spannungsquelle vom Stromkreis getrennt, da der Kondensator sonst keine Ladung abgeben würde. Demnach ist $U_0 = 0$, der Anfangswert $Q(0)$ ist beliebig. Die Gleichung
\begin{align}
	Q(t) = Q(0)\mathrm{e}^{-\frac{t}{RC}}
\end{align}
beschreibt die Entladung eines Kondensators.


\subsection{RC-Kreis mit Wechselspannungsquelle}
Nun wird ein RC-Kreis mit einer Wechselspannungsquelle $U(t) = U_0\cos(\omega t)$ betrachtet. Ist
\begin{align}
	\omega << \frac{1}{RC}
\end{align}
gilt für die Spannung am Kondensator
\begin{align}
	U(t) \simeq U_\text{C}(t) \ .
\end{align}
Da die Auf- bzw. Entladung des Kondensators aber nicht beliebig schnell stattfindet, sondern durch die Zeitkonstante festgelegt ist, verursacht eine steigende Erregerfrequenz $\omega$ eine Phasenverschiebung zwischen $U(t)$ und $U_\text{C}(t)$. Außerdem sorgt sie dafür, dass der Kondensator sich nicht mehr vollständig aufladen kann, sodass auch die Amplitude $U_{\text{C}0}$ abnimmt. Für die Kondensatorspannung gilt also
\todo[color=red]{Können wir die Amplitude A nennen? Das hatte ich gestern in der Auswertung so gemacht und jetzt ist es in meinem Kopf drin...}
\begin{align}
	U_\text{C} = U_{\text{C}0}(\omega)\cos(\omega t + \varphi(\omega)) \ .
\end{align}
Wie in \ref{sec:Gleichspannung} kann auch hier die Maschenregel
\begin{align}\label{Maschenregel}
	U(t) = I(t)R + U_\text{C}(t)
\end{align}
angewandt werden und mit
\begin{align}
	I(t) = \dot{Q}(t) = C\dot{U}_\text{C}(t)
\end{align}
folgt
\begin{align}\label{Maschenregel_DGL}
	U(t) &= C\dot{U}_\text{C}(t)R + U_\text{C}(t) \\
	\Leftrightarrow\quad U_0\cos(\omega t) &= -RCU_{\text{C}0}\omega\sin(\omega t + \varphi) + U_{\text{C}0}\cos(\omega t + \varphi) \ .
\end{align}
Da diese Gleichung für alle $t$ gelten muss, kann zur \textbf{Bestimmung von $\bm{\varphi(\omega)}$} ein beliebiger Wert für $\omega t$ eingesetzt werden (z.B. $\frac{\pi}{2}$) und die Gleichung zu
\begin{align}
	\varphi(\omega) = \arctan(-\omega RC)
\end{align}
umgeformt werden. Bei der \textbf{Bestimmung der Faktoren $\bm{U_{\text{C}0}}$} wird genauso argumentiert. So folgt nach Umformen und unter Verwendung der Gleichung für $\varphi(\omega)$
\begin{align}
	U_{\text{C}0}(\omega) = \frac{U_0}{\sqrt{1+\omega^2R^2C^2}} \ .
\end{align} \\
\ \\
Ist die anregende Frequenz sehr groß, 
\begin{align}
	\omega >> \frac{1}{RC} \quad\Leftrightarrow\quad \omega RC >> 1 \ ,
\end{align}
gilt
%\begin{align}
%	|U_\text{C}| &= \frac{U_0}{\sqrt{1+\omega^2R^2C^2}} \\
%	|U_\text{R}| &= \frac{\omega RCU_0}{\sqrt{1+\omega^2R^2C^2}} \\
%	|U| &= U_0
%\end{align}
\begin{align}
	|U_\text{C}| << |U_\text{R}| \quad \text{und} \quad |U_\text{C}| << |U| \ .
\end{align}
Gleichung \eqref{Maschenregel} bzw. \eqref{Maschenregel_DGL} vereinfacht sich damit zu
\begin{align}
	U(t) &= RC\dot{U}_\text{C}(t) \\
	\Leftrightarrow\quad U_\text{C}(t) &= \frac{1}{RC}\int_0^t U(t')\ dt' \ .
\end{align}
Für hohe Frequenzen kann der RC-Kreis mit Wechselspannungsquelle demnach als Integrator genutzt werden.