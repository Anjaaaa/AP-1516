Die Abweichung um $\SI{-8.7}{\%}$ von den erwarteten $\SI{60}{\degree}$ bei der Messung des Innenwinkels könnte noch durch das Alter des Prismas und damit verbundenen Stürzen erklärt werden. Alle weiteren Werte legen allerdings nahe, dass ein großer systematischer Fehler bei den Messung vorliegen musste. So ist der berechnete Brechwinkel in einem Bereich von fast $\SI{100}{\nano\meter}$ konstant und bricht davor und danach nach oben aus (siehe Tabelle \ref{tab:WinkelFarben}). Alle weiteren Berechnungen machen daher auch nur wenig Sinn. So steigt die Dispersionsfunktion statt abzufallen und die Abbesche Zahl ist negativ, was nicht sein kann. \\
Es wird vermutet, dass ein (kleiner) Teil der Fehler auf das Prisma zurück zuführen sind. Die Kanten waren nicht mehr scharf, teilweise abgeschlagen. Weitaus größere Fehler kommen dadurch zustande, dass die Brechwinkel mutmaßlich in einem Bereich von etwa $\SI{5}{\degree}$ liegen und derart feine Messungen immer Schwierigkeiten mit sich bringen. Der entscheidende Fehler wird allerdings von den Experimentatoren verursacht worden sein. Das Ablesen der (entscheidenden) Nachkommastellen der Gradanzeige führte zu Verwirrung und die Werte wurden von verschiedenen Leuten abgelesen, sodass hier möglicherweise verschiedene Methoden angewandt wurden.