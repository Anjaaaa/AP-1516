Tabelle \ref{fig:Vergleichswerte} zeigt die Brechzahlen einiger Gläser und ihre Abbezahl. Die berechneten Werte von $n_\text{C},n_\text{D},n_\text{F}$ liegen etwas über denen der Literaturwerte. Es könnte sein, dass das verwendete Prisma aus einem Glas mit relativ hohem Brechindex ist. Vermutlich liegt die Ursache der Abweichung aber in der Methode der Bestimmung von $n(\lambda)$. Dabei wurde Gleichung \ref{eq:dispersionsrelation} als Polynom mit nur zwei Termen genähert, sodass hier unvermeidbar ein Fehler entstand.
\begin{table}[h!]
\centering
\begin{tabular}{lcccc}
	Stoff & $n_\text{C}$ & $n_\text{D}$ & $n_\text{F}$ & $\nu$ \\
	\hline
	FK 1 & 1.46 & 1.47 & 1.48 & 67 \\
	BK 1 & 1.51 & 1.51 & 1.51 & 63 \\
	SK 1 & 1.61 & 1.61 & 1.62 & 57 \\
	LF 1 & 1.57 & 1.57 & 1.58 & 43 \\
	F 1 & 1.62 & 1.63 & 1.64 & 46 \\
	SF 4 & 1.75 & 1.76 & 1.77 & 28 \\
	SF 6 & 1.79 & 1.80 & 1.83 & 20 \\
	Quarzglas & 1.46 & 1.46 & 1.46 & 67 \\
	\hline
	$n(\lambda)$ & 1.93 & 1.93 & 1.95 & 41
\end{tabular}
\caption{Brechzahlen verschiedener Gläser bei den Fraunhofer-Linien (siehe Kapitel \ref{sec:Kenndaten}) und Abbezahlen \cite[Tabellenanhang Tab. A 4.3]{Walcher} und berechnete Werte $n(\lambda)$}
\label{fig:Vergleichswerte}
\end{table}
Die Abweichung um \SI{-8.1}{\%} von $\varphi$ von den erwarteten \SI{60}{\degree} deuteten allerdings darauf hin, dass auch die Messung nicht ganz zu vernachlässigende Fehler verursacht hat. Diese liegen vielleicht im Messprozess, wahrscheinlich aber auch am Prisma selbst, das nach einigen Jahren Verwendung schon abgestoßene Kanten hat, sodass die verwendeten Formeln das Experiment nur unzureichend beschreiben.