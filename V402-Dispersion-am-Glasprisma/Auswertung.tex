\subsection{Berechnung von Innenwinkel und Brechwinkel}
Bei der Messung des Innenwinkels des Prismas und mit Formel \eqref{Phi} ergeben sich die die Werte in Tabelle \ref{tab:WinkelPhi}. Der Mittelwert\footnote{siehe \nameref{math:ErrorAver}} ist
\begin{align}
	\overline{\phi} = \SI{55.13+-0.06}{}
 \ .
\end{align}
\begin{table}[h!]
	\centering	
	\begin{tabular}{c|c|c}
		$\phi_r$ & $\phi_l$ & $\phi$ \\
		\hline
		376.0 & 266.0 & 55.0 \\
376.0 & 265.5 & 55.2 \\
376.0 & 265.7 & 55.2 \\

	\end{tabular}
	\caption{Messwerte zur Bestimmung des Innenwinkels $\phi$ in \si{\degree}}
	\label{tab:WinkelPhi}
\end{table} \\
Tabelle \ref{tab:WinkelFarben} zeigt die Winkel $\Omega_l$ und $\Omega_r$, die zusammen mit Formel \eqref{Eta} die Werte für die Brechwinkel $\eta$ ergeben. Zu beachten ist hierbei, dass zwischen den beiden Winkel die Nulllinie überschritten wurde, sodass, um die eigentliche Differenz zu erhalten, zu $\Omega_l$ jeweils \SI{360}{\degree} addiert werden müssen.
\begin{table}[h!]
	\centering	
	\begin{tabular}{c|c|c|c}
		Wellenlänge $\lambda$ in \si{\nano\meter} & $\Omega_l$ & $\Omega_r$ & $\eta$ \\
		\hline
		405.0 & 11.4 & 252.1 & 60.7 \\
440.0 & 12.1 & 252.0 & 59.9 \\
467.0 & 13.1 & 252.9 & 59.8 \\
485.0 & 8.7  & 254.6 & 65.9 \\
546.0 & 14.1 & 254.6 & 60.5 \\
578.0 & 14.6 & 255.1 & 60.5 \\
636.0 & 15.0 & 255.5 & 60.5 \\
644.0 & 11.2 & 255.9 & 64.7 \\

	\end{tabular}
	\caption{Messwerte zur Bestimmung des Brechwinkels $\eta$ in \si{\degree}}
	\label{tab:WinkelFarben}
\end{table}
\clearpage
\subsection{Bestimmung der Dispersionskurve}
Mit Hilfe von Formel \eqref{BrechIndex} kann nun für jede Wellenlänge $\lambda$ der Brechungsindex (siehe Tabelle \ref{tab:Brechzahl}) ausgerechnet werden.
\begin{table}[h!]
	\centering	
	\begin{tabular}{c|c|c}
		$\lambda$ in \si{\nano\meter} & $n$ & Fehler $\Delta_n$ \\
		\hline
		405.0 & 1.839 & 0.006 \\
440.0 & 1.831 & 0.006 \\
467.0 & 1.830 & 0.006 \\
485.0 & 1.890 & 0.006 \\
546.0 & 1.837 & 0.006 \\
578.0 & 1.837 & 0.006 \\
636.0 & 1.837 & 0.006 \\
644.0 & 1.878 & 0.006 \\

	\end{tabular}
	\caption[Errechnete Brechzahlen $n$ mit Fehlern für jede Wellenlänge $\lambda$]{Errechnete Brechzahlen $n$ mit Fehlern\footnotemark\ für jede Wellenlänge $\lambda$}
	\label{tab:Brechzahl}
\end{table}
\footnotetext{siehe: \nameref{math:ErrorBrechindex}} \\
Wie in der Theorie beschrieben muss nun entschieden werden, ob das betrachtete Spektrum über (Fall 1) oder unter (Fall 2) der Absorptionsstelle liegt. Dazu werden Formel \eqref{Fall1} als
\begin{align}\label{Reg:1}
	n^2(\lambda) = A_0 + \frac{A_2}{\lambda^2}
\end{align}
und Formel \eqref{Fall2} als
\begin{align}\label{Reg:2}
	n^2(\lambda) = A_0' + A_2'\lambda^2
\end{align}
genähert. Es wird jeweils eine lineare Regression\footnote{siehe: \nameref{math:Regression}} mit den Wertepaaren $\{ (\frac{1}{\lambda_i^2}, n^2(\lambda_i)) \}$ (bei \eqref{Reg:1}) bzw. $\{ (\lambda_i^2, n^2(\lambda_i)) \}$ (bei \eqref{Reg:2}) durchgeführt, um die Parameter $A_0, A_2, A_0', A_2'$ auszurechnen. Die \glqq Güte\grqq\ der Parameter kann durch die quadratischen Abweichungen\footnote{siehe: \nameref{math:QuadrAbw}} $s^2, s'^2$ bestimmt werden. Alle diese Werte finden sich in Tabelle \ref{tab:Regression}.
\begin{table}[h!]
	\centering
	\begin{tabular}{c|c|c}
		Größe & Wert & relativer Fehler \\
		\hline
		$A_2$ & \SI{-2.3+-2.8e-14}{\meter\squared}
 & \SI{1.18}{\meter\squared}
 \\
		$A_2'$ & \SI{2.2+-3.6e-25}{\meter^{-2}}
 & \SI{1.00}{\meter^{-2}}
 \\
		$A_0$ & \SI{3.47+-0.11}{}
 & \SI{0.01}{}
 \\
		$A_0'$ & \SI{4.6826+-0.0063}{}
 & \SI{0.00}{}
 \\
		\hline
		$s^2(A_0,A_2)$ & \SI{5.371e-14}{}
 & \\
		$s'^2(A_0',A_2')$ & \SI{0.082}{}
 &
	\end{tabular}
	\caption{Durch die lineare Regression berechnete Werte $A_0,A_0',A_2,A_2'$ mit Fehlern und relativen Fehlern, sowie die quadratischen Abweichungen $s^2,s'^2$ der Regressionsgeraden zu den Datenpunkten}
	\label{tab:Regression}
\end{table}
\begin{figure}
	\centering
	\includegraphics[width=0.8\textwidth]{Tendenz.png}
	\caption{Graphik zur Abschätzung, ob die Absorptionsstelle größer oder kleiner als das betrachtete Spektrum ist}
	\label{Abb:Krummung}
\end{figure} \\
Eigentlich sollte die Entscheidung welche Näherung die bessere ist allein durch Betrachtung der Krümmung der Kurve, die die Messwerte in Abbildung \ref{Abb:Krummung} beschreiben, getroffen werden können. In diesem Fall ist das nicht sehr aussagekräftig, weshalb sich hier nur auf die Abweichungsquadrate verlassen wird. Demnach ist \eqref{Reg:1} die bessere Wahl und die Dispersionsgleichung wird zu
\begin{align*}
	n(\lambda) = \sqrt{3.47 - \SI{1.6e-14}\lambda^{-2}}
\end{align*}
bestimmt. Der Funktionsgraf ist in Abbildung \ref*{fig:DispKurve} zu sehen.
\begin{figure}[h!]
	\centering
	\includegraphics[width=0.8\textwidth]{Dispersionskurve.png}
	\caption{Dispersionskurve}
	\label{fig:DispKurve}
\end{figure}
Für die nächstgelegene Absorptionsstelle $\lambda_1$ gilt also $\lambda_1\gg\lambda_i$. Es wird davon ausgegangen, dass es nur eine Absorptionsstelle gibt. Es folgt dann aus \eqref{Absorptionsstelle}, dass der Brechindex für $\omega\rightarrow\omega_1$ bzw. $\lambda\rightarrow\lambda_1$ gegen 1 geht. Es gilt also für \eqref{Reg:1}:
\begin{align*}
	n^2(\lambda_1) = 1 &= A_0 + \frac{A_2}{\lambda_1^2} \\
	\Leftrightarrow\quad \lambda_1 &= \sqrt{\frac{A_2}{A_0-1}} = \SI{133}{\nano\meter}

\end{align*}
\clearpage
\subsection{Kenndaten des Prismas}
Zuletzt sollen einige Kenndaten des verwendeten Prismas berechnet werden. \\
\ \\
Die Abbesche Zahl ist ein Maß für die Farbzerstreuung bei Gläsern. Sie ist definiert als
\begin{align}
	\nu = \frac{n(\lambda_\text{D})-1}{n(\lambda_\text{F})-n(\lambda_\text{C})} = \SI{-70.789}{}
 \ ,
\end{align}
wobei die $\lambda_i$ die Wellenlängen der Fraunhoferschen Linien $\lambda_\text{D} = \SI{589}{\nano\meter}$, $\lambda_\text{F} = \SI{486}{\nano\meter}$ und $\lambda_\text{C} = \SI{656}{\nano\meter}$ sind. \\
\ \\
Das Auflösungsvermögen bezeichnet den Koeffizienten $\frac{\lambda}{\Delta\lambda}$, der beschreibt, wie nah zwei Spektrallinien liegen dürfen, sodass sie vom Prisma noch getrennt werden können. Für das verwendete Prisma kann das Auflösungsvermögen mit
\begin{align*}
	\frac{\lambda}{\Delta\lambda} = b\frac{\text{d}n}{\text{d}\lambda}
\end{align*}
berechnet werden (Werte für jede Wellenlänge siehe Tabelle \ref{tab:Aufl}). Die Länge $b=\SI{3}{\centi\meter}$ ist die Seitenlänge des Prismas.
\begin{table}[h!]
	\centering
	\begin{tabular}{c|c}
		Wellenlänge in \si{\nano\meter} & Auflösungsvermögen \\
		\hline
		406.2 & 3810 \\
438.7 & 3017 \\
467.8 & 2481 \\
480.0 & 2296 \\
508.6 & 1927 \\
546.1 & 1554 \\
578.0 & 1308 \\
643.9 & 945  \\

	\end{tabular}
	\caption{Auflösungsvermögen bei verschiedenen Wellenlängen}
	\label{tab:Aufl}
\end{table}

\clearpage


\begin{table}
\centering
\begin{tabular}{c|c|c}
	Farben & Quecksilber & Cadmium \\
	\hline
	violett1 & 407.78 / 404.66 & \\
	violett2 & 435.84 & 441.46 \\
	blau &  & 467.82 \\
	türkis & 491.60 & 479.99 \\
	grün & 546.07 & \\
	gelb & 579.07 / 576.95 & \\
	orange &  & 635.99 \\
	rot &  & 643.85 \\
\end{tabular}
\end{table}
Walcher: \\
Cadmium \\
643.85	rot			stark \\
635.99	gelbrot		schwach \\
508.58	grün		stark \\
479.99	blaugrün	stark \\
467.82	blau		stark \\
441.46	blau		mittel \qquad haben wir als violett gesehen \\
\ \\
Quecksilber \\
708.19	rot			schwach \\
690.72	rot			schwach \\
579.07 und 576.95	gelb		sehr stark \\
546.07	grün		stark \\
491.60	blaugrün	mittel \\
435.84	blau		stark \qquad haben wir als violett gesehen \\
407.78 und 404.66	violett		mittel \\