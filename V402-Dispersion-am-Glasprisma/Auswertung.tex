\begin{table}[h!]
	\centering	
	\begin{tabular}{c|c|c}
		$\varphi_1$ & $\varphi_2$ & $\varphi$ \\
		\hline
		376.0 & 266.0 & 55.0 \\
376.0 & 265.5 & 55.2 \\
376.0 & 265.7 & 55.2 \\

	\end{tabular}
	\caption{Messwerte zur Bestimmung des Winkels $\varphi$ des Prismas in \si{\degree}}
	\label{tab:WinkelPhi}
\end{table}
Mittelwert $\varphi = \SI{55.13+-0.06}{}
$
\todo[color=red, inline]{Schreibt man Winkel $\varphi$ in \si{\degree} oder $\varphi$ in Grad? Ich finde ersteres hässlich und zweiteres inkonsistent.}

\begin{table}[h!]
	\centering	
	\begin{tabular}{c|c|c|c}
		Wellenlänge $\lambda$ in \si{\nano\meter} & $\eta_1$ & $\eta_2$ & $\eta$ \\
		\hline
		405.0 & 11.4 & 252.1 & 60.7 \\
440.0 & 12.1 & 252.0 & 59.9 \\
467.0 & 13.1 & 252.9 & 59.8 \\
485.0 & 8.7  & 254.6 & 65.9 \\
546.0 & 14.1 & 254.6 & 60.5 \\
578.0 & 14.6 & 255.1 & 60.5 \\
636.0 & 15.0 & 255.5 & 60.5 \\
644.0 & 11.2 & 255.9 & 64.7 \\

	\end{tabular}
	\caption{Messwerte zur Bestimmung des Brechwinkels $\eta$ in \si{\degree}}
	\label{tab:WinkelFarben}
\end{table}

\begin{table}[h!]
	\centering	
	\begin{tabular}{c|c|c}
		$\lambda$ in \si{\nano\meter} & $n$ & Fehler $\sigma_n$ \\
		\hline
		405.0 & 1.839 & 0.006 \\
440.0 & 1.831 & 0.006 \\
467.0 & 1.830 & 0.006 \\
485.0 & 1.890 & 0.006 \\
546.0 & 1.837 & 0.006 \\
578.0 & 1.837 & 0.006 \\
636.0 & 1.837 & 0.006 \\
644.0 & 1.878 & 0.006 \\

	\end{tabular}
	\caption{Errechnete Brechzahlen $n$ für jede Wellenlänge $\lambda$}
	\label{tab:Brechzahl}
\end{table}

\begin{table}[h!]
	\centering
	\begin{tabular}{c|c|c}
		Größe & Wert & relativer Fehler \\
		\hline
		Steigung bei $\lambda^2$ & \SI{2.2+-3.6e-25}{\meter^{-2}}
 & \SI{1.00}{\meter^{-2}}

		\\
		Steigung bei $\lambda^{-2}$ & \SI{-2.3+-2.8e-14}{\meter\squared}
 & \SI{1.18}{\meter\squared}
 \\
		Achsenabschnitt bei $\lambda^2$ & \SI{4.6826+-0.0063}{}
 & \SI{0.00}{}

		\\
		Achsenabschnitt bei $\lambda^{-2}$ & \SI{3.47+-0.11}{}
 & \SI{0.01}{}
 \\
		\hline
		Abweichungsquadrat bei $\lambda^2$ & \SI{0.082}{}
 & \\
		Abweichungsquadrat bei $\lambda^{-2}$ & \SI{5.371e-14}{}
 & \\
	\end{tabular}
\end{table}
Dispersionskurve
\[ n(\lambda) = \sqrt{3.47 - \SI{1.6e-14}\lambda^{-2}} \]
\begin{figure}[h!]
	\centering
	\includegraphics[width=0.8\textwidth]{Dispersionskurve.png}
	\caption{Dispersionskurve}
	\label{fig:DispKurve}
\end{figure}


Abbesche Zahl \[ \nu = \SI{-70.789}{}
 \]

Auflösungsvermögen
\begin{table}
	\centering
	\begin{tabular}{c|c}
		Wellenlänge in \si{\nano\meter} & Auflösungsvermögen \\
		\hline
		406.2 & 3810 \\
438.7 & 3017 \\
467.8 & 2481 \\
480.0 & 2296 \\
508.6 & 1927 \\
546.1 & 1554 \\
578.0 & 1308 \\
643.9 & 945  \\

	\end{tabular}
\end{table}

Absorptionsstelle \[ \lambda_1 = \sqrt{\frac{Steigung^{-2}}{Verschiebung^{-2}-1}} = \SI{133}{\nano\meter}
 \]


\clearpage


\begin{table}
\centering
\begin{tabular}{c|c|c}
	Farben & Quecksilber & Cadmium \\
	\hline
	violett1 & 407.78 / 404.66 & \\
	violett2 & 435.84 & 441.46 \\
	blau &  & 467.82 \\
	türkis & 491.60 & 479.99 \\
	grün & 546.07 & \\
	gelb & 579.07 / 576.95 & \\
	orange &  & 635.99 \\
	rot &  & 643.85 \\
\end{tabular}
\end{table}
Walcher: \\
Cadmium \\
643.85	rot			stark \\
635.99	gelbrot		schwach \\
508.58	grün		stark \\
479.99	blaugrün	stark \\
467.82	blau		stark \\
441.46	blau		mittel \qquad haben wir als violett gesehen \\
\ \\
Quecksilber \\
708.19	rot			schwach \\
690.72	rot			schwach \\
579.07 und 576.95	gelb		sehr stark \\
546.07	grün		stark \\
491.60	blaugrün	mittel \\
435.84	blau		stark \qquad haben wir als violett gesehen \\
407.78 und 404.66	violett		mittel \\