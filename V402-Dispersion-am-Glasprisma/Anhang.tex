\label{math:ErrorAver}
\label{math:ErrorBrechindex}
\label{math:QuadrAbw}
\label{math:Regression}
Der Mittelwert mehrerer Werte $x_i$ und der Fehler dieses Mittelwerts wurden mit
\begin{align*}
	\bar{x} &=  \frac{1}{N} \sum_{i=1}^\text{N} x_i \\
	\Delta_{x} &= \frac{\sigma_x}{\sqrt{N}}
\end{align*}
berechnet.
\subsubsection*{Fehlerfortpflanzung}
Den Fehler des Brechungsindex
\begin{align*}
	n(\eta,\phi) = \frac{\sin\frac{\eta+\phi}{2}}{\sin\frac{\phi}{2}}
\end{align*}
 erhält man durch die Gauß'sche Fehlerfortpflanzung
\begin{align*}
	\Delta_n &= \sqrt{\left( \frac{\partial n}{\partial \eta} \right)^2 \Delta_\eta^2 
		+ \left( \frac{\partial n}{\partial \phi} \right)^2 \Delta_\phi^2} \\
	&= \sqrt{\left(\frac{\sin\frac{\phi}{2}\cos\frac{\eta+\phi}{2}-\sin\frac{\eta+\phi}{2}\cos\frac{\phi}{2}}{2\sin^2\frac{\phi}{2}}\right)^2 \Delta_\phi^2} \ ,
\end{align*}
wenn systematische Fehler vernachlässigt werden, sodass $\Delta_\phi=0$.
\subsubsection*{Lineare Regression}
Bei der lineare Regression mit den acht Wertepaaren $\{ (X_i^2, n^2(X_i)) \}$ wurde die Steigung
\begin{align*}
	P_2 = \dfrac{8\cdot\sum\limits_{i=1}^8X_i\cdot n^2(X_i)-\sum\limits_{i=1}^8X_i\cdot\sum\limits_{i=1}^8n^2(X_i)}
	{8\cdot\sum\limits_{i=1}^8X_i^2-\left(\sum\limits_{i=1}^8X_i\right)^2}
\end{align*}
und die Verschiebung in $y$-Richtung
\begin{align*}
	P_0 = \dfrac{\sum\limits_{i=1}^8X_i^2\cdot\sum\limits_{i=1}^8n^2(X_i)-\sum\limits_{i=1}^8X_i\cdot\sum\limits_{i=1}^8X_i\cdot n^2(X_i)}
	 {8\cdot\sum\limits_{i=1}^8X_i^2-\left(\sum\limits_{i=1}^8X_i\right)^2}
\end{align*}
verwendet. Die Fehler dieser Werte sind
\begin{align*}
	\Delta_{P_2}^2 = S^2 \cdot\dfrac{8}{8\cdot\sum\limits_{i=1}^8X_i^2-\left(\sum\limits_{i=1}^8X_i\right)^2}
\end{align*}
und
\begin{align*}
	\Delta_{P_0}^2 = S^2 \cdot \dfrac{\sum\limits_{i=1}^8X_i^2}{8\cdot\sum\limits_{i=1}^8X_i^2-\left(\sum\limits_{i=1}^8X_i\right)^2} \ .
\end{align*}
Hierbei wird die quadratische Abweichung
\begin{align*}
	S^2 = \dfrac{\sum\limits_{i=1}^8\left(n^2(X_i)-P_0-P_2X_i\right)^2}{6}
\end{align*}
verwendet. \\
\ \\
Alle verwendeten Formeln stammen aus \cite[Kap. 1.2.10]{Walcher}.