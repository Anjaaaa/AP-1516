Ein LC-Schwingkreis besteht aus einer Spule mit der Induktivität $L$ und einem Kondensator mit der Kapazität $C$. Die Spule speichert ein magnetisches Feld und der Kondensator ein elektrisches. In dem Schwingkreis wechseln sich beide Felder periodisch ab mit der Folge, dass sich der Stromfluss  mit gleicher Periode umkehrt. Zwei LC-Schwingkreise können durch einen weiteren Kondensator $C_\text{K}$ gekoppelt werden (siehe Abbildung (Anleitung 2)). \\
Der Stromfluss in beiden einzelnen gekoppelten Kreisen wird durch die Kirchhoffschen Regeln bestimmt.
\begin{align}
I_\text{K} = I_\text{1}-I_\text{2} \\
U_\text{1C} + U_\text{1L} + U_\text{K} = 0 \\
U_\text{2C} + U_\text{2L} + U_\text{K} = 0
\end{align}
Die gekoppelten Differentialgleichungen für beide Schwingkreise folgen mit
\begin{align}
U_\text{C} = \frac{1}{C} \int I \dd t \quad \text{und} \quad U_\text{L} = L \dv{I}{x} \quad .  \\
\text{1. DGL:} \quad L \dv[2]{I_1}{x} + \frac{1}{C} I_1 + \frac{1}{C_\text{K}}(I_1 + I_2) = 0 \\
\text{2. DGL:} \quad L \dv[2]{I_2}{x} + \frac{1}{C} I_2 - \frac{1}{C_\text{K}} (I_1 + I_2) = 0
\end{align}
Die Lösungen sind abhängig von den Anfangsamplituden $I_{10}$ des erstem Schwingkreises und $I_{20}$ des zweiten Schwingkreises, sowie den Frequenzen $\nu^{+}$ und $\nu^{-} $.
\begin{align}
I_1(t) = \frac{1}{2}(I_{10} + I_{20}) \cos(2 \pi \nu^+ t) + \frac{1}{2}(I_{10} - I_{20}) \cos(2 \pi \nu^- t) \\
I_2(t) = \frac{1}{2}(I_{10} + I_{20}) \cos(2 \pi \nu^+ t) - \frac{1}{2}(I_{10} - I_{20}) \cos(2 \pi \nu^- t)
\end{align}
Die Frequenzen sind
\begin{align}
\nu^+ = \frac{1}{2 \pi \sqrt{L C}} \quad \text{und} \label{Frequenz_p} \\
\nu^- = \frac{1}{2 \pi \sqrt{L\left(\frac{1}{C} + \frac{2}{C_\text{K}}\right)^{-1}}} \quad . \label{Frequenz_m}
\end{align}
Nun werden wichtige Spezialfälle dieses komplexen Verhaltens beschrieben. Die zwei \textbf{Fundamentalschwingungen} zeichnen sich dadurch aus, dass die Anfangsamplituden $I_{10}$ und $I_{20}$ gleich groß sind ($\abs{I_{10}} = \abs{I_{20}}$). Sind beide Schwingkreis in Phase ($I_{10}=I_{20}$) schwingen sie mit der Frequenz $\nu+$. Sind die Schwingkreise um eine halbe Periode phasenverschoben ($I_{10} = - I_{20}$) schwingen sie mit der etwas höheren Frequenz $\nu-$. \\
Das Phänomen der \textbf{Schwebung} tritt auf, wenn einer der Kreise stimuliert wird bzw. eine Anfangsamplitude hat und der andere Kreis keine Anfangsamplitude aufweist und die Frequenzen $\nu^+$ und $\nu^-$ nahezu übereinstimmen. \\
\todo{Ich finde du wiederholst dich hier, wenn du sagst, dass die Frequenzen nahezu übereinstimmen.}
\todo[color = red]{Wieso hast du bei einem Strom (t) dahinter geschrieben und beim anderen nicht?}
\begin{align}\label{Schwebung}
	I_1 = I_{10}  \cos \left( \pi (\nu^+ + \nu^-) t \right) \cdot  \cos \left( \pi (\nu^+ - \nu^-) t \right) \\
	I_2 (t) = I_{10}  \sin \left( \pi (\nu^+ + \nu^-) t \right) \cdot  \sin \left( \pi (\nu^+ - \nu^-) t \right) 
\end{align}
Es entsteht eine Schwingung (siehe Abbildung (Abb.3))  mit den Frequenzen der einzelnen Schwingkreisen 
\begin{equation}
	\nu = \pi (\nu^+ + \nu^-)
\end{equation}
und einer Schwebungsfrequenz von 
\todo[inline]{Ich glaube hier müsste ein Minus sein und die 2 weg.}
\begin{equation}
	f = 2 \pi (\nu^+ + \nu^-) \quad .
\end{equation}
Es handelt sich um einen periodischen Energieaustausch zwischen den beiden Schwingkreisen mit der Schwebungsfrequenz.
 