Die Bestimmung der Fundamentalfrequenzen über die Phasenbeziehung ergibt die in Tabelle \ref{fig:Freq_Phase} aufgelisteten Messdaten.
\begin{table}[h!]
\begin{center}
\begin{tabular}{c | c | c}
	$C_\text{K}$ in \SI{}{\nano\farad} & $\nu_{\varphi=0}^+$ in \SI{}{\kilo\hertz} & $\nu_{\varphi=\pi}^-$ in \SI{}{\kilo\hertz} \\
\hline
	1.0 & 33.11 & 79.62 \\
	2.2 & 32.47 & 60.39 \\
	2.7 & 32.47 & 56.56 \\
	4.7 & 32.68 & 48.08 \\
	6.8 & 32.47 & 44.01 \\
	8.2 & 32.89 & 41.67 \\
	10.0 & 33.11 & 40.65 \\
	12.0 & 32.89 & 40.00 \\
\end{tabular}
\end{center}
\caption{Fundamentalfrequenzen bei der Bestimmung über die Phasenbeziehung für verschiedene Koppelkondensatoren $C_\text{K}$}
\label{fig:Freq_Phase}
\end{table} \\
Bei der \glqq dynamischen\grqq\ Messung der Fundamentalfrequenzen wird die am Schwingkreis angelegte Frequenz kontinuierlich erhöht. Die Zeitdifferenz zwischen dem Beginn der Frequenzsteigerung und dem erreichen der beiden Fundamentalfrequenzen wird am Oszilloskop bestimmt werden. Diese Zeiten sind in Tabelle \ref{fig:Zeiten} dargestellt.
Anton & 23.15 & 13.5 & 0.5  & 9.20  & 0.19  \\
Berta & 54.53 & 6.9  & 0.1  & 5.13  & 0.12  \\
Cäsar & 14.06 & 8.4  & 0.2  & 4.28  & 0.61  \\
Dora & 27.27 & 8.7  & 0.1  & 6.47  & 0.11  \\
Emil & 36.67 & 5.8  & 0.6  & 3.52  & 0.10  \\
Friedrich & 17.16 & 10.0 & 0.2  & 4.364 & 0.063 \\
Gustav & 15.60 & 3.52 & 0.08 & 2.483 & 0.080 \\
Heinrich & 25.52 & 2.39 & 0.05 & 2.027 & 0.065 \\
Ida & 18.38 & 6.0  & 0.1  & 3.384 & 0.061 \\
Julius & 24.50 & 2.77 & 0.05 & 2.026 & 0.068 \\
Kaufmann & 32.18 & 3.8  & 0.2  & 2.75  & 0.12  \\
Ludwig & 13.21 & 8.8  & 0.2  & 4.004 & 0.088 \\
Martha & 21.87 & 4.10 & 0.06 & 2.94  & 0.13  \\
Nordpol & 21.20 & 3.70 & 0.09 & 3.207 & 0.053 \\

Über die Startfrequenz $\nu_\text{start} = \SI{11.16}{\kilo\hertz}$, die Endfrequenz $\nu_\text{end} = \SI{112.6}{\kilo\hertz}$ und das eingestellte Intervall $T = \SI{1}{\second}$ kann eine Funktion
\[ \nu(t) = \frac{\nu_\text{end}-\nu_\text{start}}{T}t+\nu_\text{start} \]
aufgestellt werden. Damit können nun die Fundamentalfrequenzen (siehe Tabelle \ref{fig:Freq_dyn}) berechnet werden. \\
\begin{table}[h!]
\begin{center}
\begin{tabular}{c | c | c}
	$C_\text{K}$ in \SI{}{\nano\farad} & $\nu_{t_1}^+$ in \SI{}{\kilo\hertz} & $\nu_{t_2}^-$ in \SI{}{\kilo\hertz} \\
\hline
	1.0 & 34.29 & 79.75 \\
	2.2 & 33.48 & 60.67 \\
	2.7 & 33.07 & 56.61 \\
	4.7 & 32.67 & 47.27 \\
	6.8 & 33.07 & 44.03 \\
	8.2 & 33.07 & 42.00 \\
	10.0 & 33.06 & 40.78 \\
	12.0 & 32.67 & 39.16 \\
\end{tabular}
\end{center}
\caption{Fundamentalfrequenzen, die sich bei der \glqq dynamischen\grqq\ Messung für verschiedene Koppelkondensatoren $C_\text{K}$ ergeben}
\label{fig:Freq_dyn}
\end{table}

\ \\
\ \\
Das Zählen der Schwingungen innerhalb einer Schwebungsperiode, also einem Schwingungsbauch hat nur teilweise funktioniert. Bei den kleineren Kapazitäten war dies (siehe Tabelle \ref{fig:Freq_Bauch}) nicht möglich. \\
Mit \eqref{Schwebung} kann die Anzahl an Schwingungen pro Schwebungsbauch in Zusammenhang mit den Fundamentalfrequenzen gebracht werden. So schwingt das System $n$-mal
\todo[color=green]{Endlich verstehe ich das. Da n nicht ganzzahlig sein muss, funktioniert das super!}
\[ \pi(\nu^+ + \nu^-)T \overset{!}{=} 2\pi n \]
während einer Schwebungsperiode (also einem Schwebungsbauch)
\[ \pi(\nu^- - \nu^+)T \overset{!}{=} \pi \ . \]
Zusammen ergibt sich
\begin{align}\label{Bauch}
	n = \frac{(\nu^+ + \nu^-)}{2(\nu^- - \nu^+)} \ .
\end{align}

Die Resonanzfrequenz des Schwingkreises bei einer Phasenverschiebung von $\varphi=0$ ist $\nu^+=\SI{33.33}{\kilo\hertz}$. Damit kann nun $\nu^-$ berechnet werden. Die so erhaltenen Werte ergänzen Tabelle \ref{fig:Freq_Bauch}.
\begin{table}[h!]
\begin{center}
\begin{tabular}{c | c | c}
	Kapazität $C_\text{K}$ in \SI{}{\nano\farad} & Bäuche & $\nu^-$ in \SI{}{\kilo\hertz} \\
	\hline
	4.7 & 2 & 41.66 \\
	6.8 & 3 & 38.89 \\
	8.2 & 4 & 37.50 \\
	10.0 & 4 & 37.50 \\
	12.0 & 5 & 36.66 \\
\end{tabular}
\end{center}
\caption{Anzahl der Schwingungen pro Schwingungsbauch und die dazu berechneten Frequenzen $\nu^-$}
\label{fig:Freq_Bauch}
\end{table} \\
Über Beziehung \eqref{Bauch} kann auch die Anzahl an Schwingungen pro Schwebungsperiode bei bekannten Frequenzen berechnet werden. Ein Vergleich zwischen der gezählten, der für die Frequenzen aus dem ersten Versuchsteil berechneten und der theoretisch zu erwartenden Anzahl an Bäuchen pro Schwebung befindet sich in Tabelle \ref{fig:Bauch} in Kapitel \ref{sec:Diskussion}.
\todo[inline]{Super Asuwertung. Gefällt mir sehr gut. Mir ist nichts aufgefallen, das fehlen würde.}