Die Tabellen \ref{fig:Bauch}, \ref{fig:FreqPlus} und \ref{fig:FreqMinus} stellen die Ergebnisse der drei Messreihen noch einmal gegenüber. Die angegebenen Theorie-Werte werden durch Einsetzen der Kenndaten der Bauteile (siehe Kapitel \ref{sec:Bauteile}) in \eqref{Frequenz_p}, \eqref{Frequenz_m} bzw. \eqref{Bauch} berechnet. \\
\ \\
Die verschiedenen Messungen der Fundamentalfrequenz $\nu^+$ ergeben sehr ähnliche Werte. Da sie nicht vom Koppelkondensator abhängt, werden hier die Mittelwerte
\[ \bar{x} = \frac{1}{N-1} \sum_{i=1}^\text{N} x_i \]
mit ihren Fehlern
\[ \Delta_{x} = \frac{\sigma_x}{\sqrt{N}} = \frac{\sqrt{\frac{1}{N} \sum_{i=1}^\text{N} (x_i - \bar{x})^2}}{\sqrt{N}} \]
miteinander verglichen. Die Werte erhalten so mehr Aussagekraft, da nicht nur einzelne, sondern mehrere Messwerte mit eingehen. Zu den Werten ist zu sagen, dass eine Abweichung von 7-8\% vertretbar wäre, jedoch sind die Fehler der Mittelwerte sehr klein, sodass hier doch ein größerer systematischer Fehler vorliegen muss.
\begin{table}[h!]
\begin{center}
\begin{tabular}{c | c | c | c | c}	
	& Messung über & dynamische & Messung & Theorie \\
	& Phasenbeziehung & Messung & beim Kalibrieren & \\
\hline
	Wert & \SI{33.17(17)} & \SI{32.76(9)} & 33.33 & 36.66 \\
\hline
	Abweichung & -9.5\% & -10.6\% & -9.1\% &  \\
\end{tabular}
\end{center}
\caption{Mittelwerte der Messungen von $\nu^+$ in \si{\kilo\hertz} und Abweichungen vom theoretisch erwarteten Wert}
\label{fig:FreqPlus}
\end{table} \\
Auch bei der Messung der Frequenz $\nu^-$ liegen die Messwerte sehr dich beieinander. Die Abweichung vom Theorie-Wert jedoch steigert sich von knapp 5\% bei großen Koppelkapazitäten $C_\text{K}$ bis etwa 36\% bei kleinen $C_\text{K}$. Die über die Anzahl der Schwingungen pro Schwebungsbauch berechneten Frequenzen $\nu^-$ dagegen weichen nur wenig von den erwarteten Werten ab. Hier gilt allerdings zu beachten, dass bei der Messung schon die Anzahl des zweiten Schwebungsbauches deutlich geringer, als der des ersten war, sodass nur die Schwingungen pro einem Bauch gezählt wurden. Der systematische Fehler der Zählung der Schwingungen ist demnach sehr groß und die gezählten Werte nicht sehr aussagekräftig. \\
\todo[]{Ich fände es gut, wenn noch dastehen würde, dass die Schwingungszahl pro Bauch stark unterschiedlich war.}
\todo[color = red]{Meintest du das so?}
\todo[color=green]{Ja. Ist gut}
\begin{table}[h!]
\begin{center}
\begin{tabular}{c | c | c | c | c}
	$C_\text{K}$ in \SI{}{\nano\farad} & Messung über & dynamische & berechnet über Anzahl der & Theorie \\
	& Phasenbeziehung & Messung & Schwingungen pro Schwebung & \\
\hline
	1.0 & 79.62 & 79.75 & - & 56.21 \\
	2.2 & 60.39 & 60.67 & - & 46.51 \\
	2.7 & 56.56 & 56.61 & - & 44.77 \\
	4.7 & 48.08 & 47.27 & 41.67 & 41.26 \\
	6.8 & 44.01 & 44.03 & 38.89 & 39.69 \\
	8.2 & 41.67 & 42.00 & 37.50 & 39.07 \\
	10.0 & 40.65 & 40.78 & 37.50 & 38.52 \\
	12.0 & 40.00 & 39.16 & 36.66 & 38.10 \\
\end{tabular}
\end{center}
\caption{Werte der Messungen von $\nu^-$ in \si{\kilo\hertz}}
\label{fig:FreqMinus}
\end{table}
 \\
Bei der Anzahl der Schwingungen pro Schwebungsbauch liegt ein ähnliches Bild vor: Die errechneten Werte unterscheiden sich nur wenig. Sie weichen aber allesamt stark vom erwarteten Wert ab, bei großen Koppelkapazitäten $C_\text{K}$ sogar um über 60\%.
\begin{table}[h!]
\begin{center}
\begin{tabular}{c | c | c | c | c}
	$C_\text{K}$ in \SI{}{\nano\farad} & gezählt & berechnet, dynamisch & berechnet, Phase & erwartet, Theorie \\
\hline
	1.0 & - & 1.3 & 1.2 & 3.4 \\
	2.2 & - & 1.7 & 1.7 & 6.3 \\
	2.7 & - & 1.9 & 1.8 & 7.5 \\
	4.7 & 2 & 2.7 & 2.6 & 12.4 \\
	6.8 & 3 & 3.5 & 3.3 & 17.5 \\
	8.2 & 4 & 4.2 & 4.2 & 21.0 \\
	10.0 & 4 & 4.8 & 4.9 & 25.3 \\
	12.0 & 5 & 5.5 & 5.1 & 30.2 \\
\end{tabular}
\end{center}
\caption{Gezählte, berechnete und erwartete Anzahl an Schwingungen pro Schwebungsbauch für verschiedene Koppelkondensatoren $C_\text{K}$}
\label{fig:Bauch}
\end{table}

\todo[inline]{Hier müsste jetzt noch etwas dazu stehen, warum die Werte so stark von der Theorie abweichen. Aber um ehrlich zu sein: Ich habe keine Ahnung.}
\todo[inline, color = red]{Wie kommst du  auf die Theoriewerte?}
\todo[inline, color = green]{Hab ich oben hinzugefügt. So verständlich?}
\todo[inline, color = green]{Hab ich oben hinzugefügt. So verständlich?}
