Niels Bohr entwickelte 1913 sein berühmtes Atommodell. Es war das erste Atommodell, dass Ideen der Quantenmechanik enthielt. Im selben Jahr gelang Franck und Hertz der experimentelle Nachweis, dass Atome diskrete Energiezustände haben. Dieser sogenannte Frack-Hertz-Versuch war eine der wichtigsten Bestätigungen des Bohrschen Modells. Dieses Protokoll beschreibt die Durchführung des Franck-Hertz-Versuchs im Anfänger Praktikum. \cite[Kap. 13.4 und 14.4.1]{Walcher}