\subsubsection*{Kapitel \ref{sec:Aufbau_Grund}: Ideale Franck-Hertz-Kurve}
Warum muss die Franck-Hertz-Kurve so aussehen? \\
Der steile Abfall ist klar, aber der Teil davor ist unklar: \\
\begin{enumerate}
	\item $|U_\text{A}|<\SI{4.9}{\volt}$: \\
	Die Elektronen werden ausgelöst und haben keine kinetische Energie. $U_\text{B}$ wird erhöht, ist aber kleiner als $U_\text{A}$, es wird also kein Strom gemessen. Sobald $|U_\text{B}|=|U_\text{A}|<\SI{4.9}{\volt}$ ist, würden sofort alle Elektronen das Gegenfeld überwinden können und als Strom gemessen. Da immer konstant viele Elektronen aus dem Glühdraht ausgelöst werden, kann dieser Strom auch nicht größer werden. Erst bei $U_\text{B}=\SI{4.9}{\volt}$ käme dann der abrupte Abfall, bis der Strom bei $U_\text{B}=\SI{4.9}{\volt}+U_\text{A}$ wieder abrupt ansteigt. Mit diesen Überlegungen sähe die  Kurve aus, wie die einer Rechteckspannung.
	\item $U_\text{A}>\SI{4.9}{\volt}$: \\
	Hier würde doch überhaupt kein Strom gemessen oder?
\end{enumerate}
Warum werden die Peaks immer höher?
\subsubsection*{Kapitel \ref{sec:Aufbau}: Kontaktpotential}
Warum sollte $\Phi_B$ groß sein? Weil da sonst auch Elektronen rauskämen?
\subsubsection*{Kapitel \ref{sec:Spektrum}: Leitungselektronen vs $E_1-E_0$}
In der Anleitung und demnach auch in diesem Theorieteil steht, dass die Anregungsenergie die Energie $E_1-E_0$ ist. Das suggeriert, das ein Elektron von der innersten Schale auf die zweit-innerste angeregt wird. Kann das überhaupt gehen? Die zweite Schale ist doch voll besetzt, da ist doch gar kein Platz mehr? Andererseits, wäre eine Anregung von der äußersten besetzten Schale in die darüber viel logischer und du hast bei einem Kommentar auch von Leitungselektronen gesprochen. Das sind doch die Valenzelektronen oder nicht? \\
Die Frage ist: Welches Elektron wird von wo nach wo angeregt?