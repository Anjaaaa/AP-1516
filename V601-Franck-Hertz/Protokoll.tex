\documentclass[a4,10pt]{article}
\usepackage{geometry}
\geometry{a4paper, top=40mm, left=30mm, right=30mm, bottom=35mm}
\setlength\parindent{0pt}
\usepackage[german]{babel}
\usepackage[utf8]{inputenc}


\usepackage{multicol} % Spalten
\usepackage{color} % Farben
\usepackage[hyphens]{url} % Internetadresse (mit automatischer Trennung)
\usepackage{enumitem} % Aufzählungen


% Literaturverzeichnis
\usepackage{csquotes}
\usepackage{biblatex}
\addbibresource{../Vorlage/Literatur.bib}


% Mathesymbole
\usepackage{amsmath, amsthm, amssymb}
\usepackage{bm} % fette Schrift in Matheumgebung
\usepackage{physics} % Derivate richtig schreiben
\usepackage[version=4]{mhchem} % Chemische Elemente
\usepackage[
	locale=DE,
	separate-uncertainty=true, % Fehler mit ±
	per-mode=symbol-or-fraction, % m/s im Text, sonst \frac
	% alternativ:
	% 	per-mode=reciprocal,
	% 	m s^{-1}
	output-decimal-marker=., % . statt , für Dezimalzahlen
	]{siunitx} % si-Einheiten


% Bilder
\usepackage{caption}
\usepackage{graphicx, wrapfig}
\usepackage{subcaption} % Bilder in Gruppe einzeln benennen
\usepackage{tikz}
\usetikzlibrary{matrix} % zeigt Koordinatensystem an
\usepackage[european]{circuitikz} % Schaltkreise
\usetikzlibrary{arrows}
\newcommand{\mymeter}[2] % Option um Schaltsymbole zu drehen
{  % #1 = name , #2 = rotation angle
	\begin{scope}[transform shape,rotate=#2]
		\draw[thick] (#1)node(){$\mathbf V$} circle (11pt);
		\draw[rotate=45,-latex] (#1)  +(-17pt,0) --+(17pt,0);
	\end{scope}
}



% Kopf- und Fußzeile
\usepackage{fancyhdr}
\pagestyle{fancy}
\renewcommand{\sectionmark}[1]{\markright{#1}}
\renewcommand{\subsectionmark}[1]{\markright{#1}}
\fancyhead{} % Default-Einstellungen im Header löschen
\fancyhead[L]{\sc{Versuch \V}}
\fancyhead[R]{\sc{\rightmark}}





\newcommand{\Versuch}{Franck-Hertz-Versuch}
\newcommand{\Betreuer}{Max \textsc{Mustermann}}
\newcommand{\Tag}{17.05.16}
\newcommand{\V}{V601}

% Bibliographie erstellen:
% 	pdflatex Protokoll.tex
% 	biber Protokoll
% 	pdflatex Protokoll.tex
% oder einfach make ausführen


\usepackage[
	disable,
colorinlistoftodos,
linecolor=yellow,
backgroundcolor=yellow,
textwidth=0.15\textwidth,
textsize=footnotesize
]{todonotes} % Notizen

\begin{document}
\begin{titlepage}
	\par
	\raisebox{-.5\height}{\includegraphics[height=1cm]{../Vorlage/Logo-TUDo.png}}
	\hfill
	\raisebox{-.5\height}{\includegraphics[height=1cm]{../Vorlage/Logo-Physik.png}}
	\par
\begin{center}
\ \\
[5.5cm]	
	\textsc{\Huge Anfängerpraktikum 2015/2016} \\
[1.5cm]
	\Huge\textbf{\Versuch} \\
[1cm]
	{\large Durchführung: \Tag} \\
	{\large \Korrektur} \\
[4.5cm]
\begin{minipage}{0.4\textwidth}
	\begin{flushleft} \large
		Clara \textsc{Rittmann}\textsuperscript{1} \\
		Anja \textsc{Beck}\textsuperscript{2}
	\end{flushleft}
\end{minipage}
\hfill
\begin{minipage}{0.4\textwidth}
	\begin{flushright} \large
		\emph{Betreuer:} \\
		\Betreuer
	\end{flushright}
\end{minipage}
\end{center}
\footnotetext[1]{clara.rittmann@tu-dortmund.de}
\footnotetext[2]{anja.beck@tu-dortmund.de}
\end{titlepage}


\tableofcontents
\clearpage


\section{Einleitung}
Niels Bohr entwickelte 1913 sein berühmtes Atommodell. Es war das erste Atommodell, dass Ideen der Quantenmechanik enthielt. Im selben Jahr gelang Franck und Hertz der experimentelle Nachweis, dass Atome diskrete Energiezustände haben. Dieser sogenannte Frack-Hertz-Versuch war eine der wichtigsten Bestätigungen des Bohrschen Modells. Dieses Protokoll beschreibt die Durchführung des Franck-Hertz-Versuchs im Anfänger Praktikum. \cite[Kap. 13.4 und 14.4.1]{Walcher}
\section{Theorie}
Ein LC-Schwingkreis besteht aus einer Spule mit der Induktivität $L$ und einem Kondensator mit der Kapazität $C$. Die Spule speichert ein magnetisches Feld und der Kondensator ein elektrisches. In dem Schwingkreis wechseln sich beide Felder periodisch ab mit der Folge, dass sich der Stromfluss  mit gleicher Periode umkehrt. Zwei LC-Schwingkreise können durch einen weiteren Kondensator $C_\text{K}$ gekoppelt werden (siehe Abbildung (Anleitung 2)). \\
Der Stromfluss in beiden einzelnen gekoppelten Kreisen wird durch die Kirchhoffschen Regeln bestimmt.
\begin{align}
I_\text{K} = I_\text{1}-I_\text{2} \\
U_\text{1C} + U_\text{1L} + U_\text{K} = 0 \\
U_\text{2C} + U_\text{2L} + U_\text{K} = 0
\end{align}
Die gekoppelten Differentialgleichungen für beide Schwingkreise folgen mit
\begin{align}
U_\text{C} = \frac{1}{C} \int I \dd t \quad \text{und} \quad U_\text{L} = L \dv{I}{x} \quad .  \\
\text{1. DGL:} \quad L \dv[2]{I_1}{x} + \frac{1}{C} I_1 + \frac{1}{C_\text{K}}(I_1 + I_2) = 0 \\
\text{2. DGL:} \quad L \dv[2]{I_2}{x} + \frac{1}{C} I_2 - \frac{1}{C_\text{K}} (I_1 + I_2) = 0
\end{align}
Die Lösungen sind abhängig von den Anfangsamplituden $I_{10}$ des erstem Schwingkreises und $I_{20}$ des zweiten Schwingkreises, sowie den Frequenzen $\nu^{+}$ und $\nu^{-} $.
\begin{align}
I_1(t) = \frac{1}{2}(I_{10} + I_{20}) \cos(2 \pi \nu^+ t) + \frac{1}{2}(I_{10} - I_{20}) \cos(2 \pi \nu^- t) \\
I_2(t) = \frac{1}{2}(I_{10} + I_{20}) \cos(2 \pi \nu^+ t) - \frac{1}{2}(I_{10} - I_{20}) \cos(2 \pi \nu^- t)
\end{align}
Die Frequenzen sind
\begin{align}
\nu^+ = \frac{1}{2 \pi \sqrt{L C}} \quad \text{und} \label{Frequenz_p} \\
\nu^- = \frac{1}{2 \pi \sqrt{L\left(\frac{1}{C} + \frac{2}{C_\text{K}}\right)^{-1}}} \quad . \label{Frequenz_m}
\end{align}
Nun werden wichtige Spezialfälle dieses komplexen Verhaltens beschrieben. Die zwei \textbf{Fundamentalschwingungen} zeichnen sich dadurch aus, dass die Anfangsamplituden $I_{10}$ und $I_{20}$ gleich groß sind ($\abs{I_{10}} = \abs{I_{20}}$). Sind beide Schwingkreis in Phase ($I_{10}=I_{20}$) schwingen sie mit der Frequenz $\nu+$. Sind die Schwingkreise um eine halbe Periode phasenverschoben ($I_{10} = - I_{20}$) schwingen sie mit der etwas höheren Frequenz $\nu-$. \\
Das Phänomen der \textbf{Schwebung} tritt auf, wenn einer der Kreise stimuliert wird bzw. eine Anfangsamplitude hat und der andere Kreis keine Anfangsamplitude aufweist und die Frequenzen $\nu^+$ und $\nu^-$ nahezu übereinstimmen. \\
\todo{Ich finde du wiederholst dich hier, wenn du sagst, dass die Frequenzen nahezu übereinstimmen.}
\todo[color = red]{Wieso hast du bei einem Strom (t) dahinter geschrieben und beim anderen nicht?}
\begin{align}\label{Schwebung}
	I_1 = I_{10}  \cos \left( \pi (\nu^+ + \nu^-) t \right) \cdot  \cos \left( \pi (\nu^+ - \nu^-) t \right) \\
	I_2 (t) = I_{10}  \sin \left( \pi (\nu^+ + \nu^-) t \right) \cdot  \sin \left( \pi (\nu^+ - \nu^-) t \right) 
\end{align}
Es entsteht eine Schwingung (siehe Abbildung (Abb.3))  mit den Frequenzen der einzelnen Schwingkreisen 
\begin{equation}
	\nu = \pi (\nu^+ + \nu^-)
\end{equation}
und einer Schwebungsfrequenz von 
\todo[inline]{Ich glaube hier müsste ein Minus sein und die 2 weg.}
\begin{equation}
	f = 2 \pi (\nu^+ + \nu^-) \quad .
\end{equation}
Es handelt sich um einen periodischen Energieaustausch zwischen den beiden Schwingkreisen mit der Schwebungsfrequenz.
 
\clearpage


\section{Aufbau und Ablauf des Experiments}
Der Versuch besteht aus zwei Teilen. \\
Zunächst wird für drei verschiedene Spannungsverläufe (Rechteck, Sägezahn und Dreieck) eine Fourier-Synthese durchgeführt. Dazu wird ein Schwingungsgenerator benutzt, der die ersten zehn Komponenten einer Fourier-Reihe generieren kann. Es müssen jeweils die passenden Koeffizienten eingestellt und überprüft werden, dass die Schwingungen alle in Phase sind. \\
Im zweiten Teil wird ein Funktionsgenerator an ein Oszilloskop angeschlossen. Mit Hilfe der \textsc{math}-Funktion des Oszilloskops wird dann eine Fourier-Transformation für verschiedene Spannungen (Rechteck, Sägezahn und Dreieck) durchgeführt. Nun können Ort und Amplitude der Pieks abgelesen werden.
\clearpage


\section{Auswertung}
\subsection{Statistische Formeln}
\subsubsection{Fehlerrechnung}
\label{sec:Fehlerrechnung}
Die Zeiten in Tabelle \ref{fig:Zeiten} und die berechnete Elementarladung in \eqref{eq:Elementarladung} sind Mittelwerte. Sie werden mit
\begin{align}
	\bar{x} =  \frac{1}{N-1} \sum_{i=1}^\text{N} x_i
\end{align}
berechnet. Ihre Fehler sind
\begin{align}
	\Delta x = \sqrt{\frac{\sum_{i=1}^\text{N} (x_i - \bar{x})^2}{N(N-1)}} \ .
\end{align}
Die Fehler der Geschwindigkeiten in Tabelle \ref{fig:Vel} und der Radien und Ladungen in den Tabellen \ref{fig:RQ_unkorr} und \ref{fig:RQ_korr} werden mit der Gaußschen Fehlerfortpflanzung berechnet. Für die Geschwindigkeiten ist das
\begin{align}
	\Delta v &= \sqrt{\left(\frac{\partial v}{\partial s}\right)^2(\Delta s)^2 + \left(\frac{\partial v}{\partial t}\right)^2(\Delta t)^2} \\
	&= \frac{s}{t^2}\Delta t
\end{align}
und für die Radien
\begin{align}
	\Delta r &= \sqrt{\left(\frac{\partial r}{\partial v_\text{ab}}\right)^2(\Delta v_\text{ab})^2 + \left(\frac{\partial r}{\partial v_\text{auf}}\right)^2(\Delta v_\text{auf})^2} \\
	&= \frac{3}{4}\frac{\sqrt{\frac{\eta_\text{L}}{g\cdot\rho_\text{Oel}}}}{\sqrt{(v_\text{ab}-v_\text{auf})}} \sqrt{(\Delta v_\text{ab})^2 + (\Delta v_\text{auf})^2} \ .
\end{align}
Der Fehler der nicht korrigierten Ladungen ist
\begin{align}
	\Delta q &= \sqrt{\left(\frac{\partial q}{\partial v_\text{ab}}\right)^2(\Delta v_\text{ab})^2 + \left(\frac{\partial q}{\partial v_\text{auf}}\right)^2(\Delta v_\text{auf})^2} \\
	&= \frac{9}{2}\frac{d\pi\eta_\text{L}}{U}\sqrt{\frac{\eta_\text{L}}{g\cdot\rho_\text{Oel}}(v_\text{ab}-v_\text{auf})}
	\sqrt{\left(1+\frac{v_\text{ab}+v_\text{auf}}{2(v_\text{ab}-v_\text{auf}) }\right)^2(\Delta v_\text{ab})^2
	+\left(1-\frac{v_\text{ab}+v_\text{auf}}{2(v_\text{ab}-v_\text{auf})}\right)^2(\Delta v_\text{auf})^2}
\end{align}
und der Fehler der korrigierten Ladungen
\begin{align}
	\Delta q_\text{korr} &= \sqrt{\left(\frac{\partial q_\text{korr}}{\partial q}\right)^2(\Delta q)^2 + \left(\frac{\partial q_\text{korr}}{\partial r}\right)^2(\Delta r)^2} \\
	&= \sqrt{\left(1+\frac{B}{pr}\right)^3(\Delta q)^2 + \left(1+\frac{B}{pr}\right) \left(\frac{3}{2}\frac{qB}{pr^2}\right)^2(\Delta r)^2} \ .
\end{align}
%\subsubsection{Regression}
%\label{sec:regression}
%Nachfolgend wird eine lineare Regression für Wertepaare $(x_i,y_i)$ durchgeführt. Dafür müssen die Steigung
\begin{equation}
	m = \dfrac{
		n\cdot\sum\limits_{i}x_iy_i-\sum\limits_{i}x_i\cdot\sum\limits_{i}y_i
		}
		{n\cdot\sum\limits_{i}x_i^2-\left(\sum\limits_{i}x_i\right)^2
		}
\end{equation}
und der y-Achsenabschnitt
\begin{equation}
	b = \dfrac{
		\sum\limits_{i}x_i^2\cdot\sum\limits_{i}y_i-\sum\limits_{i}x_i\cdot\sum\limits_{i}x_iy_i
		}{
		n\cdot\sum\limits_{i}x_i^2-\left(\sum\limits_{i}x_i\right)^2
		}
\end{equation}
berechnet werden. Die jeweiligen Fehler sind
\begin{equation}
	s_m^2 = s_y^2 \cdot \dfrac{n}{n\cdot\sum\limits_{i}x_i^2-\left(\sum\limits_{i}x_i\right)^2}
\end{equation}
\begin{equation}
	s_b^2 = s_y^2 \cdot \dfrac{\sum\limits_{i}x_i^2}{n\cdot\sum\limits_{i}x_i^2-\left(\sum\limits_{i}x_i\right)^2}\ .
\end{equation}
$s_y$ ist hierbei die Abweichung der Regressionsgeraden in y-Richtung
\begin{equation}
	s_y^2 = \dfrac{\sum\limits_{i}\left(\Delta y_i\right)^2}{n-2} = \dfrac{\sum\limits_{i}\left(y_i-b-mx_i\right)^2}{n-2} \ .
\end{equation}
%\clearpage
Alle Messungen werden mit einem x-y-Schreiber aufgezeichnet. Die Graphen, die dieser Auswertung zu Grunde liegen, sind im Anhang~\ref{sec:anhang} zu sehen.
Messung 1 (Abb.~\ref{fig:messung1})  und Messung 2 (Abb.~\ref{fig:messung2}) gehen in Kapitel~\ref{sec:auswertung1} ein, wobei Messung 1 zusätzlich in Kapitel~\ref{sec:auswertung4} verwendet wird. Messung 3 (Abb.~\ref{fig:messung3})-- die Frank-Hertz-Kurve --  wird im Auswertungsteil~\ref{sec:auswertung3} analysiert. Messung 4 (Abb.~\ref{fig:messung4})wird in Kapitel~\ref{sec:auswertung4}
 benötigt.
\subsection{Statistische Formeln}
\subsubsection{Fehlerrechnung}
\label{sec:Fehlerrechnung}
Die Zeiten in Tabelle \ref{fig:Zeiten} und die berechnete Elementarladung in \eqref{eq:Elementarladung} sind Mittelwerte. Sie werden mit
\begin{align}
	\bar{x} =  \frac{1}{N-1} \sum_{i=1}^\text{N} x_i
\end{align}
berechnet. Ihre Fehler sind
\begin{align}
	\Delta x = \sqrt{\frac{\sum_{i=1}^\text{N} (x_i - \bar{x})^2}{N(N-1)}} \ .
\end{align}
Die Fehler der Geschwindigkeiten in Tabelle \ref{fig:Vel} und der Radien und Ladungen in den Tabellen \ref{fig:RQ_unkorr} und \ref{fig:RQ_korr} werden mit der Gaußschen Fehlerfortpflanzung berechnet. Für die Geschwindigkeiten ist das
\begin{align}
	\Delta v &= \sqrt{\left(\frac{\partial v}{\partial s}\right)^2(\Delta s)^2 + \left(\frac{\partial v}{\partial t}\right)^2(\Delta t)^2} \\
	&= \frac{s}{t^2}\Delta t
\end{align}
und für die Radien
\begin{align}
	\Delta r &= \sqrt{\left(\frac{\partial r}{\partial v_\text{ab}}\right)^2(\Delta v_\text{ab})^2 + \left(\frac{\partial r}{\partial v_\text{auf}}\right)^2(\Delta v_\text{auf})^2} \\
	&= \frac{3}{4}\frac{\sqrt{\frac{\eta_\text{L}}{g\cdot\rho_\text{Oel}}}}{\sqrt{(v_\text{ab}-v_\text{auf})}} \sqrt{(\Delta v_\text{ab})^2 + (\Delta v_\text{auf})^2} \ .
\end{align}
Der Fehler der nicht korrigierten Ladungen ist
\begin{align}
	\Delta q &= \sqrt{\left(\frac{\partial q}{\partial v_\text{ab}}\right)^2(\Delta v_\text{ab})^2 + \left(\frac{\partial q}{\partial v_\text{auf}}\right)^2(\Delta v_\text{auf})^2} \\
	&= \frac{9}{2}\frac{d\pi\eta_\text{L}}{U}\sqrt{\frac{\eta_\text{L}}{g\cdot\rho_\text{Oel}}(v_\text{ab}-v_\text{auf})}
	\sqrt{\left(1+\frac{v_\text{ab}+v_\text{auf}}{2(v_\text{ab}-v_\text{auf}) }\right)^2(\Delta v_\text{ab})^2
	+\left(1-\frac{v_\text{ab}+v_\text{auf}}{2(v_\text{ab}-v_\text{auf})}\right)^2(\Delta v_\text{auf})^2}
\end{align}
und der Fehler der korrigierten Ladungen
\begin{align}
	\Delta q_\text{korr} &= \sqrt{\left(\frac{\partial q_\text{korr}}{\partial q}\right)^2(\Delta q)^2 + \left(\frac{\partial q_\text{korr}}{\partial r}\right)^2(\Delta r)^2} \\
	&= \sqrt{\left(1+\frac{B}{pr}\right)^3(\Delta q)^2 + \left(1+\frac{B}{pr}\right) \left(\frac{3}{2}\frac{qB}{pr^2}\right)^2(\Delta r)^2} \ .
\end{align}


\subsection{Anzahl der Stöße in der Röhre}
Für das Gelingen des Versuches ist es wichtig, dass weder zu viele noch zu wenige Elektronen mit Quecksilberatomen zusammenstoßen. Bei zu wenig Stößen treten kaum Wechselwirkungen auf, die beobachtet werden sollen. Gibt es hingegen zu viele Stöße, sind einige elastisch unter Änderung der Richtung, sodass sie nicht an der Kathode ankommen und fälschlicherweise nicht zu Strom beitragen. \\
Aus den Temperaturen in Kelvin lässt sich nach Gleichung \eqref{eq:weglange} und \eqref{eq:dampfdruck} der Sättigungsdruck und daraus die freie Weglänge d.h. die Strecke, die ein Teilchen ohne Kollisionen zurücklegt, berechnet. Die Ergebnisse sind in Tabelle \ref{tab:temperaturen} dargestellt.



\begin{figure}[h!]
	\centering
	\captionof{table}{Anzahl der Stöße in der \SI{1}{\centi\meter} langen Röhre}
	\begin{tabular}{cccc}
		Temperatur / \si{\kelvin} & Sättigungsdruck / \si{\milli\bar} & freie Weglänge / \si{\micro\meter} & Anzahl Stöße \\
		\hline
		298.15 & 0.01  & 5468.00 & 1.83    \\
413.15 & 3.25  & 8.91    & 1122.17 \\
453.15 & 14.14 & 2.05    & 4876.12 \\
373.15 & 0.55  & 53.06   & 188.48  \\

	\end{tabular}
	\label{tab:temperaturen}
\end{figure}




\subsection{Differentielle Energieverteilung} \label{sec:auswertung1}
Aus der Messung des Stromes bei veränderlicher Bremsspannung $U_A$ kann die Energieverteilung der der austretenden Elektronen bestimmt werten
\begin{align}
	E(U_A) \sim \frac{I(U_A)-I(U_{A+1})}{U_A - U_{A+1}}
\end{align}

Strom und Spannung sind bekannt. Sie werden mit Hilfe eines x-y-Schreibers aufgenommen. Es werden zwei Messreihen für die Temperaturen T = \SI{25}{\celsius} und T = \SI{140}{\celsius} erstellt. Die Werte des Stromes für T = \SI{25}{\celsius} sind in Tabelle~\ref{tab:stromverlauf_25} zu finden und die Steigung  (Tabelle~\ref{tab:energieverteilung_25}) in Abbildung~\ref{fig:energieverteilung_25} grafisch dargestellt, während die Daten zu T = \SI{140}{\celsius} in den Tabellen~\ref{tab:stromverlauf_140} und~\ref{tab:energieverteilung_140} und der Abbildung~\ref{fig:energieverteilung_140}  stehen. \\
Aus der Energieverteilung bei Raumtemperatur kann das Kontaktpotential bestimmt werden (siehe Kapitel~\ref{cap:kontaktpotential} ).




\begin{figure}
	\centering
	\captionof{table}{Strom in Abhängigkeit der Spannung  (T = \SI{25}{\celsius})}
	\begin{tabular}{cc}
		Spannung / \si{\volt} & Strom /  \si{\nano\ampere}   \\
		\hline
		0.00 & 3800 \\
0.18 & 3747 \\
0.61 & 3614 \\
1.05 & 3561 \\
1.49 & 3401 \\
1.93 & 3269 \\
2.37 & 3136 \\
2.81 & 3003 \\
3.25 & 2870 \\
3.68 & 2750 \\
4.12 & 2617 \\
4.56 & 2471 \\
5.00 & 2338 \\
5.44 & 2166 \\
5.88 & 1993 \\
6.32 & 1807 \\
6.75 & 1608 \\
7.19 & 1382 \\
7.63 & 1116 \\
8.07 & 771  \\
8.29 & 531  \\
8.51 & 213  \\
8.73 & 53   \\
8.95 & 0    \\

	\end{tabular}
	\label{tab:stromverlauf_25}
\end{figure}

\begin{figure}
	\centering
	\captionof{table}{Steigung es Stromverlaufes (T = \SI{25}{\celsius}) und zugehörige Spannungswerte}
	\begin{tabular}{cc}
		Spannung / \si{\volt} & Strom pro Spannung / \si{\nano\ampere\per\volt}   \\
		\hline
		0.00 & 302.94  \\
0.18 & 302.94  \\
0.61 & 121.17  \\
1.05 & 363.52  \\
1.49 & 302.94  \\
1.93 & 302.94  \\
2.37 & 302.94  \\
2.81 & 302.94  \\
3.25 & 272.64  \\
3.68 & 302.94  \\
4.12 & 333.23  \\
4.56 & 302.94  \\
5.00 & 393.82  \\
5.44 & 393.82  \\
5.88 & 424.11  \\
6.32 & 454.41  \\
6.75 & 514.99  \\
7.19 & 605.87  \\
7.63 & 787.64  \\
8.07 & 1090.57 \\
8.29 & 1454.10 \\
8.51 & 727.05  \\
8.73 & 242.35  \\

	\end{tabular}
	\label{tab:energieverteilung_25}
\end{figure}


\begin{figure}
	\centering
	\includegraphics[width=0.9\textwidth]{build/Energieverteilung_25.png}
	\caption{Energieverteilung bei  T = \SI{25}{\celsius} ist proportional zur Steigung des Stromes}
	\label{fig:energieverteilung_25}
\end{figure}

\begin{figure}
	\centering
	\captionof{table}{Strom in Abhängigkeit der Spannung  (T = \SI{140}{\celsius})}
	\begin{tabular}{cc}
		Spannung / \si{\volt} & Strom /  \si{\nano\ampere}   \\
		\hline
		0.00 & 110 \\
0.22 & 97  \\
0.44 & 88  \\
0.66 & 81  \\
0.88 & 73  \\
1.11 & 65  \\
1.33 & 58  \\
1.55 & 51  \\
1.77 & 45  \\
1.99 & 38  \\
2.21 & 32  \\
2.43 & 26  \\
2.65 & 20  \\
2.88 & 15  \\
3.10 & 12  \\
3.32 & 9   \\
3.54 & 6   \\
3.76 & 4   \\
3.98 & 3   \\

	\end{tabular}
	\label{tab:stromverlauf_140}
\end{figure}

\begin{figure}
	\centering
	\captionof{table}{Steigung es Stromverlaufes (T = \SI{140}{\celsius}) und zugehörige Spannungswerte}
	\begin{tabular}{cc}
		Spannung / \si{\volt} & Strom pro Spannung / \si{\nano\ampere\per\volt}   \\
		\hline
		0.00 & 57.82 \\
0.22 & 40.48 \\
0.44 & 34.69 \\
0.66 & 34.69 \\
0.88 & 34.69 \\
1.11 & 34.69 \\
1.33 & 28.91 \\
1.55 & 28.91 \\
1.77 & 28.91 \\
1.99 & 28.91 \\
2.21 & 28.91 \\
2.43 & 23.13 \\
2.65 & 23.13 \\
2.88 & 17.35 \\
3.10 & 11.56 \\
3.32 & 11.56 \\
3.54 & 11.56 \\
3.76 & 5.78  \\

	\end{tabular}
	\label{tab:energieverteilung_140}
\end{figure}


\begin{figure}
	\centering
	\includegraphics[width=0.9\textwidth]{build/Energieverteilung_140.png}
	\caption{Energieverteilung bei  T = \SI{140}{\celsius} ist proportional zur Steigung des Stromes}
	\label{fig:energieverteilung_140}
\end{figure}

\clearpage

\subsection{Erste Anregungsenergie des Quecksilberatoms}\label{sec:auswertung2}
Die erste Anregungsenergie wird aus den Abständen der Maxima der Frank-Hertz-Kurve bestimmt. Die Spannungsdifferenzen sind in Tabelle~\ref{tab:anregungsspannung} dargestellt. Ihr Mittelwert und deren Fehler ist

\begin{align}
	U_{\text{Mittel}} = \SI{4.95+-0.18}{\volt}
 \quad .
\end{align}

\begin{figure}[h!]
	\centering
	\captionof{table}{Die Differenzen der Strommaxima der Frank-Hertz-Kurve entsprechen der Anregungsspannung des Quecksilberatoms}
	\begin{tabular}{c}
		Spannung / \si{\volt}   \\
		\hline
		4.6216 \\
5.2297 \\
4.9865 \\

	\end{tabular}
	\label{tab:anregungsspannung}
\end{figure}

Aus der Anregungsspannung kann mit Hilfe den Formeln \eqref{eq:energie1} und \eqref{eq:energie2} die Anregungsenergie und daraus die Wellenlänge bestimmt werden.
Für die Anregungsenergie ergibt sich der Wert
\begin{align}
	\Delta E = \SI{7.92+-0.28e-19}{\joule}

\end{align}
und für die Wellenlänge 
\begin{align}
	\lambda = \SI{250.7+-9.0}{\nano\meter}
 \quad .
\end{align}
Die Fehler werden jeweils mit Hilfe der Gaußschen Fehlerfortpflanzung berechnet
\begin{align}
	\sigma_{\Delta E} = \abs{e_0 \sigma_{U_\text{Mittel}}}
\end{align}
und
\begin{align}
	\sigma_\lambda = \abs{\frac{h c_0}{e_0} \frac{1}{U^2_{\text{Mittel}}} \sigma_{U_\text{Mittel}}} \quad .
\end{align} 



\subsubsection{Bestimmung der Kontaktpotentials} \label{sec:auswertung3}
\label{cap:kontaktpotential}
\underline{Methode 1:} \\
Die Frank-Hertz-Kurve ist genau um das Kontaktpotential $K$ nach rechts verschoben. Das erste Maximum liegt bei
\begin{align}
	U_1 = \SI{8.0256}{\volt} \quad .
\end{align}
Folglich ist das Kontaktpotential
\begin{align}
K_1 = U_1 - U_{\text{Mittel}} = \SI{3.08+-0.18}{\volt}
\end{align}
\underline{Methode 2:} \\
In Abbildung \ref{fig:energieverteilung_25} ist erkennbar, dass bei einer Bremsspannung $U_{A, \text{max}}$ von \SI{8.15}{\volt} Elektronen die größte Energie haben. Das Kontaktpotential ist die Differenz der Bremsspannung bei größter Energie und der konstanten Beschleunigungsspannung $U_B = \SI{11}{\volt} $.
\begin{align}
	K_2 = U_B - U_{A, \text{max}} =  \SI{2.85}{\volt} 
\end{align}

\subsubsection{Bestimmung der Ionisierungsspannung}\label{sec:auswertung4}
Die Ionisierungsspannung $U_\text{ION}$ ergibt sich aus der letzten Messung des Stromes bei einer höheren Gegenspannung von \SI{30}{\volt}.  Die Asymptote des 1. Maximums wird aufgezeichnet. Die Spannung, an der der Strom stark zunimmt
\begin{align}
	U'_1 = \SI{13}{\volt}
\end{align}
 setzt sich zusammen aus der Ionisierungsspannung und dem Kontaktpotential $K$, wobei für K der Mittelwert der in Kapitel \ref{cap:kontaktpotential} bestimmten Werte $K_1$ und $K_2$
 \begin{align}
 K = \frac{K_1+K_2}{2} = \SI{2.965}{\volt}
 \end{align}
 angenommen wird. 
\begin{align}
	U_\text{ION} = U'_1 - K = \SI{10.035}{\volt}
\end{align}




\clearpage


\section{Diskussion}
In diesem Versuch gibt es drei Hauptfehlerquellen: Die Messgeräte werden als ideale Messgeräte angenommen. Das bedeutet, dass das Voltmeter einen unendlich hohen und das Ampèremeter einen unendlich kleinen Widerstand hat, um den Stromfluss nicht zu verändern. \\ Des weiteren gibt es eine Ableseungenauigkeit der Messgeräte. Diese wurden in den Rechnungen größtenteils berücksichtigt. \\ Auch vernachlässigt werden sogenannte Rückkopplungseffekte.  Ändert sich der Belastungsstrom, so beeinflusst das eigentlich den Innenwiderstand und die Leerlaufspannung der Quelle. \\
\begin{center}
	\captionof{table}{Messergebnisse Leerlaufspannung Monozelle und Gegenspannung}
\begin{tabular}{c|c|c|c}
	& Gemessener Wert & Erwarteter Wert & Abweichung \\
	\hline
	Monozelle & \SI{1.471}{\volt} & \SI{1.5}{\volt} & - 2.0 \% \\
	Gegenspannung & \SI{1.383}{\volt} & \SI{1.5}{\volt}  & - 7.8 \%
\end{tabular}
\end{center}
Wie in Kapitel \ref{systhematischer_fehler} gezeigt ist, kann die Abweichung nicht am Voltmeter liegen. Wahrscheinlich sind die Rückkopplungseffekte der entscheidende Faktor. \\
Die Messung mit dem RC-Generator als Stromquelle liefert niedrigere Leerlaufspannungen und höhere Innenwiderstände.
Der errechnete Innenwiederstand für die Rechteckspannung
\begin{equation}
R_{i, \text{Rechteck}} = \SI{61}{\ohm}
\end{equation} 
stimmt mit dem Literaturwert\footnote{Elektronikpraktikum, H. Pfeiffer, S. 55} von circa \SI{50}{\ohm} überein, zumal dieser Wert noch vom Gerät abhängig ist. Wieso der Innenwiderstand des Generators beim Erzeugen einer Sinusspannung so viel höher ist 
\begin{equation}
R_{i,\text{Sinus}} = \SI{0.68}{\kilo\ohm}
\end{equation}
können wir nur vermuten. Das Rechtecksignal wird in ein Dreiecksignal umgewandelt und diese beiden zusammen ergeben die Sinusspannung. Vielleicht werden hierzu große Widerstände benötigt. Oder der Innenwiderstand hängt von der Steigung des abgegebenen Signals ab, die bei der Rechteckspannung null ist und bei der Sinusspannung zwischen null und eins liegt. \\
Die Werte, die für die Leitung berechnet wurden sind sehr gut. Lediglich ein Wertepaar liegt mit seinen Fehlerbalken nicht auf der erwarteten Funktion. Ein Leistungsmaximum wird für $R_a = R_i$ erreicht, da gilt:
\begin{align*}
	\dv{N(R_a)}{R_a} &= \dv{}{R_a}\left(\left(\frac{U_0}{R_i+R_a}\right)^2R_a\right) = \frac{U_0^2(R_i-R_a)}{(R_i+R_a)^3}
	\overset{!}{=} 0 \quad  \\
	&\Rightarrow R_a = R_i \ .
\end{align*}

\clearpage
\listoftodos
\listoffigures
\listoftables
\clearpage
\nocite{\V}
\printbibliography[title = Literaturverzeichnis]

\end{document}
