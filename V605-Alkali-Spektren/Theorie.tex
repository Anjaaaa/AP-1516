In diesem Versuch wird die innere Abschirmzahl von Alkali-Atomen anhand ihrer Spektren bestimmt. \\
Das Coulomb-Feld der positiv geladenen Atomkerns wird nach außen von den negativ geladenen Elektronen abgeschirmt. Die resultierende Ladung $z_\text{eff}$, die ein Elektron in der äußersten Schicht erfährt, wird durch die Abschirmzahl $\simga$ bestimmt
\begin{align}
z_\text{eff} = z - \sigma
\end{align}
Alkali-Atome bieten den Vorteil, dass sie in der äußersten Schicht nur ein Elektron haben, was Leuchtelektron genannt wird, während die übrigen Elektronen der gefülten Schlafen Rumpfelektronen heißen. Dadurch kann eine Ein-Atom Näherung zur Bestimmung der Energien $E$ unterschiedlicher Elektronenkonfigurationen durchgeführt werden. Hierzu werden Lösungen der Schrödingergleichung
\begin{align}
\left(\sum_i \frac{\hat{p}_i^2}{2m_i} + U \right) \psi = E \psi
\end{align}
mit der Masse $m$,  dem Impulsoperator $\hat{p}$ und dem Potential des ein-atomigen Wasserstoff-Atoms
\begin{align}
	U = -\frac{e_0^2}{4 \pi \epsilon_0 r} \quad .
\end{align}
Es wird berücksichtigt, dass die Alkali-Atome $z$ Protonen im Kern haben, die um den Wert $\sigma$ abgeschirmt werden. Zudem fließen relativistische Effekte mit ein. Für die Energien ergibt sich
\begin{align}
	E = - R_\infty \left( \frac{(z-\sigma)^2}{n^2} + \alpha^2 \frac{(z-\sigma)^4}{n^3}\left(\frac{2}{2 l + 1}  - \frac{3}{4n}  - \frac{j(j+1) - l(l+1)-\frac{3}{4}}{l(l+1)(2l+1) }\right) \right)
\end{align}
mit der Ryberg-Energie $R_\infty$, der Quantenzahl $n$, der Bahndrehimpulszahl $l$ und dem Gesamtdrehimpuls $j$. Der Gesamtdrehimpuls setzt sich zusammen aus dem Bahndrehimpuls und dem dazu parallelen oder antiparallelen Spin, der für Elektronen $\frac{1}{2}$ beträgt d.h. $j = l + \frac{1}{2}$ oder $j = l - \frac{1}{2}$. Laut der Auswahlregeln folgt eine zusätzliche Bedingung für $l$: $\Delta l = \pm 1$.
Es folgt:
\begin{align}
	E = - R_\infty \left( \frac{(z-\sigma_1)^2}{n^2} + \alpha^2 \frac{(z-\sigma_2)^4}{n^3}\left(\frac{1}{j+\frac{1}{2}}  - \frac{3}{4n}\right) \right)
\end{align}
Die Abschirmzahl wird hier aufgeteilt in die Konstante der vollständigen Abschirmung $\sigma_1$ zu der sämtliche Rumpfelektronen beitragen und der inneren Abschirmung $\sigma_2$ zu der nur das Leuchtelektron beiträgt. Die Abschirmzahl wird bestimmt, indem Energiedifferenzen aus ein und der selben Schale betrachtet werden, die sich nur durch den Spin unterscheiden
\begin{align}
	\Delta E_D = \frac{R_\infty\alpha^2}{n^3}(z-\sigma_2)^4\frac{1}{l(l+1)}
\end{align}
Die zugehörigen emittierten Spektrallinien liegen im sichtbaren Bereich und unterscheiden sich nur so gering, dass sie als Dublett bezeichnet werden. Die Energiedifferenz wird über die Wellenlängen bestimmt
\begin{align}
	\Delta E_D = hc\left(\frac{1}{\lambda}-\frac{1}{\lambda'}\right) \approx hc\frac{\Delta \lambda}{\lambda^2}
\end{align}
An einem Beugungsgitter ($k =$ Ordnung der Hauptmaximums, $p =$ Spaltenanzahl des Gitters) werden die Strahlen gemäß
\begin{align}
	\sin(\varphi) = k\frac{\lambda}{g}
\end{align}
gebrochen.
Die Wellenlängendifferenz $\Delta \lambda$ ist genau
\begin{align}
	\Delta \lambda &= \Xi \cos(\varphi)\Delta s
\end{align}	
mit
\begin{align}
	\Xi &= \frac{\Delta\lambda}{\Delta t\cdot\cos\overline{\varphi}} \ , \quad \overline{\varphi} = \frac{1}{2}(\phi_1 + \phi_2) \quad ,
\end{align}
wobei $\Delta s$ und $\Delta t$, wie in Kapitel \ref{sec:aufbau} näher beschrieben ist,
\begin{align}
	\Delta s = r \Delta \phi , \quad \Delta t = r (\phi_1 - \phi_2)
\end{align}
betragen.
\todo[inline, color=red]{Was genau ist hier die Eichkonstante}