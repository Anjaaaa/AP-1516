Die bestimmten Werte für die Abschirmungszahl wichen beträchtlich von den Literaturwerten \cite{deGruyter} ab. Bedenkt man aber, dass ein gewisser systematischer Fehler durch die menschliche Komponente bei der Messung beim Drehen des Gitters und dem Ablesen der Werte nie ausgeschlossen werden kann und zudem jeder Wert nur einmal aufgenommen wurde, liegen die Abweichungen noch in einem annehmbaren Bereich.
\begin{table}[h!]
	\centering
	\caption{Vergleich der berechneten Abschirmzahlen $\sigma_2$ mit Literaturwerten}
	\label{tab:Literatur}
%	\sisetup{parse-numbers=false}
	\begin{tabular}{l|ccc}
		\toprule
		Element & $\sigma_2$ & Literaturwert & Abweichung \\
		\midrule
		Natrium & 4.14 & 7.46 & -\SI{55.5}{\%} \\
		Kalium & 8.02 & 13.06 & -\SI{61.4}{\%} \\
		Rubidium & 18.43 & 26.95 & \SI{68.4}{\%} \\
		\bottomrule
	\end{tabular}
\end{table}

