Die bestimmten Werte für die Abschirmungszahl weichen stark von den Literaturwerten \cite{deGruyter} ab. Natürlich kann das an den allgemeinen systematischen Fehlern liegen, wie ungenaues Ablesen bzw. durch die Aufnahme von nur einem Wert pro Messung. Allerdings liegen sie für jedes Element sehr dicht an der Kernzahl, siehe Tabelle \ref{tab:Literatur}, sodass vermutet wird, dass in den Rechnungen irgendwo ein größerer Fehler stecken muss. Wo dieser liegt kann auch nach ausgiebiger Prüfung nicht festgestellt werden.
\begin{table}[h!]
	\centering
	\caption{Vergleich der berechneten Abschirmzahlen $\sigma_2$ mit Literaturwerten und der Kernzahl $z$}
	\label{tab:Literatur}
	\sisetup{parse-numbers=false}
	\begin{tabular}{l|lll}
		\toprule
		Element & $z$ & $\sigma_2$ & Literaturwert \\
		\midrule
		Natrium & 11 & 10.999995421 & 7.46 \\
		Kalium & 19 & 18.99999267 & 13.06 \\
		Rubidium & 37 & 36.99998760 & 26.95 \\
		\bottomrule
	\end{tabular}
\end{table}

