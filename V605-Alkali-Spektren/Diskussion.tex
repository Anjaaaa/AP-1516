Selbstverständlich kam es zu systematischen Fehlern bei der Messung, beim Drehen des Gitters und dem Ablesen der Werte. Jedoch weichen die bestimmten Werte für die Abschirmungszahl kaum von den Literaturwerten \cite{deGruyter} ab, sodass bei der verwendeten Methode diese Fehler wohl sehr gering bleiben.
\begin{table}[h!]
	\centering
	\caption{Vergleich der berechneten Abschirmzahlen $\sigma_2$ mit Literaturwerten}
	\label{tab:Literatur}
%	\sisetup{parse-numbers=false}
	\begin{tabular}{l|ccc}
		\toprule
		Element & $\sigma_2$ & Literaturwert & Abweichung \\
		\midrule
		Natrium & 7.330 & 7.46 & \SI{-1.8}{\%} \\
		Kalium & 13.12 & 13.06 & \SI{-3.7}{\%} \\
		Rubidium & 27.06 & 26.95 & \SI{0.4}{\%} \\
		\bottomrule
	\end{tabular}
\end{table}

