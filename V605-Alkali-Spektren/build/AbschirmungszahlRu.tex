\begin{table}[h!]
    \centering
    \caption{Rubidium ($n=5,z=37$) -- Abschirmungszahl für das betrachtete Dublett, sowie bei der Berechnung verwendete Größen}
    \label{tab:Rubidium}
    \sisetup{parse-numbers=false}
    \begin{tabular}{
	S[table-format=3.0]
	@{${}\pm{}$}
	S[table-format=1.0, table-number-alignment = left]
	S[table-format=3.1]
	@{${}\pm{}$}
	S[table-format=1.1, table-number-alignment = left]
	S[table-format=3.0]
	S[table-format=3.0]
	@{${}\pm{}$}
	S[table-format=1.0, table-number-alignment = left]
	S[table-format=2.2]
	@{${}\pm{}$}
	S[table-format=1.2, table-number-alignment = left]
	}
	\toprule
	\multicolumn{2}{c}{$\lambda \ \mathrm{ in } \ \si{\nano\meter}$}		& \multicolumn{2}{c}{$\Delta\lambda \ \mathrm{ in } \ \si{\nano\meter}$}		& 
	{$\Delta s \ \mathrm{in} \ \mathrm{Skt}$}		& \multicolumn{2}{c}{$\Delta E_\text{D} \ \mathrm{ in } \ \si{\milli\electronvolt}$}		& 
	\multicolumn{2}{c}{$\sigma_2$}		\\ 
	\midrule
    631 & 2 & 110.7 & 0.6 & 644 & 345 & 2 & 36.99998760 & 0.00000002 \\

    \bottomrule
    \end{tabular}
    \end{table}
