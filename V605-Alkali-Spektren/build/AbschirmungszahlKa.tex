\begin{table}[h!]
    \centering
    \caption{Kalium ($n=4,z=19$) -- Abschirmungszahl für jedes betrachtete Dublett, sowie bei der Berechnung verwendete Größen}
    \label{tab:Kalium}
    \sisetup{parse-numbers=false}
    \begin{tabular}{
	S[table-format=3.0]
	@{${}\pm{}$}
	S[table-format=1.0, table-number-alignment = left]
	S[table-format=2.1]
	@{${}\pm{}$}
	S[table-format=1.1, table-number-alignment = left]
	S[table-format=3.0]
	S[table-format=2.1]
	@{${}\pm{}$}
	S[table-format=1.1, table-number-alignment = left]
	S[table-format=2.8]
	@{${}\pm{}$}
	S[table-format=1.8, table-number-alignment = left]
	}
	\toprule
	\multicolumn{2}{c}{$\lambda \ \mathrm{ in } \ \si{\nano\meter}$}		& \multicolumn{2}{c}{$\Delta\lambda \ \mathrm{ in } \ \si{\nano\meter}$}		& 
	{$\Delta s \ \mathrm{in} \ \mathrm{Skt}$}		& \multicolumn{2}{c}{$\Delta E_\text{D} \ \mathrm{ in } \ \si{\milli\electronvolt}$}		& 
	\multicolumn{2}{c}{$\sigma_2$}		\\ 
	\midrule
    587 & 2 & 22.2 & 0.1 & 130 & 79.8 & 0.6 & 18.99999273 & 0.00000001 \\
585 & 2 & 22.2 & 0.1 & 130 & 80.2 & 0.6 & 18.99999272 & 0.00000001 \\
537 & 2 & 20.0 & 0.1 & 119 & 86.0 & 0.8 & 18.99999259 & 0.00000002 \\
536 & 2 & 19.4 & 0.1 & 115 & 83.6 & 0.7 & 18.99999264 & 0.00000002 \\

    \bottomrule
    \end{tabular}
    \end{table}
