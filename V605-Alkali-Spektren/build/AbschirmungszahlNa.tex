\begin{table}
    \centering
    \caption{Natrium ($n=3,z=11$) -- Abschirmungszahl für jedes betrachtete Dublett, sowie bei der Berechnung verwendete Größen}
    \label{tab:Natrium}
    \sisetup{parse-numbers=false}
    \begin{tabular}{
	S[table-format=3.0]
	@{${}\pm{}$}
	S[table-format=1.0, table-number-alignment = left]
	S[table-format=2.2]
	@{${}\pm{}$}
	S[table-format=1.2, table-number-alignment = left]
	S[table-format=2.0]
	S[table-format=2.1]
	@{${}\pm{}$}
	S[table-format=1.1, table-number-alignment = left]
	S[table-format=2.9]
	@{${}\pm{}$}
	S[table-format=1.9, table-number-alignment = left]
	}
	\toprule
	\multicolumn{2}{c}{$\lambda \ \mathrm{in} \ \si{\nano\meter}$}		& \multicolumn{2}{c}{$\Delta\lambda \ \mathrm{in} \ \si{\nano\meter}$}		& 
	{$\Delta s \ \mathrm{in} \ \mathrm{Skt}$}		& \multicolumn{2}{c}{$\Delta E_\text{D} \ \mathrm{ in } \ \si{\milli\electronvolt}$}		& 
	\multicolumn{2}{c}{$\sigma_2$}		\\ 
	\midrule
    621 & 2 & 8.93  & 0.05 & 52 & 28.7 & 0.2 & 4.20 & 0.01 \\
594 & 2 & 10.25 & 0.05 & 60 & 36.0 & 0.3 & 3.80 & 0.01 \\
572 & 2 & 6.63  & 0.03 & 39 & 25.1 & 0.2 & 4.42 & 0.01 \\

    \bottomrule
    \end{tabular}
    \end{table}
