Wesentlicher Bestandteil des Experiments ist ein Geiger-Müller-Zählrohr, das die Anzahl der Zerfälle registriert. Ein Nachgeschalteter Zähler registriert die verstärkten Pulse in einem wählbaren Zeitintervall. Das Zeitintervall darf werder zu lang noch zu kurz gewählt werden, um statistische und systematische Fehler zu vermeiden. \\
Zu Beginn des Experiments wird eine Messung ohne Probe durchgeführt, um den Nulleffekt, d.h. die natürliche Anzahl an Pulsen, zu bestimmen. \\
Danach wird Brom eine halbe Stunde lang in drei minütigen Zeitintervallen gemessen und Silber eine Stunde in .... Zeitintervallen.

\todo[inline]{Wie waren die Zeitintervalle bei Silber? Hm, das ist auch so kurz...}