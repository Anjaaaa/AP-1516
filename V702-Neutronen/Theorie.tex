In der Natur kommen stabile ebenso wie instabile Atomkerne vor. Instabile Atomkerne zerfallen solange, bis eine stabile Konfiguration. Die Geschwindigkeit des Kernzerfalls schwankt stark und ist über die Halbwertszeit, d.h. die Zeit bis die Hälfte aller Kerne zerfallen ist, definiert. Der Zerfall von $N_0$ Teilchen in $N$ Teilchen wird durch
\begin{align}
	N(t) = N_0 \operatorname{e}^{- \lambda t}
\end{align}
mit der Zerfallskonstante $\lambda$ beschrieben. Und die Halbwertszeit ist entsprechend
\begin{align}
	T = \frac{\ln 2}{\lambda } \quad .
\end{align}

Ein Kern zerfällt, wenn die Anzahl von Neutronen zu Protonen ungünstig ist. Stabile Kerne weisen eine Neutronenzahl auf, die etwa \SI{20}{\percent} bis \SI{50}{\percent} über der Anzahl der Protonen liegt. In diesem Experiment werden Elemente mit einer sehr kurzen Halbwertszeit betrachtet. Ein zusätzlich in den Kern eingefügtes Neutron aktiviert diesen zu einem Zwischenkern und führt zu einem Beta-Zerfall. Ein Neutron~n zerfällt in ein Proton unter Aussendung eines Elektrons~$\beta^-$ und zusätzlicher Strahlung~$\nu^-_\text{e}$. Die Strahlung ist eine Folge des Massendefizits: Proton und Elektron gemeinsam sind leichter als ein Neutron, die Differenz der Massen entspricht der Energie der Strahlung.
In diesem Experiment wird der Zerfall von Brom in Krypton und Silber (\SI{52.3}{\percent} Isotop $\ce{^{107}_{47}Ag}$ und \SI{48.7}{\percent} Isotop $\ce{^{109}_{47}Ag}$ ) in Cadmium betrachtet:
\begin{align}
	& \ce{^{79}_{35}Br} + \text{n} \rightarrow 	\ce{^{80}_{35}Br} \rightarrow  \ce{^{80}_{36}Kr} + \beta^- + \nu^-_\text{e} \\
	& \ce{^{107}_{47}Ag} + \text{n} \rightarrow 	\ce{^{108}_{47}Ag} \rightarrow  \ce{^{108}_{48}Cd} + \beta^- + \nu^-_\text{e} \\
	& \ce{^{109}_{47}Ag} + \text{n} \rightarrow 	\ce{^{110}_{47}Ag} \rightarrow  \ce{^{110}_{48}Cd} + \beta^- + \nu^-_\text{e}
	\end{align}
	
	Die eingefügten Neutronen sollen möglichst langsam sein. Je länger sich das Neutron im Kern befindet, desto höher ist die Wahrscheinlichkeit einer Wechselwirkung. Die Neutronen werden  durch den Beschuss von Beryllium mit $\alpha$-Teilchen gewonnen und müssen anschließen abgebremst werden. Das geschieht in einer Paraffin-Schicht durch elastische Stöße an Protonen.
	
	\todo[inline]{Irgendwie habe ich schon vergessen, was Björn zur Energieminimierung alles erzählt hat. Vielleicht gucke ich das nochmal nach un änder dann die Theorie. Sie kommt mir so kurz vor...}
	
	



%An Formeln wäre super: \\
%T = ln(2)/lambda \\
%Zerfallsgesetz \\
%evtl. die Zerfallsgleichungen von Brom und Silber (in Aufbau, Ablauf), die können aber auch in die Auswertung
