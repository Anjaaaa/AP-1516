In diesem Versuch wurde qualitativ erfolgreich gezeigt, dass Einsteins Theorie der Lichtkorpuskeln den Photoeffekt beschriebt. Es ist nichts beobachtet worden, was diesen Erklärungen widerspricht. \\
Bei der Bestimmung der Gegenspannung (Abschnitt \ref{sec:gegenspannung}) mussten jeweils zwei Messwerte ausgelassen werden, um den Bereich einer linearen Steigung nicht zu verlassen.  Die Ungenauigkeiten der aus den Fit-Parametern errechneten Grenzspannung sind sehr gering. Nur die rote Spektralline zeigt eine Ungenauigkeit der Grenzspannung, die doppelt so groß ist, wie der Wert selbst. diese Linie ist nur schwach sichtbar.
Die quantitative Genauigkeit der Messung kann lediglich über das hier berechnete Planksche Wirkungsquantum (Abschnitt \ref{sec:h_durch_h}) im Vergleich mit Literaturwerten überprüft werden.\\
\begin{table}[h!]
	\centering
	\captionof{table}{Vergleich mit Literaturwerten \cite[Tabellen-Anhang, Tab. A 1.4]{Walcher}}	
	\begin{tabular}{c|c|c}
	gemessen & Literaturwert & Abweichung \\
		\hline
		\SI{3.89e-15}{\electronvolt} & 	\SI{4.1356692e-15}{\electronvolt} & \SI{-5.9}{\percent}
	
	\end{tabular}
	\label{tab:Vergleich}
\end{table}

Die Abweichung kann über einige äußere Einflüsse auf das Experiment erklärt werden. Wenn es in der Umgebung ein bisschen heller wird, zum Beispiel da an den umliegenden Versuchen Lampen verwendet werden, steigt der Photonenstrom direkt an. Des weiteren ist das Photoelement beweglich und verrutscht manchmal, was dazu führt, dass der Lichtstrahl nicht komplett einfällt. \\
Aus der gleichen linearen Regression, mit der das Planksche Wirkungsquantum bestimmt wird, ergibt sich auch die Austrittsarbeit \eqref{eq:austrittsarbeit}. Sie liegt ungefähr zwischen \num{1} und \SI{2}{\electronvolt}. Laut Tabellenwerk\footnote{\url{http://www.chemie.de/lexikon/Austrittsarbeit.html}} (VERWEIS) kommen entweder Caesium (\num{1.7}-\SI{2,14 }{\electronvolt}), \mbox{Barium~(\num{1.8}-\SI{2,52 }{\electronvolt})}, Bariumoxid (\SI{1}{\electronvolt}) oder Strontiumoxid (\SI{1}{\electronvolt}) als Kathodenmaterial in Frage. 