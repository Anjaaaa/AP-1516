In diesem Versuch wurde qualitativ erfolgreich gezeigt, dass Einsteins Theorie der Lichtkorpuskeln den Photoeffekt beschriebt. Es ist nichts beobachtet worden, was diesen Erklärungen widerspricht. \\
Die quantitative Genauigkeit der Messung kann lediglich über das hier berechnete Planksche Wirkungsquantum im Vergleich mit Literaturwerten überprüft werden. \\
\begin{table}[h!]
	\centering
	\captionof{table}{Vergleich mit Literaturwerten}	
	\begin{tabular}{c|c|c}
	gemessen & Literaturwert & Abweichung \\
		\hline
		\SI{3.2e-15}{\electronvolt} & 	\SI{4.136e-15}{\electronvolt} & \SI{-29}{\percent}
	
	\end{tabular}
	\label{tab:Vergleich}
\end{table}

Die Abweichung kann über einige äußere Einflüsse auf das Experiment erklärt werden. Wenn es in der Umgebung ein bisschen heller wird zum Beispiel da an den umliegenden Versuchen Lampen verwendet werden, steigt der Photonenstrom direkt an. Des weiteren ist das Photoelement beweglich und verrutscht manchmal, was dazu führt, dass der Lichtstrahl nicht komplett einfällt.
\todo[inline, color = red]{Quelle für h!}