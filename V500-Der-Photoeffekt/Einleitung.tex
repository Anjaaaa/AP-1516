Als photoelektrischer Effekt (auch lichtelektrischer Effekt oder kurz Photoeffekt) wird das Herauslösen einzelner Elektronen aus einer Metalloberfläche durch elektromagnetische Strahlung bezeichnet. Mit der allgemein akzeptierten Vorstellung von Licht als elektromagnetischer Welle\footnote{Der Wellencharakter des Lichts schien eindeutig, durch Interferenz-Experimente und die Maxwell-Gleichungen, aus denen direkt elektromagnetische Wellen abgeleitet werden können.}, konnte der Photoeffekt lange nicht erklärt werden. Erst Albert Einstein gelang es mit Hilfe eines Teilchenmodells eine konsistente Erklärung zu Formulieren. Der Photoeffekt galt daher lange als ein Schlüssel-Experiment, das den Teilchencharakter des Lichts beweist.\footnote{In den 1960er Jahren gelang es den Photoeffekt mit der klassischen Wellentheorie zu erklären, sodass er nicht mehr als absoluter Beweis eines Welle-Teilchen-Dualismus gesehen werden kann.}