Ein Ziel des Versuches war es, die Schallgeschwingikeit zu bestimmen. Der errechnete Wert von $c = \SI{360.1 \pm 4.5}{\metre\per\second} $
weicht von den Literaturwerten um $10 \% $ ab, unter der Annahme, dass die Raumtemperatur $\SI{25}{\celsius} $ betrug.
Der theoretische Wert berechnet sich aus dem Literaturwert für $\SI{0}{\celsius} $ von $c = \SI{331.5}{\metre\per\second} $ (Quelle: Experimentalphysik 1, Wolfgang Demtröder, 4.Auflage, Seite 354) nach:
\begin{equation}
v(T) = v(T_0) \cdot \sqrt{\frac{T}{T_0}} = 331,15 \cdot \sqrt{\frac{273,15 + 25}{273,15}} = 346,3 \quad .
\end{equation}
Die Abweichung ist vor allem durch eine ungenaue Messung der Wellenlänge zu erklären. Die Wellenlängen, die aus den Ausgleichsgeraden bestimmt wurden, unterschieden sich stark von diesem Wert, was ebenso durch den T-Test verdeutlicht wurde. Es zeigt sich eine besonders große Abweichung bei der Messung, in der sich der Schlitten mit dem Lautsprecher von dem Mikrophon wegbewegte. Nach genauerer Betrachtung der Regressionsgeraden fällt auf, dass in dieser Messung die Frequenzdifferenz bei der dritten Geschwindigkeit komplett von den anderen Werten abweicht und sogar widererwarten positiv statt negativ ist. \\
Ohne Verwendung dieses Wertes verändern sich Steigung der Regressionsgeraden und die daraus folgende Wellenlänge zu
\begin{align}
m = \SI{-60}{\per\metre}   \quad \lambda = - \frac{1}{m} = \SI{0.01667}{\metre} \quad ,
\end{align}
 was näher an der anderen Werten liegt. \\
 Ein weiteres Indiz dafür, dass systematische Fehler vorliegen ist, dass beide Regressionsgeraden einen y-Achsenabschnitt aufweisen, obwohl sie direkt durch den Ursprung gehen sollten. \\
 Systematische Fehler können in diesem Versuch beispielsweise durch eine unregeläßige Bewegung des Waagen, das ungenaue Ablesen der Wellenlänge und unterschiedliche Ausrichtungen des Mikrophons, dessen Halterung kaputt war, bedingt sein. \\
 Eine weitere Datennahme über die Schwebungsmethode war nicht möglich. Der dazu notwendige Tiefpass war ebenfalls kaputt.
