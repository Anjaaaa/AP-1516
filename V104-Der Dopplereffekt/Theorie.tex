Der Doppler-Effekt ist den meisten von vorbeifahrenden Krankenwagen bekannt. Die Sirene des vorbeifahrenden Wagens ertönt zunächst in einem hohen Ton, der dann immer tiefer wird. \\
Im Allgemeinen bezeichnet der Doppler-Effekt eine Frequenzverschiebung zwischen Sender und Empfänger, sobald sich diese relativ zueinander bewegen. In Medien ist entscheidend, ob es sich um eine bewegte Quelle oder einen bewegten Empfänger handelt. \\
Eine Quelle emitiere eine Welle mit der Frequenz $\nu_0$ und der Wellenlänge $\lambda_0$. Die Ausbreitungsgeschwindigkeit der Welle ist somit
\begin{equation}
c = \nu_0 \cdot \lambda_0 \quad .
\end{equation}
Ein bewegter Empfänger der Geschwingigkeit $v$ (auf die Quelle zu) überstreicht in gleicher Zeit mehr Wellenberge als ein ruhender, so dass die ankommende Frequenz $\nu_\text{E}$ größer als die ausgesendete erscheint
\begin{equation}
\nu_\text{E} = \nu_0 + \frac{v}{\lambda_0} = \nu_0 \left(1 + \frac{v}{c} \right) \quad .
\end{equation}
\\
Ist nun der Empfänger in Ruhe und die Quelle das mit $v$ bewegte Objekt, gibt es einen sehr ähnlichen Effekt. Während die Quelle eine Wellenlänge $\lambda_0$ aussendet, bewegt sie sich selbst (auf den Empfänger zu) und die Wellenlänge scheint verkürzt. Der Empfänger registriert Wellen mit kürzerer Wellenlänge d.h. höherer Frequenz
\begin{equation}
\nu_\text{Q} = \frac{c}{\lambda_0 - \Delta \lambda} = \nu_0 \cdot \frac{1}{1-\frac{v}{c}} \quad .
\end{equation}
Der Unterschied zwischen $\nu_\text{E}$ und $\nu_\text{Q}$ ist sehr gering. Auch Elektromagnetische Wellen zeigen einen Doppeler-Effekt. Bei ihnen ist allerdingt nicht zu unterscheiden, ob sich Sender oder Empfänger bewegt. Zusäztlich ist ein relativistischer Faktor zu berücksichtigen. \\
Dieser Versuch beschäftigt sich mit akustischen Wellen, die von einer bewegten Quelle ausgesendet und von einem ruhenen Empfänger aufgenommen werden.