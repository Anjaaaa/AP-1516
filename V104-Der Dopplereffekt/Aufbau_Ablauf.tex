Zur Messung des Dopplereffektes stand als akustische Signalquelle ein Lautsprecher und als Siganlempfänger ein Mikrophon zur Verfügung.
\\
Die Quelle war auf einen Wagen motiert und ließ sich sowohl auf den Empfänger zu wie auch von ihm weg in zehn voreigestellten Geschwindigkeiten bewegen.
Zunächst galt es, jede dieser Geschwindigkeiten zu bestimmen. Es wurde die Zeit gemessen, in der der Wagen zwei Lichtschranken passierte. Dazu war eine Schlatung nötig, die Zeitimpulse eines Zeitbasisgenerators zählte, während sich der Wagen zwischen den Schranken befand.

Eine anschließende Messung des Abstandes der Lichtschranken ermöglichte die Bestimmung der Geschwindigkeiten.
\\
Auch im weiteren Verlauf waren die Lichtschranken wichtig für die Datenaufnahme. Es sollte die beim Empfänger eingehende Frequenz gemessen werden.  Um direkt die Frequenz (Schwingungen pro Sekunde) zu erhalten, bot es sich an, während einer Sekunde alle vom Empfänger registrierten Signale -- die in elektrische Signale umgewandelt wurden -- zu zählen. Die Lichtschranke löste den Signalzähler und den Zeitbasisgenerator aus, der wiederum nach einer Sekunde einen Impuls aussendete, der die Signalzählung beendete. 
Die erste Messung erfolgte bei ruhender Quelle, so dass die Lichtschranke per Hand ausgelöst werden musste. Danach wurden eingehende Frequenzen bei bewegter Quelle gemessen, welche die Lichtschranke dann selbst auslöste. Frequenzen wurden für jede Geschwindigkeit in Vor- und Rückrichtung notiert. \\
Bei der Bestimmung der Schallgeschwindigkeit ist zudem die Wellenlänge entscheidend. In einem Oszilloskop wurde das Signal des Senders auf der y-Achse  angezeigt, während das empfangene Signal auf der x-Achse zu sehen war. Beide Wellen überlagerten sich zu Lissajous-Figuren, die angaben, ob Sender- und Empfängerwelle in Phase waren. War dies der Fall wurde der Abstand zwischen Sender und Empfänger leicht verändert, bis beide Signale wieder in Phase waren. Dieser Abstand entsprach genau einer Wellenlänge.

