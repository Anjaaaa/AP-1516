\subsubsection{Bestimmung der Wärmekapazität des Kalorimeters}
	Die Masse des kalten Wassers betrug
	\begin{equation}
		m_\text{x} = \SI{251.80}{\gram},
	\end{equation}
	die des warmen Wassers war
	\begin{equation}
		m_\text{y} = \SI{317.80}{\g}.
	\end{equation}
	Zusammen mit der Temperatur des kalten Wassers
	\begin{equation}
		T_\text{x} = \SI{21.6}{\degree\celsius},
	\end{equation}
	der des warmen Wassers
	\begin{equation}
		T_\text{y} = \SI{99.7}{\degree\celsius}
	\end{equation}
	und der Endtemperatur
	\begin{equation}
		T_\text{m} = \SI{61.0}{\degree\celsius},
	\end{equation}
	eingesetzt in Formel VERWEIS kann die Wärmekapazität des Kalorimeters zu
	\begin{equation}
		c_\text{g}m_\text{g} = \SI{252.279}{\joule\per\kelvin}
	\end{equation}
	bestimmt werden.


\subsubsection{Bestimmung der Wärmekapazität verschiedener Stoffe}
\subsubsection*{Graphit}
	Für den
	\begin{equation}
		m_\text{G} = \SI{108.01}{\gram}
	\end{equation}
	schweren Graphitkörper, der mit einer Anfangstemperatur von
	\begin{equation}
		T_\text{k} = \SI{75.0}{\degree\celsius}
	\end{equation}
	solange in
	\begin{equation}
		T_\text{w, G} = \SI{23.0}{\degree\celsius}
	\end{equation}
	kaltes Wasser getaucht wurde bis Körper und Wasser die Mischtemperatur von
	\begin{equation}
		T_\text{m, G} = \SI{26.0}{\degree\celsius}
	\end{equation}
	erreicht hatten, ergibt Formel VERWEIS eine spezifische Wärmekapazität
	\begin{equation}
		c_\text{G} = \SI{1.4997}{\joule\per\kelvin}.
	\end{equation}
\subsubsection*{Blei}
	\begin{table}[h]
\begin{center}
\begin{tabular}{c | c | c | c}
	Temp. Körper & Temp. Wasser & Mischtemperatur & Spez. Wärmekapazität \\
	\hline
	78.0 & 22.3 & 24.7 & 0.219 \\
	81.0 & 21.9 & 24.4 & 0.215 \\
	79.5 & 21.9 & 24.0 & 0.184 \\
	65.0 & 22.4 & 23.9 & 0.178 \\
\end{tabular}
	\caption{Messung der Temperatur in \si{\celsius} und mit \eqref{Warmekapazitat Stoff} resultierende speziefische Wärmekapazität in \si{\joule\per\gram\per\kelvin}}
\end{center}
\end{table}
	Die Messung erfolgte mit einem
	\begin{equation}
		m_\text{B} = \SI{544.59}{\gram}
	\end{equation}
	schweren Bleikörper. Der Mittelwert der, für jede Messung einzeln mit Zusammenhang VERWEIS berechneten, Werte ist
	\begin{equation} 
		c_\text{B} = \SI{0.199(9)}{\joule\per\kelvin}.
	\end{equation}

