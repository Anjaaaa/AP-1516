Ziel der Durchführung ist es die spezifischen Wärmekapazitäten zweier Proben (Blei und Graphit) und daraus  die spezifschen Wärmekapazitäten zu bestimmen, um das Dulong-Petit'sche Gesetz zu überprüfen.
Es ist es schwer realisierbar ein System bei konstantem Volumen zu halten. Allgemein wird unterschieden zwischen Wärmeänderungen bei konstantem Volumen $V$ und Wärmeänderungen bei konstantem Druck $P$. Sie lassen sich überführen, wenn der lineare Ausdehnungskoeffizient $\alpha$, das Kompressionsmodul $\kappa$, das Molvolumen $V_0$ und die aktuelle Temperatur $T$ begannt sind
\begin{equation}\label{konstanter Druck -> konstantem Volumen}
  C_\text{P} - C_\text{V} = 9 \cdot \alpha^2 \cdot \kappa \cdot V_0 \cdot T \quad.
\end{equation}  \\
In ein Mischungskaloriemeter wurde immer die gleiche Menge Wasser mit Gewicht $m_\text{W}$ und der Temperatur $T_\text{W}$ gefüllt.
Die Probe der Masse $m_k$ wurde in einem Wasserbad auf möglichst hohe Temperaturen $T_k$ erhitzt und dann in das Mischungskaloriemeter gehalten. Die Probe gab Wärme an das Wasser ab, bis sich nach einiger Zeit die Mischungstemperatur $T_\text{m}$ einstellte. Graphit wurde nur einmal, Blei vierfach gemessen. Alle Temperaturen wurden mit einem Thermometer gemessen.
\\
Aus der Tatsache, dass die von der Probe abgegebene Wärme, der vom Wasser und dem Mischungskaloriemeter (g) aufgenommenen entspricht, lässt sich die spezifische Wärmekapazität des Probekörpers berechnen
\begin{equation}\label{Warmekapazitat Stoff}
  c_k = \frac{(c_\text{w} m_\text{w} + c_\text{g} m_\text{g})(T_\text{m} - T_\text{W})}{m_k (T_k - T_\text{m})}
\end{equation}
Die spazifische Wärme von Wasser ist $c_\text{W} = \SI{4.18}{\joule\per\gram\per\kelvin}$ Die Wärme, die das Mischungskaloriemeter aufnimmt, wird in einer separaten Messung bestimmt. In das Kaloriemeter wird hierzu aus zwei Bechergläsern $x$ und $y$ Wasser gefüllt. $x$ hat Raumtemperatur und $y$ wurde erhitzt. Auch hier stellt sich einem Mischungtemperatur $T_\text{m}$ ein. Die gesuchte Größe ist
\begin{equation}\label{Warmekapazitat Kalorimeter}
  c_\text{g} m_\text{g} = \frac{c_\text{w} m_\text{y}(T_\text{y}-_\text{m}) - c_\text{w} m_\text{x}(T_\text{m}-T_\text{x})}{T_\text{m} - T_\text{x}} \quad.
\end{equation}
