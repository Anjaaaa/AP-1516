Bei identischen Temperaturdifferenzen gibt nicht jedes Material die selbe Wärmemenge ab. In einem Schwimmbad friert man schneller als an der Luft. Im Winter berührt man ungern Gegenstände aus Metall, während es unproblematisch ist, welche aus Plastik anzufassen.
Zwischen der Temperaturdifferenz $\Delta T$ und der Änderung der Wärmemenge $\Delta Q$ besteht linearer Zusammenhang
\begin{equation}
  \Delta Q = m \cdot c_k \cdot \Delta T
\end{equation}
mit der Masse $m$ und dem Proportionalitätsfakor $c_k$, der sogenannten spezifischen Wärmekapazität.
Die spezifische Wärmekapazität eines Mols wird als Molwärme $C$ bezeichnet. Ist das Volumen konstant und wird keine zusätzliche Arbeit von Außen am System verrichtet, so ist die Änderung der Wärme identisch zur Änderung der inneren Engie $U$
\begin{equation}
  C_\text{V} = \left(\frac{\text{d} Q}{\text{d} T} \right)_\text{V = const}= \left(\frac{\text{d} U}{\text{d} T} \right)_\text{V = const} \quad.
\end{equation}
Die innere Energie ist mit der kinetsichen Energie $E_\text{kin}$ wie folgt verknüpft
\begin{equation}
  U = 2 \cdot \frac{R}{k} \cdot E_\text{kin}
\end{equation}
mit der Boltzmankonstante $b = \SI{1.38e-23}{\joule\per\kelvin}$ und der Gaskonstante $R = \SI{8.314}{\joule\per\mol\per\kelvin}$.
Nach dem Äquipatitionstheorem ist die kinetische Energie pro Freiheitsgrad
\begin{equation}
E_\text{kin}=\frac{1}{2}\cdot k \cdot T
\end{equation}
und somit die innere Energie
\begin{equation}
U = 3 \cdot R \cdot T \quad.
\end{equation}
Hieraus ist direkt das Dulong-Petitsche Gesetz abzulesen. Es besagt, dass alle Elemente die selbe Atomwärme haben
\begin{equation}
  C_\text{V} = 3 \cdot R \quad.
\end{equation}
Dieses Gesetz erweist sich in der Praxis in vielen Fällen als richtig. Es gilt aber nicht mehr für kleine Temperaturen oder Elemtene mit niedrigem Atomgewicht. Eine Erklärung dafür stammt aus dem Quantenmechanik. Energien können nicht in beliebig kleinen Beträgen aufgenommen werden, sondern in gequantelten Portionen zu diskteten Zeiten mit bestimmten Wahrscheinlichkeiten. Das Dulong-Petit-Gesetz ist lediglich ein Spezialfall für hohe Energien d.h. Temperaturen. \\
