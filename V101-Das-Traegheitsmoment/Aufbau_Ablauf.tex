Im Mittelpunkt des Versuchs stand eine Drillachse, bei der wir zunächst mit Hilfe eines Federkraftmessers die rücktreibende Kraft für zehn verschiedene Auslenkwinkel bestimmt haben (Tabelle 1). \\
Danach wurde eine Metallstange auf der Drillachse befestigt. Auf beiden Seiten der Drehachse und mit gleichem Abstand zu ihr war je ein Gewicht angebracht. Die Stange mit den Gewichten wurde ausgelenkt und die Schwingungsdauer gemessen. Wie Tabelle 2 zu entnehmen ist, haben wir die Messung für verschiedene Abstände der Gewichte zur Mitte wiederholt. \\
Im letzten Teil des Versuchs befestigten wir verschiedene Körper auf der Drillachse und stoppten die Schwingungsdauern. In den Tabellen 3 und 4 finden sich die Werte zweier Zylinder, die wir vermessen und quer auf der Apparatur befestigt haben. \\
Tabelle 5 zeigt die Daten einer Modellpuppe aus Holz. Sie wurde zunächst an allen Körperteilen mehrfach vermessen, damit man sie später gut als einen aus Zylindern zusammengesetzten Körper nähern kann. Dann haben wir die Puppe auf der Drillachse die Schwingungsdauer gemessen.