Die Messung der Fallzeiten ist die Grundlage aller Berechnungen. Durch systematische Fehler kann sie verfälscht werden. Das könnte einerseits durch beim Verschließen im Rohr verbliebene Luftblasen geschehen. Sie wirken durch ihren Auftrieb der Schwerkraft entgegen und verlängern somit die Fallzeiten. Denselben negativen Effekt hat auch der (in den Rechnungen vernachlässigte) Reibungseffekt zwischen Kugel und Wand. \\
Die Abweichung der Konstanten in der Andrade-Gleichung von Literaturwerten\footnote{\url{http://www.chemie.de/lexikon/Andrade-Gleichung.html}, abgerufen am 28.01.2016 um 14:00 Uhr} ist in Tabelle \ref{fig:Andrade} zu sehen.
\begin{table}[h!]
\centering
\begin{tabular}{c|c|c|c}
	& Literatur & Berechnet & Abweichung \\
	\hline
	$A$ & \SI{9.644e-4}{} & \SI{2.4e-6}{} & -99.8\% \\
	$B$ & 2036.8 & 1775 & -12.9\%
\end{tabular}
\caption{Abweichung der Konstanten der Andrade-Gleichung}
\label{fig:Andrade}
\end{table}