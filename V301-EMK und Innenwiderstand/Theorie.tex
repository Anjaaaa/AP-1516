Wird in der Elektrotechnik ein Stromkreis betrachtet, so gilt das 2. Kirchhoffsche Gesetz
\begin{equation}\label{Maschenregel}
	\sum_n U_i = \sum_n I_iZ_i \quad.
\end{equation}
Wobei die $U_i$ die angelegten Spannungen, $Z_i$ die Impedanzen und $I_i$ der Strom, der durch $Z_i$ fließt ist. Für einen Stromkreis mit einer Spannungsquelle $U$ und einem Lastwiderstand $R_a$ gilt dann
\[U = IR_a \quad.\]
Wird ein solcher Stromkreis, an dem die Spannung $U_0$ anliegt, im Experiment betrachtet gilt diese Gleichung nicht für $U = U_0$. Das liegt daran, dass eine reale Spannungsquelle immer einen Innenwiderstand $R_i$ hat, der den Stromfluss beeinflusst. Im Schaltbild und in der Rechnung kann eine solche reale Spannungsquelle wie eine ideale Spannungsquelle betrachtet werden, hinter der ein Widerstand der Größe $R_i$ geschaltet ist (siehe Abbildung \ref{Stromkreis}). Dann gilt wie gewohnt \eqref{Maschenregel}:
\begin{equation}\label{Ersatzschaltung}
	U_0 = IR_i+IR_a \quad.
\end{equation}
Die angelegte Spannung $U_0$ wird auch Leerlaufspannung genannt. Die im Schaltbild als $U_k$ bezeichnete Klemmspannung, ist hingegen die Spannung, die dann am eigentlichen Stromkreis anliegt. Sie kann mit \eqref{Maschenregel}
\begin{equation}
	U_k = IR_a = U_0-IR_i
\end{equation}
berechnet werden. 
\begin{figure}[h!]
\begin{center}
\begin{circuitikz}
	\draw (0,0)
	to[battery1=$U_0$] (0,1.5)
	to[R=$R_i$] (0,3)
	to[short] (4,3)
	(0,0) to[short] (4,0)
	to[short] (4,0)
	to[R, l_=$R_a$] (4,3);
	\draw (2,0) edge[<->] (2,3)
	(2.5,1.5) node[]{$U_k$};
	\draw[dashed] (-1.5,-0.5) rectangle (1,3.5)
	(-2.7,1.5) node[text width=2cm, align=center]{Ersatz\-schaltbild};
\end{circuitikz}
\caption{Reale Spannungsquelle in einem Stromkreis}
\label{Stromkreis}
\end{center}
\end{figure}
Eine ideale Spannungsquelle müsste demnach einen Innenwiderstand von $R_i = 0$ haben, sodass $U_k = U_0$ gilt. Es ist zu beachten, dass es bei Generatoren zu Rückkopplungseffekten kommen kann, sodass für ihren Innenwiderstand ein differentieller Zusammenhang
\begin{equation}
	R_i = \dv{U_k}{I}
\end{equation}
gilt. \\
\ \\
Die Leistung $N$, die an einen Widerstand $R$ abgegeben wird ist definiert als
\begin{equation}
	N = IU_R = I^2R \quad.
\end{equation}
Wird weiterhin ein Stromkreis mit nur einem Lastwiderstand betrachtet, gilt für den Strom nach \eqref{Ersatzschaltung}
\begin{equation}
	I = \frac{U_0}{R_i + R_a},
\end{equation}
sodass die Leistung am Lastwiderstand
\begin{equation}
	N(R_a) = \left(\frac{U_0}{R_i + R_a}\right)^2R_a
\end{equation}
ist. Könnte der Innenwiderstand $R_i$ Null sein, wäre es mit $R_a\rightarrow 0$ möglich beliebig hohe Leistungen zu erreichen. Das Maximum der Leistung wird bei $R_a = R_i$ erreicht. Eine dementsprechende Wahl von $R_a$ wird Leistungsanpassung genannt. 