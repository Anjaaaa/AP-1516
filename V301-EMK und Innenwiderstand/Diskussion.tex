Wieso liegt der Wert der Leerlaufspannung der Monozelle aus Messung b nicht genau auf 1.5 und die Gegenspannung nicht genau 2V? \\
--- Sollte die Gegenspannung nicht 2V größer, als die Leerlaufspannung der Monozelle sein? Und die Gegenspannung wird doch nirgends berechnet? \\
- Voltmeter kann es nicht sein nach Auswertung \\
--- Wieso nicht? Dass das Voltmeter keinen unendlich großen Widerstand hat, wäre eine gute Erklärung. \\
- Amperemeter keinen unendlich kleinen Widerstand \\
- Rückkopplungseffekte

RC-Generator: \\
- woher kommen die großen Unterschiede zwischen Rechteck und Sinusspannung? \\
--- Bei der Rechteckspannung ist der Effektivwert höher und das wurde gemessen? \\
Der errechnete Innenwiederstand des RC-Generators für die Rechteckspannung stimmt mit dem Literaturwert\footnote{Elektronikpraktikum, H. Pfeiffer, S. 55} von circa \SI{50}{\ohm} überein, zumal dieser Wert vom Gerät abhängig ist. Wieso der Innenwiderstand des Generators beim Erzeugen einer Sinusspannung so viel höher ist können wir nur vermuten. Das Rechtecksignal wird in ein Dreiecksignal umgewandelt und diese beiden zusammen ergeben die Sinusspannung. Anscheinend ist der große Innenwiderstand darauf zurückzuführen. \\
--- Ich könnte mir vorstellen, dass das was damit zu tun hat, dass der Innenwiderstand hier eine differentielle Größe ist. Und bei der Rechteckspannung wird das sehr klein, weil es ja eigentlich keine Steigung, sonder nur konstante Stücke gibt und bei der Sinus- (bzw. dem Dreiecksanteil der) Spannung ist die Steigung nicht null.



% Link dazu: https://books.google.de/books?id=T9_PBgAAQBAJ&pg=PA54&lpg=PA54&dq=rc+generator+innenwiderstand&source=bl&ots=pYmou0fpgV&sig=GJPhSBjWjCqLw_08xIqD878BgYs&hl=de&sa=X&ved=0ahUKEwidsq61m9TJAhXGYQ8KHQwZAREQ6AEILDAC#v=onepage&q=rc%20generator%20innenwiderstand&f=false

Leistung: \\
-die Werte sind bis auf einen recht gut! \\
-Maximum händisch bestimmen \\
\begin{align*}
	\dv{N(R_a)}{R_a} &= \dv{}{R_a}\left(\left(\frac{U_0}{R_i+R_a}\right)^2R_a\right) \\
	&= \frac{U_0^2(R_i-R_a)}{(R_i+R_a)^3} \\
	&\overset{!}{=} 0 \quad \Rightarrow R_a = R_i
\end{align*}