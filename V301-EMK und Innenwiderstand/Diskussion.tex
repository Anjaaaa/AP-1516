Wieso liegt der Wert der Leerlaufspannung der Monozelle aus Messung b nicht genau auf 1.5 und die Gegenspannung nicht genau 2V? \\
- Voltmeter kann es nicht sein nach Auswertung \\
- Amperemeter keinen unendlich kleinen Widerstand \\
- Rückkopplungseffekte

\\

RC-Generator: \\
- woher kommen die großen Unterschiede zwischen Rechteck und Sinusspannung? \\
Der errechnete Innenwiederstand des RC-Generators für die Rechteckspannung stimmt mit dem Literaturwert\footnote{Elektronikpraktikum, H. Pfeiffer, S. 55} von circa \SI{50}{\ohm} überein, zumal dieser Wert vom Gerät abhängig ist. Wieso der Innenwiderstand des Generators beim Erzeugen einer Sinusspannung so viel höher ist können wir nur vermuten. Das Rechtecksignal wird in ein Dreiecksignal umgewandelt und diese beiden zusammen ergeben die Sinusspannung. Anscheinend ist der große Innenwiderstand darauf zurückzuführen.



% Link dazu: https://books.google.de/books?id=T9_PBgAAQBAJ&pg=PA54&lpg=PA54&dq=rc+generator+innenwiderstand&source=bl&ots=pYmou0fpgV&sig=GJPhSBjWjCqLw_08xIqD878BgYs&hl=de&sa=X&ved=0ahUKEwidsq61m9TJAhXGYQ8KHQwZAREQ6AEILDAC#v=onepage&q=rc%20generator%20innenwiderstand&f=false

Leistung: \\
-die Werte sind bis auf einen recht gut! \\
-Maximum händisch bestimmen \\