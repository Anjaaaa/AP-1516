\begin{enumerate}
\item \textbf{Leerlaufspannung} Im ersten Messschritt wird die Leerlaufspannung, der im weiteren Verlauf verwendeten Monozelle, mit einem Voltmeter gemessen.
\item \textbf{Klemmspannung einer Monozelle}
\begin{enumerate}[label=\alph*)]
	\item Mit der in Abbildung \ref{Klemmspannung_Bild} dargestellten Schaltung wird jeweils die Klemmspannung $U_k$ und der Strom $I$ bei verschiedenen Lastwiderständen im Bereich von $0-50\ \si{\ohm}$ gemessen. Hier ist zu beachten, dass das Voltmeter an der eingezeichneten Stelle angeschlossen wird, denn nur dort kann die Spannung gemessen werden, die in den Stromkreis eingespeist wird. Würde das Voltmeter beispielsweise nach dem Amperemeter (also zwischen den Punkten H und G) angeschlossen, würde es die Klemmspannung minus die Spannung, die am Amperemeter abfällt, messen.
	\item Die Klemmspannung kann auch mit Hilfe einer Gegenspannung bestimmt werden. Dazu wird eine Gegenspannung, die etwa \SI{2}{\volt} größer ist, als die Leerlaufspannung hinter den Lastwiderstand (siehe Abbildung \ref{Klemmspannung_Gegenspannung}) geschaltet. Auch hier findet die Messung der Klemmspannung $U_k$ und des Stroms $I$ bei verschiedenen Lastwiederständen im Bereich von $0-50\ \si{\ohm}$ statt.
\end{enumerate}
\begin{figure}[h!]
\begin{subfigure}{0.52\textwidth}
\begin{center}
\begin{circuitikz}
	\draw (0,0)
	to[battery1=$U_0$] (0,3)
	to[short, -*] (1.5,3)
	to[ammeter] (4,3)
	to[short, *-] (5,3)
	(4,3) to[short, *-] (4,2.5)
	(4,0) to[short, *-] (4,0.5)
	(0,0) to[short, -*] (1.5,0)
	to[short] (5,0)
	to[vR, l_=$R_a$] (5,3)
	(1.5,3) to[voltmeter, color=white, name=Voltmeter] (1.5,0); % The voltage source
	\mymeter{Voltmeter}{0}
	\draw (4,2.3) node{H}
	(4,0.7) node{G};
\end{circuitikz}
\caption{direkt}
\label{Klemmspannung_Bild}
\end{center}
\end{subfigure}
\begin{subfigure}{0.52\textwidth}
\begin{center}
	\begin{circuitikz}
		\draw (0,0)
		to[battery1=$U_0$] (0,3)
		to[short, -*] (1.5,3)
		to[ammeter] (4.5,3)
		(4.5,0) to[dcvsource=$U_\text{gegen}$] (4.5,1.5)
		to[vR, l_=$R_a$] (4.5,3)
		(1.5,3) to[voltmeter, color=white, name=Voltmeter] (1.5,0) % The voltage source
		(0,0) to[short, -*] (1.5,0)
		to[short] (4.5,0);
		\mymeter{Voltmeter}{0}
	\end{circuitikz}
	\caption{mittels Gegenspannung}
	\label{Klemmspannung_Gegenspannung}
\end{center}
\end{subfigure}
\caption{Messung der Klemmspannung}
\end{figure}
\item \textbf{Klemmspannung eines RC-Generators} Der Stromkreis \ref{Klemmspannung_Bild} eignet sich auch zur Bestimmung der Klemmspannung einer Wechselspannungsquelle. Zunächst wird der \SI{1}{\volt}-Rechteckausgang eines RC-Generators als Spannungsquelle verwendet. Wie bei den vorherigen Messungen werden wieder die Spannung $U_k$ und der Strom $I$ bei variablem Lastwiderstand, dieses Mal zwischen $20$ und $250$ \si{\ohm}, gemessen. Die Messung wird für den \SI{1}{\volt}-Sinusausgang des RC-Generators und einem Lastwiderstand von $0.1-5$ \si{\kilo\ohm} wiederholt.
\end{enumerate}
