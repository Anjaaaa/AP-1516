\subsection{Bestimmung der Leerlaufspannung und Berechnung des Innenwiderstands für drei Spannungsquellen}
Die Ableseungenauigkeit der Stommessung beträgt $2 \% $ und die der Spannungsmessung $3 \% $. In den folgenden Diagrammen sind diese Fehler durch Fehlerbalken gekennzeichnet. \\
Mit einer linearen Regression mittels Python nach Formel \eqref{Klemmspannung}, folgen für die Monozelle die Leerlaufspannung
\[U_0 = \SI{1.47(1)}{\volt}\]
und der Innenwiderstand
\[\label{Innenwiderstand_Monozelle}
R_i = \SI{5.5(1)}{\ohm} \ .\]
Abbildung \ref{fig:Regression_Monozelle} zeigt die Regressionsgerade mit den Datenpunkten.
\begin{figure}[h!]
	\centering
	\includegraphics[width=0.95\textwidth]{Spannung_Messung_b.pdf}
	\caption{Lineare Regression zur Monozelle}
	\label{fig:Regression_Monozelle}
\end{figure}

Beim Anlegen der Gegenspannung (Abbildung \ref{Klemmspannung_Gegenspannung}) fließt der Strom in die entgegengesetzte Richtung, wodurch Formel \eqref{Klemmspannung} zu
\begin{equation}
U_k = IR_a = U_0+IR_i
\end{equation}
wird. Die lineare Regression (Graph siehe Abbildung \ref{fig:Regression_Gegenspannung}) liefert in diesem Fall :
\begin{align}
&U_0 = \SI{1.38(3)}{\volt} \ , \\
&R_i = \SI{5.9(2)}{\ohm}  \ .
\end{align}
\begin{figure}[h!]
	\centering
	\includegraphics[width=0.95\textwidth]{Spannung_Messung_c.pdf}
	\caption{Lineare Regression zur Gegenspannung}
	\label{fig:Regression_Gegenspannung}
\end{figure}


Für den RC-Generator wird der gleiche Aufbau (siehe Abbildung \ref{Klemmspannung_Bild}) wie bei der Monozelle verwendet. Die Regression ergibt für die Rechteckspannung (Abbildung \ref{fig:Regression_Rechteck})
\begin{align}
&U_0 = \SI{0.555(6)}{\volt} \ , \\
&R_i = \SI{61(2)}{\ohm} 
\end{align}
und für die Sinusspannung (Abbildung \ref{fig:Regression_Sinus})

\todo[color = lightgray]{Wird \SI{675.6(125)}{\volt} wirklich zu \SI{0.68(1)}{\kilo\volt}? Das ist ja schon ein großer Unterschied.}
\begin{align}
&U_0 = \SI{0.233(2)}{\volt} \ , \\
&R_i = \SI{0.68(1)}{\kilo\ohm} \ .
\end{align}





\begin{figure}[h!]
	\centering
	\includegraphics[width=0.95\textwidth]{Spannung_Messung_d.pdf}
	\caption{Lineare Regression zum RC-Generator (Rechteckspannung)}
	\label{fig:Regression_Rechteck}
\end{figure}

\begin{figure}[h!]
	\centering
	\includegraphics[width=0.95\textwidth]{Spannung_Messung_e.pdf}
	\caption{Lineare Regression zum RC-Generator (Sinusspannung)}
	\label{fig:Regression_Sinus}
\end{figure}

\clearpage

\subsection{Systematischer Fehler -- endlicher Widerstand des Voltmeters}
\label{systhematischer_fehler}
Der endliche Widerstand des Voltmeters ($R_v = \SI{10}{\mega\ohm}$) führt zu einem systematischen Fehler. Der Widerstand des Messgerätes müsste unendlich groß sein, um den Stromkreis und -fluss nicht zu beeinflussen. Um den dadurch verursachten Fehler auszurechnen wird die Leerlaufspannung der Monozelle
\begin{equation}
U_k = \SI{1.5}{\volt}
\end{equation}
direkt gemessen.
Der Innenwiderstand der Monozelle (siehe \eqref{Innenwiderstand_Monozelle}) wird aus einer anderen Messreihe übernommen. Durch Umstellen der Formel \eqref{Klemmspannung} nach $U_0$ vereinfacht sich der absolute Fehler zu 
\todo[color = lightgray]{Wir haben das hier jetzt einfach auch mal gerundet? Ist das falsch? Wieso hast Du das hier nicht angestrichen?}
\begin{equation}
\Delta U = U_0 - U_k = I\cdot R_i = U_k \cdot \frac{R_i}{R_v} = \SI{8e-7}{\volt}
\end{equation}
und der relative Fehler wird 
\begin{equation}
\frac{\Delta U}{U_k} = \frac{R_i}{R_v} = \SI{5e-7}{} \ .
\end{equation}
Dieser Fehler ist vernachlässigbar klein.

\clearpage 
 
\subsection{Das Maximum der umgesetzten Leistung}
Wie bereits in der Theorie erklärt, ist die umgesetzte Leistung abhängig vom Lastwiderstand $R_a$ und nimmt sogar ein lokales Maximum ein. Hier werden die Messreihen der Monozelle betrachtet. Die Leistung $N_\text{Mess} = U_k \cdot I $ wird über den Belastungswiderstand $R_a = U_k / I$ aufgetragen. Ein Fehler entsteht durch die Ableseungenauigkeit der Messgeräte.

\todo[color = lightgray]{Bei der Tabelle wolltest Du die Formeln für die Fehlerfortpflanzung wissen. Aber das sind die Gerätefehler (siehe erster Satz in Abschnitt 3). Was soll sich da fortpflanzen? \\
Im gesamten Protokoll gibt es nur diese Fehler und Fehler, die aus der Regression kommen und die Fehler werden anders berechnet (bzw. leiten sich doch nicht mal aus berechneten Größen ab). Daher keine Fehlerfortpflanzung.}

\begin{center}
	\captionof{table}{Belastungswiderstand und Leistung mit Fehlern}
\begin{tabular}{c|c||c|c}
Widerstand in $\si{\ohm}$ & Fehler in $\si{\ohm}$ & Leistung in $\si{\watt}$ & Fehler in $\si{\watt}$ \\
\hline
  0.36  & 0.013  & 0.0225   & 0.00081  \\
  5.21  & 0.188  & 0.1022   & 0.00368  \\
  9.00  & 0.324  & 0.0900   & 0.00325  \\
 15.75  & 0.568  & 0.0840   & 0.00303  \\
 21.70  & 0.782  & 0.0610   & 0.00220  \\
 27.11  & 0.978  & 0.0549   & 0.00198  \\
 32.05  & 1.156  & 0.0488   & 0.00176  \\
 37.50  & 1.352  & 0.0434   & 0.00156  \\
 43.33  & 1.562  & 0.0390   & 0.00141  \\
 49.07  & 1.769  & 0.0358   & 0.00129  \\
 50.96  & 1.837  & 0.0345   & 0.00124  \\
\end{tabular}
\label{Werte_Leistung}
\end{center}

\todo[inline, color = lightgray]{Wenn man \glqq industriell\grqq\ runden will, was passiert dann mit 0.978? Wird das zu 1? oder zu 1.0?}
\todo[inline, color = lightgray]{Was passiert mit dem ehemals ersten Wert $0.36\pm0.013$, wenn man \glqq industriell\grqq\ runden will? Schummelt man dann eine Null hinten hin, damit man auch wieder drei Nachkomma-Stellen hat?}

Abbildung \ref{fig:Werte_Leistung} zeigt die theoretische Abhängigkeit der Leistung vom Widerstand (siehe \eqref{Leistung}) und die Werte aus Tabelle \ref{Werte_Leistung}.

\begin{figure}[h!]
	\centering
	\includegraphics[width=0.6\textwidth]{Leistungskurve.pdf}
	\caption{Leistung $N(R_a)$ am Lastwiderstand}
	\label{fig:Werte_Leistung}
\end{figure}



