Mit Hilfe der, in Abblidung VERWEIS
%%%%% Abbildung
gezeigten Apparatur, sollte die Dampfdruckkurve für $p\leq 1$ bestimmt werden. \\
Nach dem Evakuieren des Mehrhalskolbens, wurde das darin enthaltene Wasser erhitzt. Sobald das Wasser anfing zu verdampfen, wurden Druck und Temperatur in $\SI{5}{\celsius}$-Schritten bis zur Siedetemperatur gemessen. \\
\ \\
Abbildung VERWEIS
%%%% Abbildung
zeigt die Messapparatur, mit der die Dampfdruckkurve für $p > 1$ gemessen wurde. Zunächst wurde der Kolben inklusive dem darin enthaltenen Wasser bei konstantem Druck erhitzt. Bei Temperaturen etwas unter und über \SI{100}{\celsius} wurde der zugehörige Druck abgelesen. Bei der Berechnung später ist zu beachten, dass das Druckmanometer nicht richtig geeicht war, sodass für alle Werte gilt
\begin{equation}
	p(T) = p(T) - (p(\SI{100}{\celsius})-\SI{10e+5}{\pascal}).
\end{equation}