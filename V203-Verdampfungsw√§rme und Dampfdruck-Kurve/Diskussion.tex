Die errechnete Verdampfungswärme für kleine Drücke stimmt mit dem Literaturwert bei einer Temperatur von \SI{100}{\celsius} (entnommen \cite[Kapitel~5.6.1: Koexistenz von Flüssigkeit und Dampf]{Gerthsen}) überein. Es gibt zwar eine geringe Abweichung vom Mittelwert unserer Messung, der Literaturwert liegt aber innerhalb der Standardabweichung.
\begin{figure}[h!]
	\captionof{table}{Verdampfungswärme bei niedrigem Druck}	
	\centering
	\begin{tabular}{c|c|c}
		gemessene Verdampfungswärme & Literaturwert &  Abweichung \\ 
		\hline
		\SI{40800(500)}{\joule\per\mol} & \SI{40.590}{\kilo\joule\per\mol} & - 0.05 \%
	\end{tabular}}
\end{figure}

Auch stimmt mit der Literatur überein, dass die innere Verdampfungswärme ist stark dominierend ist. Bei Wasser beträgt der Wert der äußeren Verdampfungswärme unter Normaldruck nur etwa 10 \% des Wertes der Inneren. \cite{TUGraz} \\
%\footnote{siehe \url{http://portal.tugraz.at/portal/page/portal/Files/i5110/files/Lehre/Praktika/GP1/Vorbereitung/Verdampf_Gesamt.pdf}}

Bei der Messung der Temperaturabhängigkeit der Verdampfungswärme zeigt sich deutlich, dass bei zunehmender Temperatur die Verdampfungswärme abnimmt. (Vgl. \ref{fig:L_groser_druck_temperaturabhangig}). Dieser Trend  stimmt mit den Literaturwerten in \ref{tab:wikipedia} überein Leider sind die gemessenen Werte viel zu hoch. Das könnte teilweise daran liegen, dass die Temperatur der Heizstäbe, nicht die des Wassers oder des Wasserdampfes gemessen wurde. Um eine möglichst geringe Abweichung zu erhalten wurde sehr langsam erhitzt. Trotzdem sind Unterscheide von mehreren Grad denkbar.


\begin{table}[h!]
\begin{center}
\begin{tabular}{c | c}
	Temperatur & $L_\text{Wasser}$ \\
	\hline
	80 & 41.585 \\
	100 & 40.657 \\
	120 & 39.684 \\
	140 & 37.518
\end{tabular}
\end{center}
\caption{Literaturwerte aus Wikipedia}
\label{tab: wikipedia}
\end{table}