Im Versuch wird der Ultraschall mit einem Ultraschallechoskop erzeugt und auf einem Rechner visualisiert. Das Computerprogramm zeigt die detektierte Spannung in Abhängigkeit von der Zeit. Wird die Schallgeschwindigkeit des Mediums dort eingetragen, kann auch die Tiefe des betrachteten Körpers angezeigt werden. Als Kontaktmittel zwischen verschiedenen Körpern bzw. den Körpern und den Ultraschallsonden wird Wasser oder Ultraschallgel verwendet. \\
Zunächst wird mit Hilfe der Impuls-Echo-Methode die Durchlaufzeit und die Reflexionsamplitude eines Acryl-Zylinders mit bekannter Länge gemessen. Daraus können die Schallgeschwindigkeit in Acrylglas und der Absorptionskoeffizient bestimmt werden. \\
Danach wird auch die Durchlaufzeit für weitere Zylinder(-kombinationen) mit bekannter Länge gemessen. \\
Schließlich werden dieselben Messungen unter Verwendung der Durchschallungsmethode wiederholt.