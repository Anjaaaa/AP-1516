\grqq Der Frequenzbereich oberhalb der [hörbaren Schallwellen] wird Ultraschall genannt\grqq\ \cite{\V}.\todo[inline, color = red]{Ich habe das so hingeschrieben und dann gesehen, dass das genauso in der Anleitung steht. Mir ist keine annehmbar Umformung eingefallen und da stand ich vor einem Problem: Man schreibt sich was aus dem Gedächtnis auf und dann purzeln da halt auch mal dieselben Formulierungen raus. Was ist, wenn man das aber nicht schön umformen kann? Muss man das dann dennoch, so wie ich es hier jetzt bewusst gemacht habe, als richtiges Zitat behandeln? Vor allem, wenn die Quelle sowieso am Ende angegeben wird? Das verschlechtert die Lesbarkeit enorm und es ist ja eigentlich nicht wirklich abgeschrieben, sondern die sprachlichen Möglichkeiten sind halt begrenzt. \\
Der spezielle Satz hier ist mir gar nicht so wichtig, nur würde ich gerne mal die Jessica (evtl. sogar Frau Siegmann) fragen, wie das in so einem Fall ist.} Das wohl bekannteste Einsatzgebiet von Ultraschall-Technik ist die Medizin. Sowohl für diagnostische wie auch für therapeutische Zwecke können dort viele Anwendungen gefunden werden.