Die hier gemessene Schallgeschwindigkeit für Acryl kommt sehr nahe an den Literaturwert\footnote{\url{http://www.olympus-ims.com/de/ndt-tutorials/thickness-gage/appendices-velocities/}} heran. Die Durchschallungs Methode liefert etwas genauere Ergebnisse (siehe Tabelle \ref{tab:literaturwert}). \\
Leider ist die die Bestimmung der Dicke der Acrylplatten vergleichsweise schlecht.
Die ausgerechneten Werte zeigen große Abweichungen von den mit der Schieblehre bestimmten Werten (siehe Tabelle \ref{tab:dicke}). Das könnte daran liegen, dass die Anpassungsschicht ebenso wie die Wasserschicht zwischen den  Platten  nicht in der Rechnung berücksichtigt werden. Die Schallgeschwindigkeit in Wasser ist geringer als in Acryl, was dazu führt, dass die errechnete Dicke größer ist als die reelle.\\
Auch die ausgerechneten Dicken der Anpassungsschichten sind so fehlerbehaftet, dass sie kaum aussagekräftig sind. Allerdings ist zu erkennen, dass beim Durchschallungs-Verfahren die Schicht zirka doppelt so dick ist. Das entspricht der Erwartung, da auf beiden Seiten des Zylinders eine Kontaktmittelschicht ist.



   \begin{figure}[h!]
   	\centering
   	\captionof{table}{Schallgeschwindigkeit in Acryl}
   	\begin{tabular}{c|c|c|c}
   		&Literaturwert in \si{\meter\per\second}& Gemessener Wert in \si{\meter\per\second} & Abweichung\\
   		\hline
	Echo-Impuls & 2730 & 2693 & \SI{-1.35}{\percent} \\
	Durchschallung & 2730 & 2728 & \SI{0.07}{\percent}
   	\end{tabular}
   	\label{tab:literaturwert}
   \end{figure}
   
      \begin{figure}[h!]
      	\centering
      	\captionof{table}{Schichtdicke der Acrylplatte}
      	\begin{tabular}{c|c|c}
Schieblehre & Ultraschall & Abweichung \\
\hline
\SI{6.1}{mm} & \SI{12.5}{mm} & \SI{104.9}{\percent} \\
\SI{11.85}{mm} & \SI{17.5}{mm} & \SI{47.7}{\percent} \\
      	\end{tabular}
      	\label{tab:dicke}
      \end{figure}

