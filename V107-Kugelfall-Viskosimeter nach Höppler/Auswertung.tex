\subsection{Berechnung von Kugelvolumen und -dichte}
\begin{table}[h!]
\centering
\begin{tabular}{c | c | c}
	& groß & klein \\
	\hline
	& 15.802 & 15.651 \\
	& 15.796 & 15.652 \\
	& 15.804 & 15.650 \\
	\hline
	Mittelwert & \SI{15.801(2)}{} & \SI{15.651(1)}{}
\end{tabular}
\caption{Durchmesser $d$ der beiden Kugeln in \SI{e-3}{\metre}}
\label{fig:DatenKugel}
\end{table}
Aus den Durchmessern der zwei Glaskugeln (siehe Tabelle \ref{fig:DatenKugel}) ergeben sich die Volumina
\begin{align}
	V_\text{kl} &= \SI{2.0074(2)e-6}{\metre\cubed} \quad \text{und} \\
	V_\text{gr} &= \SI{2.0655(8)e-6}{\metre\cubed} \ .
\end{align}
Die kleinere Kugel wiegt
\[ m_\text{kl} = \SI{4.44e-3}{\kilo\gram} \]
und die größere
\[ m_\text{gr} = \SI{4.63e-3}{\kilo\gram} \ , \]
damit können auch die Dichten
\begin{align}
	\rho_\text{kl} &= \SI{2211.9(2)}{\kilo\gram\per\metre\cubed} \quad \text{und} \\
	\rho_\text{gr} &= \SI{2241.6(8)}{\kilo\gram\per\metre\cubed}
\end{align}
berechnet werden. Die Fehler ergeben sich hierbei durch die Gaußsche Fehlerfortpflanzung
\begin{align}
	V(d) &= \frac{4}{3}\pi\left(\frac{d}{2}\right)^3: && \sigma_{V} = \left|\pdv{V}{d}\sigma_{d}\right| = 2\pi\left(\frac{d}{2}\right)^2\sigma_{d} \\
	\rho(d) &= \frac{m}{\frac{4}{3}\pi\left(\frac{d}{2}\right)^3}: && \sigma_{\rho} = \left|\pdv{\rho}{d}\sigma_{d}\right| = \frac{9m}{2\pi\left(\frac{d}{2}\right)^4}\sigma_{d} \ .
\end{align}
\clearpage

\subsection{Bestimmung der Apparatekonstanten}
\todo[color=red, inline]{Ich meine, Lena hätte gesagt, dass die Aperatkonstante überhaupt nicht mehr stimmt. Ich würde daher die Viskosität von Wasser bei 20 Grad nachschlagen  (\SI{1.005}{\milli\pascal\second}) und daraus die Konstante berechnen. Und die Viskosität gibt man standartmäßig in \si{\pascal\second} an. Abgesehen davon ist der Wert $\eta_{20}$ in der falschen Größenordnung. Als ich in gerade nachgerechnet habe, kam \SI{1.79}{\milli\pascal\second} heraus. }
\todo[color = green, inline]{Ah ja, das mit der Apparatekonstante stimmt. Das hatte ich vergessen. \\
Oh ja. Bei dem $\eta_{20}$ ist das milli verloren gegangen. \\
Die Einheit würde ich gerne beibehalten, weil Pascal keine SI-Einheit ist.}
\todo[color=green, inline]{Gegenargument: Newton ist auch keine SI-Einheit und ich würde trotzdem nicht \si{\kilo\gram\meter\per\second\squared} schreiben. Ich glaube Einheiten sind viel mehr eine Frage der Gewohnheit (solange sie sich aus SI-Einheiten zusammensetzen). Sehr wichtig ist mir das jetzt aber ehrlich gesagt nicht wirklich ;-) }
\todo[inline]{Na, gut. Gegen das Newton-Argument bleibt mir natürlich nichts :P}
Die Messung der Fallzeit ergibt die Werte in Tabelle \ref{fig:DatenZeit}.
\begin{table}[h!]
\centering
\begin{tabular}{c|c|c}
	& klein & groß \\
	\hline
	& 12.91 & 92.62 \\
	& 12.87 & 92.35 \\
	& 13.00 & 92.44 \\
	& 12.78 & 92.07 \\
	& 12.59 & 93.31 \\
	& 12.73 & 92.72 \\
	& 12.93 & 93.71 \\
	& 12.76 & 91.91 \\
	& 12.85 & 92.25 \\
	& 12.70 & 91.95 \\
	\hline
	Mittelwerte & \SI{12.81(4)}{} & \SI{92.5(2)}{}
\end{tabular}
\caption{Fallzeiten in \si{\second} für ein \SI{0.10}{\metre} langes Rohr}
\label{fig:DatenZeit}
\end{table} \\
Mit der Viskosiät von Wasser bei \SI{20}{\celsius} bzw. \SI{293.15}{\kelvin}\footnote[3]{W. Walcher: Praktikum der Physik, Teubner Studienbücher, 1985, Tabellen-Anhang 1.7}
\[ \eta_{20} = \SI{1.002e-3}{\kilo\gram\per\metre\per\second} \]
und der Dichte von Wasser bei derselben Temperatur\footnotemark[3]
\[ \rho_\text{W} = \SI{992.8}{\kilo\gram\per\metre\cubed} \]
können durch Umstellen und Einsetzen in Formel \eqref{Visk} die Apparatekonstanten für die kleine
\begin{align}
	K_\text{kl} = \frac{\eta_{20}}{(\rho_\text{kl}-\rho_\text{W})t_\text{kl}} = \SI{6.41(2)e-8}{\metre\squared\per\second\squared}
\end{align}
und die große Kugel
\begin{align}
	K_\text{gr} = \SI{8.67(2)e-9}{\metre\squared\per\second\squared}
\end{align}
bestimmt werden.
Auch bei diesen Werten ergibt sich der Fehler durch die Gaußsche Fehlerfortpflanzung
\begin{align}
	K &= \frac{\eta}{(\rho-\rho_\text{W})t}: && \sigma_K = \sqrt{\left(\pdv{K}{\rho}\sigma_\rho\right)^2+\left(\pdv{K}{t}\sigma_t\right)^2} \\
	& && \quad\ = \sqrt{\left(\frac{\eta\sigma_\rho}{(\rho-\rho_\text{W})^2t}\right)^2+\left(\frac{\eta\sigma_t}{(\rho-\rho_\text{W})t^2}\right)^2} \ .
\end{align}
\clearpage

\subsection{Konstantenbestimmung der Andradeschen Gleichung}
Bei der Fallzeit-Messung mit ansteigender Wasser-Temperatur werden die Werte in Tabelle \ref{fig:DatenTemperatur} gemessen. Eingesetzt in Gleichung \eqref{Visk} kann so die Viskosität des Wassers in Abhängigkeit der Temperatur berechnet werden. Diese Werte finden sich in derselben Tabelle. Wieder berechnet sich der Fehler der Viskosität nach Gauß ($K$ und $\rho$ sind dabei die Werte der großen Kugel):
\begin{align}
	\eta = K(\rho-\rho_\text{W})t: \qquad \sigma_{\eta} &= \sqrt{\left(\pdv{\eta}{K}\sigma_K\right)^2+\left(\pdv{\eta}{\rho}\sigma_{\rho}\right)^2 + \left(\pdv{\eta}{t}\sigma_{t}\right)^2} \\
	&= \sqrt{\left((\rho-\rho_\text{W})t\right)^2+\left(Kt\sigma_\rho\right)^2+\left(K(\rho-\rho_\text{W})t\sigma_t\right)^2} \ .
\end{align}
\begin{table}[h!]
\centering
\begin{tabular}{c|c|c|c|c}
	$T$ in \SI{}{\celsius} & \multicolumn{2}{ |c| }{Fallzeit in \si{\second}} & Mittelwert der Zeitmessungen & Viskosität $\eta(T)$ in \si{\metre\squared\per\second\squared} \\
	\hline
	28 & 83.25 & 78.72 & \SI{81(2)}{} & \SI{1.04(3)}{} \\
	31 & 74.78 & 73.89 & \SI{74.3(4)}{} & \SI{0.953(7)}{} \\
	35 & 67.97 & 76.66 & \SI{72(4)}{} & \SI{0.93(6)}{} \\
	40 & 62.19 & 62.64 & \SI{62.4(2)}{} & \SI{0.800(4)}{} \\
	45 & 57.69 & 56.56 & \SI{57.1(6)}{} & \SI{0.732(8)}{} \\
	51 & 51.68 & 51.78 & \SI{51.73(5)}{} & \SI{0.663(2)}{} \\
	55 & 48.21 & 49.78 & \SI{49.0(8)}{} & \SI{0.63(1)}{} \\
	60 & 45.63 & 45.06 & \SI{45.3(3)}{} & \SI{0.581(4)}{} \\
	65 & 42.47 & 41.50 & \SI{42.0(5)}{} & \SI{0.538(6)}{} \\
\end{tabular}
\caption{Fallzeiten der großen Kugel für ein \SI{0.10}{\metre} langes Rohr bei verschiedenen Wassertemperaturen und daraus berechnete Viskositäten}
\label{fig:DatenTemperatur}
\end{table}
 \\
Wird nun Gleichung \eqref{Andra} auf beiden Seiten logarithmiert, ergibt sich
\[ \ln\eta = \ln A + \frac{B}{T} \ , \]
mit
\[ X = \frac{1}{T} \quad \text{und} \quad Y = \ln\eta \]
kann so eine lineare Ausgleichsrechung mit den Werten in Tabelle \ref{fig:DatenRegression} durchgeführt werden.
\begin{table}[h!]
\centering
\begin{tabular}{c|c}
	$X=\frac{1}{T}$ in \SI{e-3}{\per\kelvin} & $Y=\ln\eta$ in \si{\ln\kilogram\per\metre\per\second} \\
	\hline
	3.41 & -6.91 \\
	3.32 & -7.04 \\
	3.29 & -7.12 \\
	3.25 & -7.15 \\
	3.19 & -7.30 \\
	3.14 & -7.39 \\
	3.08 & -7.49 \\
	3.05 & -7.54 \\
	3.00 & -7.62 \\
	2.96 & -7.70
\end{tabular}
\caption{Werte, mit denen die Regression durchgeführt wird}
\label{fig:DatenRegression}
\end{table}

Mit Hilfe von Python errechnen sich die Konstanten
\begin{align}
	B &= \SI{1775(42)}{\kelvin} \\
	\ln A = \SI{-13.0(1)}{\ln\kilo\gram\per\metre\per\second} \quad \Rightarrow \quad A &= \SI{2.4(3)e-6}{\kilo\gram\per\metre\per\second} \ .
\end{align}
Die Gerade ist mit den Regressionswerten in Abbildung \ref{fig:Regression} zu sehen.
\begin{figure}[h!]
	\centering
	\includegraphics[width = 0.9\textwidth]{Regression.png}
	\caption{Regressiongerade mit Regressionswerten nach \eqref{Andra}}
	\label{fig:Regression}
\end{figure}
Die Andradesche Gleichung ist somit
\begin{align}
	\eta(T) = \SI{2.4e-6}{}\exp\left(\frac{1775}{T}\right) \ .
\end{align}
\clearpage

\subsection{Die Reynoldsche Zahl}
\todo[inline, color = red]{In meinem Buch (Fußnote oben) steht wirklich, dass der Kugelradius, nicht der Durchmesser relevant ist. In dem Altprotokoll von der Fachschaft nehmen die jedoch den Radius. Im Prinzip ist es ja auch egal. Ich finde nur, wir sollten eine Quelle zur Formel haben.}
\todo[inline]{In allen Quellen (leider immer nur Internet), die ich gelesen habe, steht, das muss die Länge des Körpers in Strömungsrichtung sein. Bei uns also der Durchmesser der Kugel. \\
Ich habe mir aber jetzt mal mehr Information gesucht und auch in den Geschke auf Google reingeschaut. Scheinbar ist das so: Ganz grundsätzlich kann man sich das hindefinieren, wie man will. Das ist ja keine physikalische Größe, sondern ein Vergleichswert. Häufig wird dann diese \glqq charakteristische Länge\grqq benutzt, weil das bei nicht kugelförmigen Gegenständen eine sehr einfach zu berechnende Größe ist. Zum Beispiel bei Flugzeugen ist das sehr gut vergleichbar, sodass auch die Reynoldszahlen gut vergleichbar sind. In so Beschreibungen für Praktikumsversuche, wo sowieso immer alles gerade und schön und vor allem rund ist, wird dann gerne der Radius der Kugel verwendet, da das halt die Standard-Kugel-Größe ist. \\
Ich habe noch ne nette Umformung dazu gefunden ($A$ ist die Grundfläche, $V$ das Volumen und $m$ die Masse):}
%\begin{align*}
%	\text{RE} &= \frac{\rho vl}{\eta} \\
%	&= \frac{m v V}{V \eta A} \\
%	&= \frac{m v^2}{\eta A v} \\
%	&= \frac{2 E_\text{kin}}{\eta A v} \\
%	&= \frac{2 E_\text{kin}}{W_\text{Reibung}}
%\end{align*}
\todo[inline]{Wenn man statt der charakteristischen Länge, lieber den Radius nehmen möchte, dann kommt halt $\frac{E_\text{kin}}{W_\text{Reibung}}$ raus. \\
Unterm Strich: Lass uns den Radius nehmen, wenn du dazu ne Buchquelle hast, ich passe die Auswertung dann an.}
Zuletzt soll die Reynoldsche Zahl berechnet werden. Dafür wird benötigt
\begin{itemize}
	\item die Dichte von Wasser, sie ist eigentlich von der Temperatur abhängig, schwankt aber kaum im betrachteten Temperaturbereich, sodass weiterhin \[ \rho_\text{W}=\SI{992.8}{\kilo\gram\per\metre\cubed} \] angenommen wird;
	\item die Fließgeschwindigkeit des Wassers, welche gleich der Fallgeschwindigkeit
	\[ v = \frac{s}{t}, \quad s=\SI{0.10}{\metre} \] der Kugel ist;
	\item eine \glqq charakteristischen Länge\grqq\ der Kugel, das kann beispielsweise der Durchmesser sein, hier wird der Radius der Kugel \[ \frac{d}{2} = \SI{7.900e-3}{\metre} \] verwendet; und
	\item die temperaturabhängige Viskosität $\eta$.
\end{itemize}
Eingesetzt in \eqref{Reynolds} ergeben sich so die Werte in Tabelle \ref{fig:Reynolds}, die Fehler wiederum mit Gauß:
\begin{align}
	\text{RE} = \frac{\rho_\text{W} s d}{2\eta t}: \qquad \sigma_{\text{RE}} &= \sqrt{\left(\pdv{\text{RE}}{t}\sigma_t\right)^2+\left(\pdv{\text{RE}}{d}\sigma_d\right)^2+\left(\pdv{\text{RE}}{\eta}\sigma_\eta\right)^2} \\
	&= \sqrt{\left(-\frac{\rho_\text{W}sd}{2\eta t^2}\sigma_t\right)^2+\left(\frac{\rho_\text{W}s}{2\eta t}\sigma_d\right)^2+\left(-\frac{\rho_\text{W}sd}{2\eta^2 t}\sigma_\eta\right)^2} \ .
\end{align}
\begin{table}[h!]
\centering
\begin{tabular}{c|c}
	Temperatur in \si{\celsius} & $\text{RE}$ \\
	\hline
	20 & \SI{8.54(3)}{} \\
	28 & \SI{11.0(6)}{} \\
	31 & \SI{13.1(2)}{} \\
	35 & \SI{14(2)}{} \\
	40 & \SI{18.6(1)}{} \\
	45 & \SI{22.2(4)}{} \\
	51 & \SI{27.07(7)}{} \\
	55 & \SI{30(1)}{} \\
	60 & \SI{35.2(4)}{} \\
	65 & \SI{41(1)}{}
\end{tabular}
\caption{Errechnete Reynolds-Zahlen bei verschiedenen Temperaturen}
\label{fig:Reynolds}
\end{table}
