\subsection{Berechnung von Kugelvolumen und -dichte}
\begin{table}[h!]
\centering
\begin{tabular}{c | c | c}
	& groß & klein \\
	\hline
	& 15.802 & 15.651 \\
	& 15.796 & 15.652 \\
	& 15.804 & 15.650 \\
	\hline
	Mittelwert & \SI{15.801(2)}{} & \SI{15.651(1)}{}
\end{tabular}
\caption{Durchmesser $d$ der beiden Kugeln in \SI{e-3}{\metre}}
\label{fig:DatenKugel}
\end{table}
Aus den Durchmessern der zwei Glaskugeln (siehe Tabelle \ref{fig:DatenKugel}) ergeben sich die Volumina
\begin{align}
	V_\text{kl} &= \SI{2.0074(2)e-6}{\metre\cubed} \quad \text{und} \\
	V_\text{gr} &= \SI{2.0655(8)e-6}{\metre\cubed} \ .
\end{align}
Die kleinere Kugel wiegt
\[ m_\text{kl} = \SI{4.44e-3}{\kilo\gram} \]
und die größere
\[ m_\text{gr} = \SI{4.63e-3}{\kilo\gram} \ , \]
damit können auch die Dichten
\begin{align}
	\rho_\text{kl} &= \SI{2211.9(2)}{\kilo\gram\per\metre\cubed} \quad \text{und} \\
	\rho_\text{gr} &= \SI{2241.6(8)}{\kilo\gram\per\metre\cubed}
\end{align}
berechnet werden. Die Fehler ergeben sich hierbei durch die Gaußsche Fehlerfortpflanzung
\begin{align}
	V(d) &= \frac{4}{3}\pi\left(\frac{d}{2}\right)^3: && \sigma_{V} = \left|\pdv{V}{d}\sigma_{d}\right| = 2\pi\left(\frac{d}{2}\right)^2\sigma_{d} \\
	\rho(d) &= \frac{m}{\frac{4}{3}\pi\left(\frac{d}{2}\right)^3}: && \sigma_{\rho} = \left|\pdv{\rho}{d}\sigma_{d}\right| = \frac{9m}{2\pi\left(\frac{d}{2}\right)^4}\sigma_{d} \ .
\end{align}
\clearpage

\subsection{Bestimmung der Apparatekonstante für die große Kugel}
Die Messung der Fallzeit ergibt die Werte in Tabelle \ref{fig:DatenZeit}.
\begin{table}[h!]
\centering
\begin{tabular}{c|c|c}
	& klein & groß \\
	\hline
	& 12.91 & 92.62 \\
	& 12.87 & 92.35 \\
	& 13.00 & 92.44 \\
	& 12.78 & 92.07 \\
	& 12.59 & 93.31 \\
	& 12.73 & 92.72 \\
	& 12.93 & 93.71 \\
	& 12.76 & 91.91 \\
	& 12.85 & 92.25 \\
	& 12.70 & 91.95 \\
	\hline
	Mittelwerte & \SI{12.81(4)}{} & \SI{92.5(2)}{}
\end{tabular}
\caption{Fallzeiten in \si{\second} für ein \SI{0.10}{\metre} langes Rohr}
\label{fig:DatenZeit}
\end{table}
Da die Apparatekonstante für die kleinere Kugel
\[ K_\text{kl} = \SI{0.07640e-6}{\metre\squared\per\second\squared} \]
bekannt ist, kann die Viskosität für \SI{20}{\celsius} bzw. \SI{293.15}{\kelvin} warmes Wasser ($\rho_\text{W} = \SI{1000}{\kilo\gram\per\metre\cubed}$) mit Formel \eqref{Visk} berechnet werden:
\begin{align}
	\eta_{20} = \SI{1.186(3)}{\kilo\gram\per\metre\per\second} \ .
\end{align}
Durch Umstellen ebendieser Formel ergibt sich mit der Dichte und den Fallzeiten für die große Kugel die Apparatekonstante
\begin{align}
	K_\text{gr} = \frac{\eta_{20}}{(\rho_\text{gr}-\rho_\text{W})\cdot t_\text{gr}} = \SI{1.032(4)e-5}{\metre\squared\per\second\squared} \ .
\end{align}
Auch bei diesen beiden Werten ergibt sich der jeweilige Fehler durch die Gaußsche Fehlerfortpflanzung (aus Gründen der Übersichtlichkeit wird auf die Indizes verzichtet, $\rho_\text{Wasser}=\rho_\text{W}$)
\begin{align}
	\eta &= K(\rho-\rho_\text{W})t: && \sigma_{\eta} = \sqrt{\left(\pdv{\eta}{\rho}\sigma_{\rho}\right)^2 + \left(\pdv{\eta}{t}\sigma_{t}\right)^2} \\
	& && \quad= \sqrt{\left(Kt\sigma_\rho\right)^2+\left(K(\rho-\rho_\text{W})t\sigma_t\right)^2} \\
	K &= \frac{\eta}{(\rho-\rho_\text{W})t}: && \sigma_K = \sqrt{\left(\pdv{K}{\eta}\sigma_\eta\right)^2+\left(\pdv{K}{\rho}\sigma_\rho\right)^2+\left(\pdv{K}{t}\sigma_t\right)^2} \\
	& && \quad\ = \sqrt{\left(\frac{\sigma_\eta}{(\rho-\rho_\text{W})t}\right)^2+\left(\frac{\eta\sigma_\rho}{(\rho-\rho_\text{W})^2t}\right)^2+\left(\frac{\eta\sigma_t}{(\rho-\rho_\text{W})t^2}\right)^2} \ .
\end{align}
\clearpage

\subsection{Konstantenbestimmung der Andraschen Gleichung}
Bei der Fallzeit-Messung mit ansteigender Wasser-Temperatur werden die Werte in Tabelle \ref{fig:DatenTemperatur} gemessen. Eingesetzt in Gleichung \eqref{Visk} kann so die Viskosität des Wassers in Abhängigkeit der Temperatur berechnet werden. Diese Werte finden sich in derselben Tabelle. Wieder berechnet sich der Fehler der Viskosität nach Gauß ($K$ und $\rho$ sind dabei die Werte der großen Kugel):
\begin{align}
	\eta = K(\rho-\rho_\text{W})t: \qquad \sigma_{\eta} &= \sqrt{\left(\pdv{\eta}{K}\sigma_K\right)^2+\left(\pdv{\eta}{\rho}\sigma_{\rho}\right)^2 + \left(\pdv{\eta}{t}\sigma_{t}\right)^2} \\
	&= \sqrt{\left((\rho-\rho_\text{W})t\right)^2+\left(Kt\sigma_\rho\right)^2+\left(K(\rho-\rho_\text{W})t\sigma_t\right)^2} \ .
\end{align}
\begin{table}[h!]
\centering
\begin{tabular}{c|c|c|c|c}
	$T$ in \SI{}{\celsius} & \multicolumn{2}{ |c| }{Fallzeit in \si{\second}} & Mittelwert der Zeitmessungen & Viskosität $\eta(T)$ in \si{\metre\squared\per\second\squared} \\
	\hline
	28 & 83.25 & 78.72 & \SI{81(2)}{} & \SI{1.04(3)}{} \\
	31 & 74.78 & 73.89 & \SI{74.3(4)}{} & \SI{0.953(7)}{} \\
	35 & 67.97 & 76.66 & \SI{72(4)}{} & \SI{0.93(6)}{} \\
	40 & 62.19 & 62.64 & \SI{62.4(2)}{} & \SI{0.800(4)}{} \\
	45 & 57.69 & 56.56 & \SI{57.1(6)}{} & \SI{0.732(8)}{} \\
	51 & 51.68 & 51.78 & \SI{51.73(5)}{} & \SI{0.663(2)}{} \\
	55 & 48.21 & 49.78 & \SI{49.0(8)}{} & \SI{0.63(1)}{} \\
	60 & 45.63 & 45.06 & \SI{45.3(3)}{} & \SI{0.581(4)}{} \\
	65 & 42.47 & 41.50 & \SI{42.0(5)}{} & \SI{0.538(6)}{} \\
\end{tabular}
\caption{Fallzeiten der großen Kugel für ein \SI{0.10}{\metre} langes Rohr bei verschiedenen Wassertemperaturen und daraus berechnete Viskositäten}
\label{fig:DatenTemperatur}
\end{table}




Wird nun Gleichung \eqref{Andra} auf beiden Seiten logarithmiert, ergibt sich
\[ \ln\eta = \ln A + \frac{B}{T} \ , \]
mit
\[ X = \frac{1}{T} \quad \text{und} \quad Y = \ln\eta \]
kann so eine lineare Ausgleichsrechung mit den Werten in Tabelle \ref{fig:DatenRegression} durchgeführt werden.
\begin{table}[h!]
\centering
\begin{tabular}{c|c}
	$X=\frac{1}{T}$ in \SI{e-3}{\per\kelvin} & $Y=\ln\eta$ in \si{\ln\kilogram\per\metre\per\second} \\
	\hline
	3.41 & -6.91 \\
	3.32 & -7.04 \\
	3.29 & -7.12 \\
	3.25 & -7.15 \\
	3.19 & -7.30 \\
	3.14 & -7.39 \\
	3.08 & -7.49 \\
	3.05 & -7.54 \\
	3.00 & -7.62 \\
	2.96 & -7.70
\end{tabular}
\caption{Werte, mit denen die Regression durchgeführt wird}
\label{fig:DatenRegression}
\end{table}

Mit Hilfe von Python errechnen sich die Konstanten
\begin{align}
	B &= \SI{1804(52)}{\ln\second\squared\per\metre\squared} \\
	\ln A = \SI{-5.965(164)}{\ln\metre\squared\per\second\squared} \quad \Rightarrow \quad A &= \SI{0.003(1178)}{\metre\squared\per\second\squared} \ .
\end{align}
Die Gerade ist mit den Regressionswerten in Abbildung \ref{fig:Regression} zu sehen.
\begin{figure}[h!]
	\centering
	\includegraphics[width = 0.9\textwidth]{Regression.png}
	\caption{Regressiongerade mit Regressionswerten nach \eqref{Andra}}
	\label{fig:Regression}
\end{figure}
Die Andradsche Gleichung ist somit
\begin{align}
	\eta(T) = 0.003\exp\left(\frac{1804}{T}\right) \ .
\end{align}
\clearpage

\subsection{Die Reynoldsche Zahl}
Zuletzt soll die Reynoldsche Zahl berechnet werden. Dafür wird benötigt
\begin{itemize}
	\item die Dichte von Wasser, sie ist eigentlich von der Temperatur abhängig, schwankt aber kaum im betrachteten Temperaturbereich, sodass weiterhin \[ \rho_\text{W}=\SI{1000}{\kilo\gram\per\metre\cubed} \] angenommen wird;
	\item die Fließgeschwindigkeit des Wassers, welche gleich der Fallgeschwindigkeit
	\[ v = \frac{s}{t}, \quad s=\SI{0.10}{\metre} \] der Kugel ist;
	\item der Durchmesser des Zylinders, welcher gleich dem Durchmesser der Kugel \[ d = \SI{15.801e-3}{\metre} \] angenommen werden kann; und
	\item die temperaturabhängige Viskosität $\eta$.
\end{itemize}
Eingesetzt in \eqref{Reynolds} ergeben sich so Werte zwischen
\begin{align}
	\text{RE}_\text{min} &= \text{RE}_{\eta(\SI{28}{\celsius})} = \SI{0.019(1)}{} \quad \text{und} \\
	\text{RE}_\text{max} &= \text{RE}_{\eta(\SI{65}{\celsius})} = \SI{0.070(2)}{} \ .
\end{align}
Und auch hier ergibt sich der Fehler wiederum mit Gauß:
\begin{align}
	\text{RE} = \frac{\rho_\text{W} s d}{\eta t}: \qquad \sigma_{\text{RE}} &= \sqrt{\left(\pdv{\text{RE}}{t}\sigma_t\right)^2+\left(\pdv{\text{RE}}{d}\sigma_d\right)^2+\left(\pdv{\text{RE}}{\eta}\sigma_\eta\right)^2} \\
	&= \sqrt{\left(-\frac{\rho_\text{W}sd}{\eta t^2}\sigma_t\right)^2+\left(\frac{\rho_\text{W}s}{\eta t}\sigma_d\right)^2+\left(-\frac{\rho_\text{W}sd}{\eta^2 t}\sigma_\eta\right)^2} \ .
\end{align}