\subsection{Bestimmung des Wärmestroms in Messing}
Diagramm 1 zeigt die Temperaturdifferenz zwischen den Temperaturen, die an den Thermoelementen T1 und T2 bei der statischen Messung gemessen wurden. Mit Hilfe eines Lineals wurde die Temperaturdifferenz im Diagramm (siehe Tabelle \ref{Delta T Messing}) gemessen.
\begin{table}[h!]
\begin{center}
\begin{tabular}{c | c}
	Zeit in \si{\second} & $\Delta T_\text{2-1}$ in \si{\celsius} \\
\hline
	50 & 4.91 \\
	100 & 4.60 \\
	150 & 3.96 \\
	200 & 3.49 \\
	250 & 3.19 \\
	300 & 2.98
\end{tabular}
\end{center}
\caption{Temperaturdifferenz bei statischer Messung in Messing}
\label{Delta T Messing}
\end{table}
Eingesetzt in Gleichung \eqref{Warmestrom} ist der gemittelte Wärmestrom in Messing
\begin{equation}
	\frac{\Delta Q}{\Delta t} = \SI{-1.89(27)e-4}{\joule\per\second},
\end{equation}
mit dem Literaturwert der Wärmeleitfähigkeit (siehe Diskussion) und dem Querschnitt\footnote{aus der Versuchsanleitung zu V204: Wärmeleitung von Metallen, Anfängerpraktikum, TU Dortmund}
\begin{equation}
	A = \SI{48}{\milli\metre\squared}.
\end{equation}
\clearpage

\subsection{Bestimmung der Wärmeleitfähigkeit}
Nach den dynamischen Messungen wurden die Diagramme 2 (Temperaturen bei Messing), 3 (Temperaturen bei Aluminium) und 4 (Temperaturen bei Edelstahl) ausgedruckt. Mit einem Lineal wurden jeweils die Phasenverschiebung und die Amplituden im Diagramm gemessen. Die Amplitudenmessung erfolgte dabei wie folgt: Zunächst wurde $\Delta T$ an mehreren Stellen bei steigender Temperatur gemessen, also zwischen Tief- und Hochpunkt. Danach wurde $\Delta T$ beim jeweils nachfolgenden Temperaturabfall, also vom Hoch- zum Tiefpunkt, gemessen. Die jeweils zusammengehörenden Werte wurden addiert und durch vier geteilt. \\
Die benötigten Konstanten (Tabelle \ref{Konstanten}) stammen aus der Versuchsanleitung zu V204: Wärmeleitung von Metallen des Anfängerpraktikums der TU Dortmund.
\begin{center}
\begin{table}[h]
\begin{tabular}{c | c | c | c | c}
	& lin. Ausdehnungs- & Kompressions- & Molvolumen & Molmasse \\
	& koeffizient & modul & & \\
	& in \SI{e-6}{\per\kelvin} & in \SI{e+9}{\newton\per\metre\squared} & in \si{\cubic\metre\per\mol} & in \si{\gram\per\mol} \\
	\hline
	Blei & 29.0 & 42 & \SI{1.826e-05}{} & 207.2 \\
	Graphit & 8 & 33 & \SI{5.357e-06}{} & 12.0
\end{tabular}
\caption{Benötigte Konstanten\tablefootnote{aus Versuchsanleitung zu V201, Anfängerpraktikum TU Dortmund}}
\label{Konstanten}
\end{table}
\end{center}


\begin{table}
\begin{center}
\begin{tabular}{c | c | c}
	$A_\text{nah}$ in \si{\celsius} & $A_\text{fern}$ in \si{\celsius} & $\Delta t$ in \si{\second} \\
\hline
	3.26 & 0.98 & 8.33 \\
	3.41 & 1.10 & 6.25 \\
	3.49 & 1.14 & 6.25 \\
	3.56 & 1.02 & 8.33 \\
	3.45 & 1.06 & 8.33 \\
	3.07 & 0.95 & 8.33 \\
	3.56 & 1.10 & 8.33 \\
	3.41 & 0.98 & 8.33 \\
\hline
	\SI{3.40(2)}{} & \SI{1.04(1)}{} & \SI{7.81(11)}{}
\end{tabular}
\end{center}
\caption{Amplituden und Phasendifferenz bei der dynamischen Messung bei Messing inklusive Mittelwerte}
\label{Amplituden Messing}
\end{table}

\begin{table}
\begin{center}
\begin{tabular}{c | c | c}
	$A_\text{nah}$ in \si{\celsius} & $A_\text{fern}$ in \si{\celsius} & $\Delta t$ in \si{\second} \\
\hline
	2.93 & 1.59 & 6.25 \\
	3.17 & 1.71 & 6.25 \\
	3.23 & 1.83 & 6.25 \\
	3.17 & 1.71 & 6.25 \\
	3.26 & 1.72 & 8.33 \\
	2.99 & 1.59 & 8.33 \\
	3.38 & 1.86 & 8.33 \\
	3.23 & 1.71 & 8.33 \\
\hline
	\SI{3.17(2)}{} & \SI{1.72(1)}{} & \SI{7.29(13)}{}
\end{tabular}
\end{center}
\caption{Amplituden und Phasendifferenz bei der dynamischen Messung bei Aluminium inklusive Mittelwerte}
\label{Amplituden Aluminium}
\end{table}

\begin{table}
\begin{center}
\begin{tabular}{c | c | c}
	$A_\text{nah}$ in \si{\celsius} & $A_\text{fern}$ in \si{\celsius} & $\Delta t$ in \si{\second} \\
\hline
	2.52 & 0.52 & 41.10 \\
	2.59 & 0.50 & 34.25 \\
	2.57 & 0.45 & 41.10 \\
	2.90 & 0.45 & 41.10 \\
	2.67 & 0.42 & 41.10 \\
	2.71 & 0.42 & 47.95 \\
	2.67 & 0.40 & 47.95 \\
	2.62 & 0.33 & 41.10 \\
	2.59 & 0.38 & 54.79 \\
	2.67 & 0.38 & 47.95 \\
	2.74 & 0.40 & 47.95 \\
\hline
	\SI{2.66(1)}{} & \SI{0.42(1)}{} & \SI{44.2(5)}{}
\end{tabular}
\end{center}
\caption{Amplituden und Phasendifferenz bei der dynamischen Messung bei Edelstahl inklusive Mittelwerte}
\label{AmplitudenEdelstahl}
\end{table}

Mit den jeweiligen Werten aus den Tabellen, Formel \eqref{Warmekappazitat} und einem Abstand der Thermoelemente $\Delta x = \SI{3}{\centi\metre}$ ergeben sich folgende Wärmeleitfähigkeiten:
\begin{align}
	&\text{Messing:}\qquad & \kappa_\text{M} &= \SI{159.7(26)}{\watt\per\metre\per\kelvin} \\
	&\text{Aluminium:}\qquad & \kappa_\text{A} &= \SI{233.4(53)}{\watt\per\metre\per\kelvin} \\
	&\text{Edelstahl:}\qquad & \kappa_\text{E} &= \SI{17.7(2)}{\watt\per\metre\per\kelvin}\quad.
\end{align}


\subsection{Vergleich der Temperaturverläufe}
Diagramm VERWEIS und Diagramm VERWEIS zeigen die Temperaturverläufe der entfernteren Thermoelemente bei der statischen Messung. Wie erwartet nähern sich die Kurven mit kleiner werdender positiver Steigung einem Grenzwert. Die Verläufe von Messing und Aluminium sind vergleichbar, die Temperatur in Edelstahl steigt sehr viel langsamer an. Der Vergleich der beiden Messingstäbe zeigt, dass ein breiterer Stab mehr Wärme leitet. \\
In den Diagrammen VERWEIS und VERWEIS sind die Temperaturdifferenzen $T_\text{fern}-T_\text{nah}$ des breiten Messingstab und Edelstahl bei statischer Messung zu sehen. Bei beiden zeigt sich, dass die Temperaturen sich nach einiger Zeit annähern. Erwartungsgemäß geschieht dies bei Messing sehr viel schneller, da laut der Berechnung, Messing eine erheblich höhere Wärmeleitfähigkeit besitzt.