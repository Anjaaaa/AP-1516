In diesem Experiment wird die Wärmeleitung von vier Metallstäben gemessen: Ein Stab ist aus Aluminium, einer aus Edelstahl und zwei mit verschiedenen Querschnitten aus Kupfer.
Die Stäbe werden zeitgleich mit einem Peltier-Element erhitzt bzw. gekühlt. Die Temperatur der Stäbe wird jeweils an zwei Stellen eines Stabes mit Thermoelementen gemessen und in einem Datenlogger gespeichert. \\
Als erstes werden die Stäbe einfach nur erhitzt, um den Wärmestrom $\frac{\Delta Q}{\Delta t}$ des Messingstabes zu bestimmen. Hierzu werden die Temperaturkurven sowie die Temperaturdifferenzen der Thermoelemente mit Hilfe des Datenloggers erstellt und ausgedruckt. \\
In weiteren Messungen wird periodisch geheizt und gekühlt. Für Messing und Aluminium werden die Messung bei einer Periode von \SI{80}{\second} durchgeführt und für Edelstahl bei  \SI{200}{\second}. Es wird für jeden Stab eine Grafik mit den Temperaturkurven beider Thermoelemente erstellt.
