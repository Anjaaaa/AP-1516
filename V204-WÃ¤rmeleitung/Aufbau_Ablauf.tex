In diesem Experiment wurde die Wärmeleitung von vier Metallstäben gemessen: Ein Stab war aus Aluminium, einer aus Edelstahl und zwei mit verschiedenen Querschnitten aus Kupfer.
Die Stäbe wurden zeitgleich mit einem Peltier-Element erhitzt bzw. gekühlt. Die Temperatur der Stäbe wurde jeweils an zwei Stellen eines Stabes mit Thermoelementen gemessen und in einem Datenlogger gespeichert. \\
Als erstes wurden die Stäbe einfach nur erhitzt, um den Wärmestrom $\frac{\Delta Q}{\Delta t}$ des Messingstabes zu bestimmen. Hierzu wurden die Temperaturkurven sowie die Temperaturdifferenzen der Thermoelemente mit Hilfe des Datenloggers erstellt und ausgedruckt. \\
In weiteren Messungen wurde periodisch geheizt und gekühlt. Für Messing und Aluminium wurde die Messung bei einer Periode von \SI{80}{\second} durchgeführt und für Edelstahl bei  \SI{200}{\second}. Es wurde für jeden Stab eine Grafik mit den Temperaturkurven beider Thermoelemente erstellt.
