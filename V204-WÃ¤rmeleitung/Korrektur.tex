\subsection{Analyse der Temperaturwellen}
Da die Frequenz einer Welle das Inverse der Periodenlänge ist, ergibt sich
\begin{align}
	&f_\text{Messing} = f_\text{Aluminium} &&= \frac{1}{80}\si{\hertz} \\
	&f_\text{Edelstahl} &&= \frac{1}{200}\si{\hertz}.
\end{align}
Aus Funktion \eqref{Welle} kann die Wellenzahl
\[k = \sqrt{\frac{\omega\rho c}{2\kappa}} = \sqrt{\frac{\pi f\rho c}{\kappa}}\]
abgelesen werden, sodass für die Wellenlänge folgt
\[ \lambda = \frac{1}{k} = \sqrt{\frac{\kappa}{\pi f\rho c}}. \]
Mit den Konstanten aus Tabelle \ref{Konstanten} und den für $\kappa$ berechneten Werten, ergeben sich die Wellenlängen in Tabelle \ref{Wellenlangen}.
\begin{table}[h!]
\begin{center}
\begin{tabular}{c|c}
	Metall & Wellenlänge in \si{\centi\metre} \\
\hline
	Messing & \SI{3.52(3)}{} \\
	Aluminium & \SI{5.06(6)}{} \\
	Edelstahl & \SI{1.88(1)}{}
\end{tabular}
\caption{Wellenlängen der Temperaturwelle}
\label{Wellenlangen}
\end{center}
\end{table}