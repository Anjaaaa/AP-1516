In diesem Experiment gibt es zwei Hauptfehlerquellen. Direkt zu Beginn fällt auf, dass das Peltier-Element nicht zuverlässig heizt. Auf \glqq Kühlen\grqq ein\-ge\-stel\-lt war, heizt und kühlt es manchmal abwechselnd. Während der Messungen beobachteten wir stets eine Temperaturveränderung in die erwartete Richtung. Es ist jedoch fraglich, ob eine konstante Wärme- /Kältezufuhr gewährleistet ist. Dieser Fehler fließt sowohl in die statische und wie auch in die dynamische Messmethode mit ein. \\
Der zweite gravierende Fehler betrifft nur die dynamische Messmethode. Die periodische Wärmezufuhr wird per Hand generiert. Wenn einmal zu früh oder spät die Kühlung bzw. Heizung angestellt wird, veränderte sich direkt die Periode und somit auch der Phasenunterschied der beiden gemessenen Amplituden. Besonders bei der Edelstahlmessung mit großen Amplituden variiert $\Delta t$ stark (siehe Tabelle \ref{AmplitudenEdelstahl}). \\
Trotz dieser Fehler sind die Ergebnisse zufriedenstellend. 
Für den Wärmestrom kann leider kein Vergleichswert gefunden werden, da dieser neben dem Material auch von der Form des Stabes abhängt. \\ 
In Tabelle \ref{Literaturwert} werden Literaturwerte und gemessene Werte der dynamischen Methode gegenübergestellt. Komisch ist, dass Aluminium und Messing in unterschiedliche Richtungen abweichen, obwohl beide Messreihen zeitgleich unter identischen Bedingungen ablaufen.


\begin{table}[h!]
\begin{center}
\begin{tabular}{c|c|c|c}
	Material & $\kappa$ Literatur bei Raumtemperatur & $\kappa$ gemessen & Abweichung \\
\hline
	Aluminium & 237 & 233.4 & -1.5\% \\
	Messing & 120 & 159.7 & +33.1\% \\
	Edelstahl & 15 & 17.7 & +18.0\%\\
\end{tabular}
\end{center}
\caption[Vergleich zwischen Literatur- und gemessenem Wert der Wärmeleitfähigkeit]{Vergleich zwischen Literatur-\footnotemark\ und gemessenem Wert der Wärmeleitfähigkeit}
\end{table}
\footnotetext{\url{http://www.chemie.de/lexikon/Wärmeleitfähigkeit.html}}