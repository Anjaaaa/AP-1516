In diesem Experiment gab es zwei Hauptfehlerquellen. Direkt zu Beginn fiel auf, dass das Peltier-Element nicht zuverlässig heizte. Als es auf \glqq Kühlen\grqq ein\-ge\-stel\-lt war, heizte und kühlte es abwechselnd. Während der Messungen beobachteten wir stets eine Temperaturveränderung in die erwartete Richtung. Es ist jedoch fraglich, ob eine konstante Wärme- /Kältezufuhr gewährleistet war. Dieser Fehler fließt sowohl in die statische und wie auch in die dynamische Messmethode mit ein. \\
Der zweite gravierende Fehler betrifft nur die dynamische Messmethode. Die periodische Wärmezufuhr wurde per Hand generiert. Wenn einmal zu früh oder spät die Kühlung bzw. Heizung angestellt wurde, veränderte sich direkt die Periode und somit auch der Phasenunterschied der beiden gemessenen Amplituden. Besonders bei der Edelstahlmessung mit großen Amplituden variiert $\Delta t$ stark (siehe Tabelle \ref{AmplitudenEdelstahl}). \\
Trotz dieser Fehler sind zufriedenstellende Ergebnisse erzielt worden. 
Für den Wärmestrom konnte leider kein Vergleichswert gefunden werden, da dieser neben dem Material auch von der Form des Stabes abhängt. \\ 
In Tabelle \ref{Literaturwert} wurden Literaturwerte und gemessene Werte der dynamischen Methode gegenübergestellt. Komisch ist, dass Aluminium und Messing in unterschiedlische Richtungen abweichen, obwohl beide Messreihen zeitgleich unter identischen Bedingungen abliefen.


\begin{table}[h!]
\begin{center}
\begin{tabular}{c|c|c|c}
	Material & $\kappa_\text{Literatur}$ (bei Raumtemperatur) & $\kappa_\text{gemessen}$  & Abweichung \\
\hline
	Aluminium & 237 & 233.4 & -1.5\% \\
	Messing & 120 & 159.7 & +33.1\% \\
	Edelstahl & 15 & 17.7 & +18.0\%\\
\end{tabular}
\end{center}
\caption[Vergleich zwischen Literatur- und gemessenem Wert der Wärmeleitfähigkeit]{Vergleich zwischen Literatur-\footnotemark\ und gemessenem Wert der Wärmeleitfähigkeit}
\label{Literaturwert}
\end{table}
\footnotetext{\url{http://www.chemie.de/lexikon/Wärmeleitfähigkeit.html}}