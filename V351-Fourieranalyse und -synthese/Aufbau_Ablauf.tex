Der Versuch besteht aus zwei Teilen. \\
Zunächst wird für drei verschiedene Spannungsverläufe (Rechteck, Sägezahn und Dreieck) eine \textbf{Fourier-Synthese} durchgeführt. Dazu wird ein Schwingungsgenerator benutzt, der die ersten zehn Komponenten einer Fourier-Reihe generieren kann. Es müssen jeweils die passenden Koeffizienten eingestellt und überprüft werden, dass die Schwingungen alle in Phase sind. \\
Im zweiten Teil wird ein Funktionsgenerator an ein Oszilloskop angeschlossen. Mit Hilfe der \textsc{math}-Funktion des Oszilloskops wird dann eine \textbf{Fourier-Transformation} für verschiedene Spannungen (Rechteck, Sägezahn und Dreieck) durchgeführt. Nun können Ort und Amplitude der Pieks abgelesen werden.