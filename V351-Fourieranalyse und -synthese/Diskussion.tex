Eine Fehleranalyse ist bei diesem Protokoll wenig zielführend. Es wird kein Maß dafür eingeführt, wie weit die durch die \textbf{Fouriersynthese} modellierte Funktion von der Zielfunktion abweicht. Allgemein zeigt sich, dass die hier modellierten Signale (Abbildung \ref{Synthese_Dreieck}, \ref{Synthese_Rechteck}, \ref{Synthese_Saege}) schon mit wenigen Oberschwingungen nahe am gewünschten Ergebnis liegen. Besonders fällt dies bei der Dreieckspannung auf, wo die Amplituden der k-ten Oberschwingung mit~$\sim \frac{1}{k^2}$ abfällt, was an der Stetigkeit der Funktion liegt. \\
Auch das in der Theorie beschriebene Phänomen der Gibb'schen Überschwinger sind in Abbildung \ref{Synthese_Rechteck} und \ref{Synthese_Saege} gut erkennbar. \\
Bei der \textbf{Fourieranalyse} kann eine Ungenauigkeit in der Messung der Koeffizienten mit Hilfe von Formel \eqref{eq:Abweichung} bestimmt werden. Die Steuung ergibt sich aus der Abweichung dividiert durch den Mittelwert der Messung. Bei diesen geringen Fehlern (siehe Tabelle \ref{tab:Fehler}) handelt es sich wahrscheinlich um Ableseungenauigkeiten.
\todo[color = red]{Ist der Mittelwert von dem du hier sprichst $\Delta a$?}
\begin{figure}[h!]
	\centering
		\captionof{table}{Abweichung der Fourierkoeffizienten}
		\begin{tabular}{c|cc}
			 & Mittlerer Fehler & Streuung \\
			 \hline
			Dreieckspannung & 16.44 \si{\milli\volt} & 0.040 \\
			Rechteckspannung & 97.66 \si{\milli\volt} & 0.032 \\
			Sägezahnspannung & 12.55 \si{\milli\volt} & 0.054 \\
		\end{tabular}
		\label{tab:Fehler}
\end{figure}



\todo[color=red, inline]{Sollen wir die Herleitung der Fourierkoeffizienten in einen Anhang kacken? In Annikas Protokoll stehen die in der Diskussion. Ich finde grundsätzlich nicht, dass die in s Protokoll gehören, aber vielleicht will er das ja so?!}
\todo[color=green, inline]{Ich wollte die ganze Theorie ursprünglich sehr mathematisch halten, mit Funktionenraum und Skalarprodukt und so was, habe mich aber dann bewusst dagegen entschieden, weil das a) sehr lang geworden wäre, und b) das für den Versuch gar nicht so relevant ist. Es geht ja erstmal nur darum zu sehen, dass Fourier funktioniert, nicht wo er herkommt. Daher würde ich sagen: Nein, lass uns die Arbeit sparen. Wie man sie berechnet steht in der Theorie und alles andere führt für meinen Geschmack zu weit.}
