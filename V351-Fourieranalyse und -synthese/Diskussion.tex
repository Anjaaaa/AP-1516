Eine Fehleranalyse ist bei diesem Protokoll wenig zielführend. Es wird kein Maß dafür eingeführt, wie weit die durch die \textbf{Fouriersynthese} modellierte Funktion von der Zielfunktion abweicht. Allgemein zeigt sich, dass die hier modellierten Signale (Abbildung \ref{Synthese_Dreieck}, \ref{Synthese_Rechteck}, \ref{Synthese_Saege}) schon mit wenigen Oberschwingungen nahe am gewünschten Ergebnis liegen. Besonders fällt dies bei der Dreieckspannung auf, wo die Amplituden der k-ten Oberschwingung mit~$\sim \frac{1}{k^2}$ abfällt, was an der Stetigkeit der Funktion liegt. \\
Auch das in der Theorie beschriebene Phänomen der Gibb'schen Überschwinger sind in Abbildung \ref{Synthese_Rechteck} und \ref{Synthese_Saege} gut erkennbar. \\
Bei der \textbf{Fourieranalyse} kann mit Hilfe der Formel \eqref{eq:Abweichung} eine Ungenauigkeit in der Messung der Koeffizienten bestimmt werden. Die Streuung ergibt sich aus der Abweichung dividiert durch den Mittelwert der Messung
\begin{equation}
M = \frac{1}{N}  \sum_{i=1}^{N}  a_{\text{gem},i}\ \quad.
\end{equation} 
Bei diesen geringen Fehlern (siehe Tabelle \ref{tab:Fehler}) handelt es sich wahrscheinlich um Ableseungenauigkeiten, wenngleich ein unvermeidbarer systematischer Fehler noch erwähnt werden sollte: Bei der Fourier-Transformation wird über einen unendlich langen Zeitraum integriert. Da das im Experiment nicht durchführbar ist, erhält man nicht die charakteristischen $\delta$-Pieks, sondern etwas breitere Maxima und kleine Nebenmaxima (siehe Abbildungen \ref{Frequenz_Dreieck}, \ref{Frequenz_Rechteck} und \ref{Frequenz_Saege}).
\begin{figure}[h!]
	\centering
		\captionof{table}{Abweichung der Fourierkoeffizienten}
		\begin{tabular}{c|cc}
			 & Mittlerer Fehler & Streuung \\
			 \hline
			Dreieckspannung & 16.44 \si{\milli\volt} & 0.040 \\
			Rechteckspannung & 97.66 \si{\milli\volt} & 0.032 \\
			Sägezahnspannung & 12.55 \si{\milli\volt} & 0.054 \\
		\end{tabular}
		\label{tab:Fehler}
\end{figure}


