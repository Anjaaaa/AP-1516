Im nachfolgend beschriebenen Versuch werden periodische Signale untersucht. \\
Eine Funktion wird (zeitlich) periodisch genannt, wenn für alle $t$ gilt, dass
\[ f(t) = f(t+T) \ , \]
wobei $T$ die Periode bezeichnet. Sie können meist durch Kombinationen von $\sin(\omega t)$ oder $\cos(\omega t)$ bzw. $e^{i\omega t}$ ausgedrückt werden. Für die Frequenz gilt hierbei $\omega = \frac{2\pi}{T} $. \\
\ \\
Das Fouriersche Theorem besagt, dass eine Reihe der Form
\begin{align}
	C + \sum_{n = 1}^{\infty} A_n\cos(n\omega t)+B_n\sin(n\omega t)
\end{align}
wenn sie gleichmäßig konvergent ist immer eine periodische, abschnittsweise stetige Funktion darstellt. Eine solche Reiche wird auch Fourier-Reihe genannt. Die Koeffizienten $A_n$, $B_n$ und $C$ werden berechnet durch
\begin{align}
	A_n &= \frac{2}{T}\int_0^T f(t)\cos(n\omega t)\ dt \\
	B_n &= \frac{2}{T}\int_0^T f(t)\sin(n\omega t)\ dt \\
	C &= A_0 = \frac{2}{T}\int_0^Tf(t)\ dt \ .
\end{align}
In Tabelle \ref{fig:Fourier_Beispiele} sind beispielhaft die Fourier-Reihen, der später betrachteten Signale dargestellt.
\begin{table}[h!]
\begin{center}
\begin{tabular}{c|c|c}
	Form & Funktion & Fourier-Reihe \\
\hline
	Rechteck & $ f_{\text{Rechteck}}(t) =
	\begin{cases}
	u, & 0\leq t < \frac{T}{2} \\
	-u, & -\frac{T}{2}\leq t < 0
	\end{cases} $ & $ f_{\text{R}}(t) = 	\sum_{n=1}^\infty \frac{4u}{(2n-1)\pi}\sin((2n-1)\omega t) $ \\
	Sägezahn & $ f_{\text{Sägezahn}}(t) = \frac{2u}{T}t - u \qquad 0\leq t < T $ & $ f_{\text{S}}(t) = \sum_{n=1}^\infty -\frac{Tu}{n\pi}\sin(n\omega t) $ \\
	Dreieck & $ f_{\text{Dreieck}}(t) = 	\begin{cases}
	u - \frac{-4u}{T}t, & 0\leq t < \frac{T}{2} \\
     u + \frac{-4u}{T}t, & -\frac{T}{2}\leq t < 0
	\end{cases}  $ & $ f_{\text{D}}(t) = \sum_{n=1}^\infty \frac{8u}{((2n-1)\pi)^2}\cos((2n-1)\omega t) $
\end{tabular}
\end{center}
\captionof{table}{Fourier-Reihen verschiedener periodischer, nicht-differenzierbarer Funktionen}
\label{fig:Fourier_Beispiele}
\end{table}
Ist $f$ an einer Stelle $t_i$ unstetig, dann weicht die Fourier-Reihe an dieser Stelle von der Funktion ab. Diese Abweichung ist endlich und wird mit wachsendem $n$ nicht kleiner. Dieses \glqq Überschwingen\grqq\ wird Gibb'sches Phänomen genannt. \\
\ \\
\ \\
Manchmal ist es hilfreich oder interessant, das Frequenz-Spektrum einer Funktion zu betrachten. Der Übergang
\[ f(t)\rightarrow g(\omega) \]
wird von der Fourier-Transformation
\begin{align}
	g(\omega) = \frac{1}{\sqrt{2\pi}}\int_{-\infty}^{\infty}f(t)e^{i\omega t}\ dt
\end{align}
geleistet. \\
Ist $f$ periodisch besteht $g$ aus $\delta$-Distributionen an den Stellen $k\omega$. Die Höhe der Pieks entspricht dann den Koeffizienten des Cosinus- bzw. Sinus-Terms mit der Frequenz $k\omega$ der Fourierreihe. Nicht-periodische Funktionen zeigen hier ein kontinuierliches Spektrum an Frequenzen.