\subsection{Alpha-Zerfall}
Beim Alpha-Zerfall zerfällt ein instabildes Atom des Elements $X$ in ein Atom eines anderen Elements $Y$ und einen zweifach positiv geladenen Helium Kern, das Alpha-Teilchen. Die charakteristische Größe bei radioaktiven Zerfällen ist die Halbwertszeit $\tau$, das ist die Zeit, nach der die Hälfte der zu Beginn vorhandenen Atome zerfallen ist. Für die Anzahl $N$ der zu einem Zeitpunkt $t$ noch vorhandenen Atome gilt das Zerfallsgesetz
\begin{align}
	N(t) = N_02^{-\frac{t}{\tau}} \ .
\end{align}


\subsection{Energieverteilung}
Der Alpha-Zerfall kann mit der Reaktionsgleichung
\begin{align}
        X \rightarrow Y + \alpha
\end{align}
beschrieben werden. Es gelten Impuls- und Energieerhaltung, sodass
\begin{align}
        p_Y = - p_\alpha
\end{align}
und
\begin{align}
        E_X &= E_Y + E_\alpha \\
        m_Xc^2 &= \sqrt{m_Y^2c^4 + p_Y^2c^2} + \sqrt{m_\alpha^2c^4 + p_\alpha^2c^2} \\
        &= \sqrt{m_Y^2c^4 + p_\alpha^2c^2} + \sqrt{m_\alpha^2c^4 + p_\alpha^2c^2}
\end{align}
ist. Aufgelöst nach $p_\alpha$ gilt dann
\begin{align}
        p_\alpha = c\sqrt{\frac{m_X^2}{4} - \frac{m_Y^2}{2} - \frac{m_\alpha^2}{2}+\frac{\left(m_Y^2 - m_\alpha^2\right)^2}{4m_X^2}}
\end{align}
\todo{Mit Zustand ist hier gemeint, wie viele Neutronen das Atom im Kern hat oder? Denn es kommt ja auf die Masse an.}
Alle Alpha-Teilchen haben demnach dieselbe Geschwindigkeit bzw. Energie. Das Energiespektrum ist diskret mit nur einem Peak. \\
Es kann allerdings passieren, dass $Y$ nach der Reaktion nicht im Grundzustand, sondern in einem seiner angeregten Zustände $Y',Y'',...$ vorliegt. Dann ändert sich in der Reaktionsgleichung und obiger Rechnung $Y\rightarrow Y',Y'',...$, sodass je nachdem, in welchem Zustand sich das Atom befindet unterschiedlich schnelle Alpha-Teilchen emittiert werden. Das Spektrum ist nach wie vor diskret, hat aber mehrere Peaks.


\subsection{Reichweite}
Bei der Bewegung durch Gas verliert ein Alpha-Teilchen Energie durch Ionisierung, Anregung und Dissoziation (Spaltung) von anderen Teilchen oder durch elastisches Stoßen. Dieser Energieverlust hängt von der Dichte des Gases und der Energie des Teilchens. \todo[color=red]{In der Anleitung steht: Je kleiner die Geschwindigkeit, desto größer die Wahrscheinlichkeit der Wechselwirkung mit Molekülen. Warum?} 
\todo[color=green, inline]{Björn hatte dazu gesagt, dass man sich vorstellen kann, dass langsamere Teilchen länger in dem Bereich sind, in dem WW auftreten und dadurch stärker betroffen sind.}
Ist die Energie groß genug kann der Energieverlust mit der Bethe-Bloch-Gleichung
\begin{align}
	& -\frac{dE_\alpha}{dx} = \frac{z^2e^4}{4\pi\epsilon_0m_e}\frac{nZ}{v^2}\ln{\left(\frac{2m_ev^2}{I}\right)} \\
        & z: \text{ Ladung der Strahlung}\notag \\
	& v: \text{ Geschwindigkeit der Strahlung}\notag \\
	& n: \text{ Teilchendichte im Gas}\notag \\
	& Z: \text{ Ordnungszahl der Teilchen im Gas}\notag \\
	& I: \text{ Ionisierungsenergie des Gases}\notag
\end{align}\todo[color=red]{Diese Gleichung wird in der Auswertung gar nicht verwendet oder?}
\todo[color=green]{Ne, irgendwie nicht}
beschrieben werden. \\
Die Reichweite von Alpha-Strahlung ist definiert als der Abstand $R$ zur Strahlungsquelle, bei dem nur noch die Hälfte der ausgesendeten Strahlung ankommt. Für Strahlung mit hoher Energie, d.h. falls die Bethe-Bloch-Gleichung gilt, ist die Reichweite in einem Gas
\begin{align}
	R = -\int_0^{E_\alpha}\frac{dE_\alpha}{dE_\alpha/dx} \ .
\end{align}
Ist die Energie allerdings kleiner werden empirisch gefundene Kurven verwendet. Später wird
\begin{align}
	R = 3.1\cdot E_\alpha^\frac{3}{2}10^6
\end{align}
benutzt ($E_\alpha$ in \si{\electronvolt}). Dieser Zusammenhang ist gültig für $E_\alpha \leq \SI{2.5}{\mega\electronvolt}$ und beliebige Gase. Die Reichweite der Strahlung ist aber proportional zum Druck $p$, sodass gemessene Längen $x_0$ durch die effektive Länge
\begin{align}
	& x = x_0\frac{p}{p_0} \\
	& p_0: \text{ Normaldruck } \SI{1013}{\milli\bar}
\end{align}
ersetzt werden müssen.

