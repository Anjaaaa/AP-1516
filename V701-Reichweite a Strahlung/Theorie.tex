Alpha-Strahlung ist eine Form von radioaktiver Strahlung, die aus einem zweifach positiv geladenen Helium-Kern besteht. Im Versuch wird als Strahlungsquelle Americium verwendet
\begin{align}
	\text{\ce{^241_95Am}}\rightarrow\text{\ce{^237_93Np}}+\text{\ce{^4_2He^2+}} \ .
\end{align}
Ihre Reichweite liegt in Luft im \si{\centi\meter}-Bereich. Bei der Bewegung durch Gas verliert die Alpha-Strahlung Energie durch Ionisierung, Anregung und Dissoziation (Spaltung) von anderen Teilchen oder durch elastisches Stoßen. Dieser Energieverlust hängt von der Dichte des Gases und der Energie der Strahlung ab. \todo[color=red]{In der Anleitung steht: Je kleiner die Geschwindigkeit, desto größer die Wahrscheinlichkeit der Wechselwirkung mit Molekülen. Warum?} Ist die Energie groß genug kann der Energieverlust mit der Bethe-Bloch-Gleichung
\begin{align}
	& -\frac{dE_\alpha}{dx} = \frac{z^2e^4}{4\pi\epsilon_0m_e}\frac{nZ}{v^2}\ln{\left(\frac{2m_ev^2}{I}\right)} \\
        & z: \text{ Ladung der Strahlung}\notag \\
	& v: \text{ Geschwindigkeit der Strahlung}\notag \\
	& n: \text{ Teilchendichte im Gas}\notag \\
	& Z: \text{ Ordnungszahl der Teilchen im Gas}\notag \\
	& I: \text{ Ionisierungsenergie des Gases}\notag
\end{align}\todo[color=red]{Diese Gleichung wird in der Auswertung gar nicht verwendet oder?}
beschrieben werden. \\
Die Reichweite von Alpha-Strahlung ist definiert als der Abstand $R$ zur Strahlungsquelle, bei dem nur noch die Hälfte der ausgesendeten Strahlung ankommt. Für Strahlung mit hoher Energie, d.h. falls die Bethe-Bloch-Gleichung gilt, ist die Reichweite in einem Gas
\begin{align}
	R = -\int_0^{E_\alpha}\frac{dE_\alpha}{dE_\alpha/dx} \ .
\end{align}
Ist die Energie allerdings kleiner werden empirisch gefundene Kurven verwendet. Später wird
\begin{align}
	R = 3.1\cdot E_\alpha^\frac{3}{2}10^6
\end{align}
benutzt ($E_\alpha$ in \si{\electronvolt}). Dieser Zusammenhang ist gültig für $E_\alpha \leq \SI{2.5}{\mega\electronvolt}$ und beliebige Gase. Die Reichweite der Strahlung ist aber proportional zum Druck $p$, sodass gemessene Längen $x_0$ durch die effektive Länge
\begin{align}
	& x = x_0\frac{p}{p_0} \\
	& p_0: \text{ Normaldruck } \SI{1013}{\milli\bar}
\end{align}
ersetzt werden müssen. \\
Hier mache ich evtl. noch was zu der statistischen Verteilung.
