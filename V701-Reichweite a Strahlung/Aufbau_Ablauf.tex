Der Versuchsaufbau besteht im wesentlichen aus einem evakuierbaren Glaskolben, in dem sich die Strahlungsquelle und ein Detektor mit variablem Abstand befinden. Der Detektor ist ein Halbleiter-Sperrschichtzähler, ein darauf auftreffendes Ion erzeugt Elektronen-Loch-Paare, die einen Strompuls auslösen, der dann mit einem Vielkanalanalysator je nach Stärke detektiert wird. Die Stärke des Strompulses ist proportional zur Energie der Alpha-Teilchen. Die so aufgenommen Daten werden an einem Computer visualisiert. Zu sehen ist dabei ein Histogramm mit den verschiedenen Channels auf der x-Achse und der jeweiligen Anzahl der Pulse. \\
Vor Messbeginn sollte der Diskriminator richtig eingestellt werden, damit kann bestimmt werden ab welcher Höhe die Pulse überhaupt gezählt werden. Ist er zu hoch eingestellt, werden relevante, aber niedrige Pulse nicht mehr gezählt. Ist er zu niedrig eingestellt, werden unter Umständen auch kleine Störungen zum Beispiel im Detektor als Pulse gezählt, was das Ergebnis verfälschen würde.
\subsection*{Messung}
Zunächst werden Reichweite und Energie der Alpha-Strahlung bestimmt. Dafür wird zwischen Quelle und Detektor ein Abstand von \SI{2.1}{\centi\meter} eingestellt. Für verschiedene Drücke im Glaskolben werden dann für jeweils 2 Minuten die Pulse gezählt. Notiert wird der Channel (als Maß für die Energie) mit den meisten Pulsen und die Zahl der Gesamtpulse. Diese Messung wird für einen zweiten Abstand \SI{3.0}{\centi\meter} wiederholt. \\
Mit der zweiten Messung soll die statistische Verteilung der Energien bestimmt werden. Dazu wird der Abstand auf \SI{3}{\centi\meter} gelassen und 100-mal die Gesamtzahl der Pulse innerhalb von \SI{10}{\second} aufgenommen.
