Warum zu Hölle ist ist unsere Poissonverteilung so schlecht???!!!!!

      \begin{figure}[h!]
   	  	\centering
   	  	\captionof{table}{Energie der $\alpha$-Strahlung}
   	  	\begin{tabular}{c|c|c}
   	  		 & Empirisch bestimmt & Bestimmung durch Regression \\
   	  		\hline
   	  	Abstand \SI{2.1}{\centi\meter} &  \input{build/E1_emp.txt}  &  \input{build/E1_reg.txt} \\
   	  	Abstand \SI{3}{\centi\meter} &  \input{build/E2_emp.txt} &  \input{build/E2_reg.txt}
   	  	\end{tabular}
   	  	\label{tab:energie_vergleich}
   	  \end{figure}

Ich glaube, dass die Poisson-Verteilung nicht schlecht ist. Sondern, dass der Zerfall halt nunmal durch eine Gauß-Kurve beschrieben wird, sodass es keinen passenden Poisson-Fall gibt. \\
Auf Wikipedia steht allerdings, dass ab etwa lambda = 30 beide sehr ähnlich sind, wobei $\mu = \lambda$ und $\lambda=\sigma^2$. Von daher glaube ich einfach, dass das zu viele Pulse sind, weil die ja das $\lambda,\mu$ bestimmen.
