Die Zeiten in Tabelle \ref{fig:Zeiten} und die berechnete Elementarladung in \eqref{eq:Elementarladung} sind Mittelwerte. Sie werden mit
\begin{align}
	\bar{x} =  \frac{1}{N-1} \sum_{i=1}^\text{N} x_i
\end{align}
berechnet. Ihre Fehler sind
\begin{align}
	\Delta x = \sqrt{\frac{\sum_{i=1}^\text{N} (x_i - \bar{x})^2}{N(N-1)}} \ .
\end{align}
Die Fehler der Geschwindigkeiten in Tabelle \ref{fig:Vel} und der Radien und Ladungen in den Tabellen \ref{fig:RQ_unkorr} und \ref{fig:RQ_korr} werden mit der Gaußschen Fehlerfortpflanzung berechnet. Für die Geschwindigkeiten ist das
\begin{align}
	\Delta v &= \sqrt{\left(\frac{\partial v}{\partial s}\right)^2(\Delta s)^2 + \left(\frac{\partial v}{\partial t}\right)^2(\Delta t)^2} \\
	&= \frac{s}{t^2}\Delta t
\end{align}
und für die Radien
\begin{align}
	\Delta r &= \sqrt{\left(\frac{\partial r}{\partial v_\text{ab}}\right)^2(\Delta v_\text{ab})^2 + \left(\frac{\partial r}{\partial v_\text{auf}}\right)^2(\Delta v_\text{auf})^2} \\
	&= \frac{3}{4}\frac{\sqrt{\frac{\eta_\text{L}}{g\cdot\rho_\text{Oel}}}}{\sqrt{(v_\text{ab}-v_\text{auf})}} \sqrt{(\Delta v_\text{ab})^2 + (\Delta v_\text{auf})^2} \ .
\end{align}
Der Fehler der nicht korrigierten Ladungen ist
\begin{align}
	\Delta q &= \sqrt{\left(\frac{\partial q}{\partial v_\text{ab}}\right)^2(\Delta v_\text{ab})^2 + \left(\frac{\partial q}{\partial v_\text{auf}}\right)^2(\Delta v_\text{auf})^2} \\
	&= \frac{9}{2}\frac{d\pi\eta_\text{L}}{U}\sqrt{\frac{\eta_\text{L}}{g\cdot\rho_\text{Oel}}(v_\text{ab}-v_\text{auf})}
	\sqrt{\left(1+\frac{v_\text{ab}+v_\text{auf}}{2(v_\text{ab}-v_\text{auf}) }\right)^2(\Delta v_\text{ab})^2
	+\left(1-\frac{v_\text{ab}+v_\text{auf}}{2(v_\text{ab}-v_\text{auf})}\right)^2(\Delta v_\text{auf})^2}
\end{align}
und der Fehler der korrigierten Ladungen
\begin{align}
	\Delta q_\text{korr} &= \sqrt{\left(\frac{\partial q_\text{korr}}{\partial q}\right)^2(\Delta q)^2 + \left(\frac{\partial q_\text{korr}}{\partial r}\right)^2(\Delta r)^2} \\
	&= \sqrt{\left(1+\frac{B}{pr}\right)^3(\Delta q)^2 + \left(1+\frac{B}{pr}\right) \left(\frac{3}{2}\frac{qB}{pr^2}\right)^2(\Delta r)^2} \ .
\end{align}