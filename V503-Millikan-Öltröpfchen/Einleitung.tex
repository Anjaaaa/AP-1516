Der Millikan-Öltröpfchenversuch erlaubte es in Jahr 1910 erstmalig, die Elementarladung $e_0$ mit großer Genauigkeit zu bestimmen. Dazu wird die Fallgeschwindigkeit von Öltröpfchen -- ungehindert und mit angelegtem elektrischen Feld -  bestimmt. Aus der Abweichung ergibt sich die Landung der Öltröpfchen. Sie ist eine ganzzahlige Vielfache der Elementarladung.