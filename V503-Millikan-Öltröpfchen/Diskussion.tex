Der berechnete Wert für die Elementarladung weicht nur \SI{+1.11}{\%} vom Literaturwert \cite[Tabellen Anhang Tab. A 1.4]{Walcher} ab. Das ist ein sehr zufriedenstellendes Ergebnis. Allerdings muss zugegeben werden, dass die verwendete Methode eher bestätigenden Charakter hat, da auch die Wahl $q_1 = 4e_0$ oder $q_1 = 5e_0$ graphisch schöne Lösungen sind. Um die Elementarladung tatsächlich bestimmen zu können, müssten mehr Elektronen und diese noch häufiger gemessen werden. Dann wären vermutlich eindeutige Ansammlungen von Datenpunkten bei bestimmten Werten zu sehen. \\
Die systematischen Fehler bei diesem Versuch beschränken sich auf den menschlichen Faktor bei der Zeitmessung, welcher allerdings durch wiederholte Messungen eingeschränkt wird.