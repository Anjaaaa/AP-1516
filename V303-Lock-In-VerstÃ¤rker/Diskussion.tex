\todo[noline, color=green]{Das mit dem falsch eingestellten Gain finde ich eine tolle Idee!}
\textbf{Zu \ref{nach_Tiefpass}:} \\
Beide Fits ergeben den zu erwartenden Sinus \glqq ohne\grqq\ Phasenverschiebung. Nur die Amplitude ist jeweils um einen Faktor drei zu klein. Dass die Amplitude nicht passt fiel auch schon während der Messung auf. Die Vermutung liegt nahe, dass diese Abweichung durch das Alter des Geräts verursacht wurde. Es wäre möglich, dass das Gerät enorme Wärmeverluste hat. Auch könnte es sein, dass die letzte Eichung schon länger her ist, sodass ein Gain von 20 eigentlich eher einem Gain von 19 entspricht, was die Werte wieder näher an den Theoriewert bringen würde. \\


\textbf{Zu \ref{Rauschunterdruckung}:} \\
Die berechnete Funktion $U(r)$ zeigt, dass das Signal der LED noch bei einem Abstand von \SI{0.5}{\meter} eindeutig gemessen werden kann, auch mit dem Hintergrundrauschen, das durch die Raumbeleuchtung verursacht wurde.

\todo[inline, color=red]{Zu \ref{Rauschunterdruckung} wollte ich eigentlich schreiben, dass nach x metern halt nur noch zum Beispiel 0.001 von der Ausgangsspannung messbar war und man deshalb bis dahin ein ganz gutes Ergebnis bekommt. Dann ist mir aufgefallen, dass wir keine verwertbare Ausgangsspannung haben. Wie geht das? Oder übersehe ich hier etwas?}
\todo[inline, color=yellow]{Ich kann mich auch nicht daran erinnern, dass die Intensität bzw. die Spannung der Lampe angegeben war. Als (nicht ganz tolle) Alternative könnte man sich auf den ersten Messwert beziehen.}

\clearpage
