\[U_\text{out, theoretisch} = \frac{2}{\pi}\cdot\SI{10}{\milli\volt}\cdot \cos\phi\]

\begin{align}
	U_\text{out}(t) &= \frac{1}{T}\int_{t_0}^{t_0+T}\underbrace{U_\text{ref}(t')\cdot U_\text{in}(t')}_{Mischer}\ dt' \\
	&= \frac{1}{T}\int_{t_0}^{t_0+T}\underbrace{U_{0,\text{ref}}}_{theor.\ = 1}\sin(wt'+\Delta\varphi)\cdot U_{0,\text{in}}\sin(wt')\ dt' \\
	&= \frac{1}{T}U_{0,\text{in}}\left[\frac{t}{2}\cos(t)-\frac{\sin(2wt+\Delta\varphi)}{4w}\right]_{t_0}^{t_0+T} \\
	&= \underbrace{\frac{U_{0,\text{in}}}{4Tw}\left[-\sin(2wt_0+2wT+\varphi)+\sin(2t_0w+\varphi)\right]}_{\overset{T\rightarrow\infty}{\longrightarrow} 0}+\frac{U_{0,\text{in}}}{2}\cos(\varphi) \\
	&= \frac{U_{0,\text{in}}}{2}\cos(\varphi)
\end{align}
Das $\frac{4}{\pi}$ kommt dann meiner Meinung nach durch die Rechteckspannung, aber das Integral über die Reihe konnte Wolframalpha für mich nicht lösen. \\
Claras Rechnung ohne Integration:
Ich benötige das Additionstheorem
\begin{equation}
	\sin(x+y) \cdot \sin(x-y)  = \sin^2 - \sin^2 y \ .
\end{equation}
Was sich durch $\sin^2(x) = \frac{1}{2}(1-\cos(2\cdot x)$ umschreiben lässt
\begin{equation}
\sin(x+y) \cdot \sin(x-y)  = \frac{1}{2} \left(\cos(2x)-\cos(2y) \right) \ .
\end{equation}
Im Mischer werden zwei Signale multipliziert (hier nehme mich ein Sinussignal an)
\begin{equation}
U_{out}(t)=U_0 \cdot \sin(\omega t + \phi) \cdot \sin(\omega t)
\end{equation}
Durch die Substitutionen $x= \omega t + \frac{1}{2} \phi$ und $y=\frac{1}{2} \phi$ und Anwendung des Additionstheorems erhält man:
\begin{equation}
U_{out}=\frac{U_0}{2}\cdot(\cos{\phi}-\cos{(2\omega t + \phi)}) \ .
\end{equation}
Das $\frac{U_0}{2}\cos \phi$ meinte ich mit der Verschiebung entlang der y-Achse. Die Schwingung $-\frac{U_0}{2}\cos{(2\omega t + \phi)}$ ist nicht mehr um $y=0$. Dieter hat es mir nun so erklärt, dass der Tiefpass alle Schwingungsanteile herausfiltert und nur die Verschiebung entlang der y-Achse übrig bleibt. Der Tiefpass ist an dieser Stelle also wirklich ein normaler Tiefpass. Was der Kevin von einem Integrierglied meinte, war - behaupte ich - nicht korrekt.
	