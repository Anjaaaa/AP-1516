\todo[inline, color=green]{Ich meine bei input werden Umlaute nicht erkannt. Deshalb geht Geräte.tex nicht. Das hatte ich auch schon mehrfach...}
\todo[inline]{Super! Ich hatte gar nicht mehr daran gedacht so ein Kapitel einzufügen. Das war nach wie vor eine tolle Idee :)}
\subsection{Hoch- und Tiefpassfilter}
Zur technischen Umsetzung eines einfachen Tiefpassfilters benötigt man einen Kondensator oder eine Spule in Verbindung mit einem ohmschen Widerstand $R$.
Die Impedanzen des Kondensators und der Spule sind frequenzabhängig
\begin{align}
&X_C = \frac{1}{i \omega C} \\
&X_L = i \omega L
\end{align}
mit der Kapazität $C$ und der Induktivität $L$.
Das Verhältnis von Eingangs- und Ausgangsspannung $\frac{U_2}{U_1}$ kann mit Hilfe der 2. Kirchhoffschen Regel (Maschenregel) berechnet werden.


\begin{figure}[h!]
	\centering
	\begin{minipage}{0.4\textwidth}
		\begin{circuitikz}
	\draw (0,0) to [open] (0,3)
	to [C=$C$, o-] (2,3)
	to [short, -o] (4,3)
	to[open] (4,0)
	to[short, o-o] (0,0)
	(2,0) to [R=$R$] (2,3)
	(0, 1.5 )node[]{$U_1$}
	(4, 1.5)node[]{$U_2$};
\end{circuitikz}
	
		\caption{RC-Hochpass}
	\end{minipage}
	\hspace{1cm}
	\begin{minipage}{0.4\textwidth}
		\begin{circuitikz}
	\draw (0,0) to [open] (0,3)
	to [R=$R$, o-] (2,3)
	to [short, -o] (4,3)
	to[open] (4,0)
	to[short, o-o] (0,0)
	(2,0) to [C=$C$] (2,3)
	(0, 1.5 )node[]{$U_1$}
	(4, 1.5)node[]{$U_2$};
\end{circuitikz}
		\caption{RC-Tiefpass}
		\label{RC-Tiefpass}
	\end{minipage}
\end{figure}		


Für den RC-Hochpass ergibt sich ein Verhältnis von
\begin{align}
&\frac{U_1}{U_2} = \frac{R}{R + X_C} = \frac{R}{R+\frac{1}{i \omega C}} \\
& \bigr| \frac{U_1}{U_2} \bigl| = \frac{1}{\sqrt{1 + \left(\frac{1}{\omega C}\right)^2}}\ .
\end{align}

Und für den RC-Tiefpass
\begin{align}
&\frac{U_1}{U_2} = \frac{X_C}{R + X_C} = \frac{\frac{1}{i \omega C}}{R+\frac{1}{i \omega C}} \\
& \bigr| \frac{U_1}{U_2} \bigl| = \frac{1}{\sqrt{1 + (\omega R C)^2}} \ .
\end{align}


\begin{figure}[h!]
	\centering
	\begin{minipage}{0.4\textwidth}
		\begin{circuitikz}
	\draw (0,0) to [open] (0,3)
	to [C=$C$, o-] (2,3)
	to [short, -o] (4,3)
	to[open] (4,0)
	to[short, o-o] (0,0)
	(2,0) to [L=$L$] (2,3)
	(0, 1.5 )node[]{$U_1$}
	(4, 1.5)node[]{$U_2$};
\end{circuitikz}
		\caption{LC-Hochpass}
	\end{minipage}
	\hspace{1cm}
	\begin{minipage}{0.4\textwidth}
		\begin{circuitikz}
	\draw (0,0) to [open] (0,3)
	to [L=$L$, o-] (2,3)
	to [short, -o] (4,3)
	to[open] (4,0)
	to[short, o-o] (0,0)
	(2,0) to [C=$C$] (2,3)
	(0, 1.5 )node[]{$U_1$}
	(4, 1.5)node[]{$U_2$};
\end{circuitikz}
		\caption{LC-Tiefpass}
	\end{minipage}
\end{figure}		



Bei einem LC-Hochpass ist das Verhältnis von Ausgangs- zu Eingangsspannung
\begin{align}
&\frac{U_1}{U_2} = \frac{X_L}{X_L + X_C} = \frac{i \omega L}{i \omega L+\frac{1}{i \omega C}} \\
& \bigr| \frac{U_1}{U_2} \bigl| = \frac{1}{|1-\frac{1}{ \omega^2 L C}|}\ .
\end{align}
Und für eine LC-Tiefpass:
\begin{align}
&\frac{U_1}{U_2} = \frac{X_C}{X_L + X_C} = \frac{\frac{1}{i \omega C}}{i \omega L+\frac{1}{i \omega C}} \\
& \bigr| \frac{U_1}{U_2} \bigl| = \frac{1}{|1- \omega^2 L C|}\ .
\end{align}

Im Normalfall sind Filter komplexer und unterscheiden sich vor allem durch ihre Güte d.h. wie stark die Flanke des Spannungsverhältnisses steigt. Die Güte wird auch als Trennschärfe bezeichnet.

\subsection{RC-Tiefpass als Integrierglied}
Der RC-Tiefpass, wie er in Abbildung \ref{RC-Tiefpass} dargestellt ist, kann als sogenanntes Integrierglied eingesetzt werden. Das bedeutet, dass das Ausgangssignal das Integral zum Eingangssignal bildet. Dazu muss folgendes gelten
\begin{equation}
R \gg \frac{1}{\omega C} \quad \Leftrightarrow \quad  \omega R C \gg 1 \ .
\end{equation}
Die Impedanz verschiebt das Eingangssignal um deine bestimmte Phase $\delta$
\begin{equation}
\tan \delta = \frac{\Im \left(\frac{U_2}{U_1}\right)}{\Re \left(\frac{U_2}{U_1}\right)} = - \omega R C \ .
\end{equation}
Ist das Eingangssignal nun eine Sinusfunktion
\begin{equation}
U_1 = U_0 \sin(\omega t)
\end{equation}
so wir das Ausgangssignal ein
\begin{equation}
U_2 = \frac{U_0}{\sqrt{1 + (\omega R C)^2}} \sin(\omega t + \delta) = - \frac{\cos{(\omega t)}}{\omega R C} \ .
\end{equation}
Ebenso können Rechtecksignale in der Fourierdarstellung als Summe von Sinus- und Cosinusfunktionen gesehen und integriert werden.

Wo kommt des Vorfaktor her?!

\subsection{Phasenwandler}
Geht aus dem anderen hervor...


		
		