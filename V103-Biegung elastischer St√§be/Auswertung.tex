subsection{Bestimmung der Flächenträgheitsmomente}
Das Flächenträgheitsmoment ist eine Größe, die im weiteren Verlauf wichtig ist, um das Elastizitätsmodul der Stäbe zu ermitteln.
Es hängt von dem Querschnitt des Stabes, genauer der Abstände $y$ der Flächenelemente d$q$ zur neutralen Faser ab
\begin{equation}
I = \int_{Q} y ^2 \, \text{d}q \quad .
\end{equation}
Für den eckigen Stab benötigt man eine Formal für quadratische Quarschnitte. Die Kantenlänge sei $h$. Für $I_\text{E}$ gilt:
\begin{equation}
I_\text{E} = \int_{-\frac{h}{2}}^{\frac{h}{2}} \int_{-\frac{h}{2}}^{\frac{h}{2}} y^2\,  \text{d}x \text{d}y = \frac{1}{12} \cdot h^4 \quad .
\end{equation}
Um das Flächenträgheitsmoment für runde Querschnitte mit Radius $r$ zu berechnen, bietet sich die Verwendung von Polarkoordinaten an. Der Abstand zur $y$-Achse ist dann $r^2 \cdot \sin^2(x)$.  Mit der Jakobideterminante $r$ ist $I_R$
\begin{equation}
I_\text{R} = \int_{0}^{r}  \int_{0}^{2\pi} r^2 \cdot \sin^2(x) \cdot r  \, \text{d}\phi \text{d}r = \frac{1}{12}\cdot \pi \cdot r^4 \quad .
\end{equation}
Sowohl $h$ in wie auch der Durchmesser $2 \cdot r$ wurden sehr genau mit einer Schieblehre gemessen. Zu dem Fehler des Mittelwertes kommt allerdings eine Ableseungenauigkeit von \SI{0.05}{\milli\metre}  hinzu.

\begin{center}
	\captionof{table}{Breite $h$ des eckigen Stabes und Durchmesser $2 \cdot r$ des Runden}
\begin{tabular}{c|c}
	Breite $h$ in \SI{}{\milli\metre}& Durchmesser $2 \cdot r$ in \SI{}{\milli\metre}	\\
	\hline
	10.00 & 10.00 \\
	10.00  & 10.00 \\
	10.00  & 10.00 \\
	10.00  & 9.90 \\
	10.00  & 9.90 \\
	10.00  & 9.95 \\
	10.00  & 10.00 \\
	10.00  & 9.90 \\
	10.00  & 9.90 \\
	10.00 & 9.95 \\
\end{tabular}	
\end{center}

Für den eckigen Stab folgt aus dem Mittelwert der Breite
\begin{align}
  h = \SI{0.01000 \pm 0.00005}{\metre}
\end{align}
ein Flächenträgheitsmoment von
\begin{align}
I_\text{E}=\SI{3.88(17)e-10}{\metre\raiseto{4}} \quad .
\end{align}

Für den runden Stab folgt aus einem Radius von duschnittlich
\begin{equation}
  r = \SI{0.004975 \pm 0.000032}{\metre}
\end{equation}
das Flächenträgheitsmoment
\begin{equation}
  I_\text{R} = \SI{4.81(12)e-10}{\metre\raiseto{4}} \quad .
\end{equation}









\subsection{Bestimmung der Dichte}
Zur Bestimmung der Dichte wurden beide Stäbe , der runde und der eckige, vermessen und gewogen. Das Gewicht und die Länge sind keine fehlerbehafteten Größen. Der Durchmesser hingegen wurde mit einer Schieblehre mehrfach gemessen. Ein Ablesefehler von \SI{0.05}{\milli\metre}  kommt zu dem Fehler des Mittelwertes hinzu.
