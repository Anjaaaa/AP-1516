Die Kenntnis der Dichte lässt zu, die Materialien der Stäbe zu bestimmen.
Bei dem runden, golden glänzenden Stab, handelt es sich wahrscheinlich um einen Messingstab. Der eckige Stab, der matt silbern wirkt, scheint aus Aluminuim zu sein. Die Dichten wurden mit großer Genauigkeit berechnet und stimmen mit den Literaturwerten der Dicht von Messing und Alumium stark überein (siehe Tabelle \ref{tab:vergleich_dichte}).

\begin{center}
	\captionof{table}{Vergleich mit Literaturwert der Dichten}
\begin{longtable}{c|c|c|c}
	Art & Literatur\footnote{W. Walcher: "Praktikum der Physik", Tabellen-Anhang} & Berechnet & Abweichung \\
	\hline
	Rund -- Messing & \SI{8500}{\kilo\gram\per\cubic\metre} & \SI{8455.90(10902)}{\kilo\gram\per\cubic\metre}  & 0\% \\
	Eckig -- Aluminium & \SI{2700}{\kilo\gram\per\cubic\metre} & \SI{2785.00(2785)}{\kilo\gram\per\cubic\metre}  & +3\% \\
\end{longtable}
\label{tab:vergleich_dichte}
\end{center}

Die Literaturwerte für die Elastizitätsmoduln von Messing und Aluminium können nun mit den von uns gemessenen Werten verglichen werden (siehe Tabelle \ref{tab:vergleich_elastizitatsmodul}). Die Abweichungen von den Literaturwerten sind sehr gering.
Die Messungen scheinen recht genau und aussagekräftig zu sein. Auch haben alle drei Wege, den Elastizitätsmodul des eckigen Stabes zu bestimmen, zu sehr ähnlichen Ergebnissen geführt. \\
Ein systematischer Fehler könnte zu kurze Wartezeiten vor den Messungen sein. Sowohl Durchbiegung wie auch Entspannung der Stäbe erfolgen zeitverzögert.

\begin{center}
\captionof{table}{Vergleich mit Literaturwertwert der Elastizitätsmoduln}
\begin{longtable}{c | c | c | c}
	Art & Literatur\footnote{W. Walcher: "Praktikum der Physik", Tabellen-Anhang} & Berechnet & Abweichung \\
	\hline
	Rund & \SI{10e+10}{\newton\per\metre\squared} & \SI{5.29(4)e+10}{\newton\per\metre\squared} & -52.9\% \\
	Eckig & \SI{7.2e+10}{\newton\per\metre\squared} &\SI{7.37(5)e+10}{\newton\per\metre\squared} & +2.4\%\\
	Beidseitig links & \SI{7.2e+10}{\newton\per\metre\squared} & \SI{18.0(6)e+10}{\newton\per\metre\squared} ?? & ?? \% \\
	Beidseitig rechts & \SI{7.2e+10}{\newton\per\metre\squared} & \SI{13.8(6)e+10}{\newton\per\metre\squared} ?? & ??\% \\
\end{longtable}
\label{tab:vergleich_elastizitatsmodul}
\end{center}
