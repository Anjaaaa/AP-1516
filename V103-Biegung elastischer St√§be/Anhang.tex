In der Versuchsanleitung stehen zwei verschiedene Formeln für die Berechnung der Durchbiegung $D(x)$ bei zweiseitiger Auflage:
\begin{align*}
	D_1(x) &= \frac{F}{48\cdot EI}\left(3L^2x-4x^3\right) \qquad && \text{ für }\quad 0\leq x\leq\frac{L}{2}\quad\text{ und } \\
	D_2(x) &= \frac{F}{48\cdot EI}\left(4x^3-12x^2+9L^2x-L^3\right) \qquad && \text{ für }\quad \frac{L}{2}\leq x\leq L.
\end{align*}
Zur Auswertung wurde ein
\begin{equation*}
	x' = L-x \qquad \text{ für } \frac{L}{2}\leq x\leq L
\end{equation*}
definiert. Das heißt
\begin{align*}
	D_1(x') &= D_1(L-x) \\
	&= \frac{F}{48\cdot EI}\left(3L^2(L-x)-4(L-x)^3\right) \\
	&= \frac{F}{48\cdot EI}\left(3L^3-3L^2x-4(-x^3+3x^2L-3xL^3+L^3) \right) \\
	&= \frac{F}{48\cdot EI}\left(3L^3-3L^2x+4x^3-12x^2L+12xL^3-4L^3\right) \\
	&= \frac{F}{48\cdot EI}\left(4x^3-12x^2L+9xL^2-L^3\right) \\
	&= D_2(x).
\end{align*}
Anstatt also die Werte in $x\leq\frac{L}{2}$ und $x\geq\frac{L}{2}$ aufzuteilen und jeweils $D_1(x)$ und $D_2(x)$ zu verwenden, ist es genauso richtig Die Abstände zum nähergelegenen Rand $x$ und $x'$ zu benutzen und immer in die erste Formel einzusetzen.

Die bei der Regression verwendeten Werte siehe Tabelle \ref{Werte der Regression}.
\begin{table}
\begin{center}
\begin{tabular}{c | c | c | c}
	$x$ & $C(x)$ & $D_\text{links}$ & $D_\text{rechts}$ \\
	 in \si{\metre} & in \SI{e+8}{\newton\per\metre} & in \si{\milli\metre} & in \si{\milli\metre} \\
	\hline
	0.240 & 1.920 & 1.300 & 1.195 \\
	0.235 & 1.906 & 1.300 & 1.150 \\
	0.230 & 1.890 & 1.315 & 1.110 \\
	0.225 & 1.872 & 1.285 & 1.070 \\
	0.220 & 1.853 & 1.270 & 1.070 \\
	0.215 & 1.833 & 1.265 & 1.045 \\
	0.210 & 1.811 & 1.260 & 1.020 \\
	0.205 & 1.787 & 1.270 & 0.985 \\
	0.200 & 1.762 & 1.225 & 0.970 \\
	0.195 & 1.736 & 1.230 & 1.000\\
	0.185 & 1.679 & 1.170 & 0.885 \\
	0.175 & 1.618 & 1.140 & 0.820 \\
	0.165 & 1.551 & 1.105 & 0.765 \\
	0.155 & 1.480 & 1.080 & 0.700 \\
	0.145 & 1.405 & 1.030 & 0.640 \\
	0.125 & 1.242 & 0.900 & 0.500 \\
	0.105 & 1.066 & 0.780 & 0.390 \\
	0.085 & 0.877 & 0.655 & 0.250 \\
	0.065 & 0.680 & 0.510 & 0.180 \\
	0.045 & 0.475 & 0.365 & 0.075 \\
	0.025 & 0.266 & 0.140 & 0.050
\end{tabular}
\end{center}
\caption{In der Regression verwendete Werte}
\label{Werte der Regression}
\end{table}
Dabei ist $x$ der Abstand zum nähergelegenen Rand, $C(x) = \frac{F}{48\cdot I}\left(3L^2x-4x^3\right)$ der Wert, der bei der Regression gegen die jeweilige Durchbiegung aufgetragen wurden und $D_\text{links}$ bzw. $D_\text{rechts}$ die gemessene Durchbiegung.