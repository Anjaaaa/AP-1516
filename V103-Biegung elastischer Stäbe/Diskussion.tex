\begin{tabular}{c | c | c | c}
	Art & Literatur & Berechnet & Abweichung \\
	\hline
	Rund & \SI{10e+10}{\newton\per\metre\squared} & \SI{5.29(4)e+10}{\newton\per\metre\squared} & -52.9\% \\
	Eckig & \SI{7e+10}{\newton\per\metre\squared} &\SI{7.37(5)e+10}{\newton\per\metre\squared} & +5.3\%\\
	Beidseitig links & \SI{7e+10}{\newton\per\metre\squared} & \SI{18.0(6)e+10}{\newton\per\metre\squared} & +157.1\% \\
	Beidseitig rechts & \SI{7e+10}{\newton\per\metre\squared} & \SI{13.8(6)e+10}{\newton\per\metre\squared} & +97.1\% \\
\end{tabular} \\
Für Messing stand auf Wikipedia $(78-123)\cdot\SI{e+9}{\newton\per\metre\squared}$. \\
Für Aluminium $\SI{70e+9}{\newton\per\metre\squared}$.

\ \\
\ \\

\input{Biegung_beidseitig.tex}
\ \\
\ \\
Die mittlere Biegung des Stabes ohne Gewicht
\begin{align*}
	D_\text{links} &= \SI{7.669(33)e-3}{\metre} \\
	D_\text{rechts} &= \SI{8.385(13)e-3}{\metre}
\end{align*}

\begin{figure}[h!]
	\centering
	\includegraphics[width=\textwidth]{Werte_ohne.png}
	\caption{Messwerte für D und x ohne Gewicht}
\end{figure}
\begin{figure}[h!]
	\centering
	\includegraphics[width=\textwidth]{Werte_mit.png}
	\caption{Messwerte für D und x mit Gewicht}
\end{figure}
\begin{figure}[h!]
	\centering
	\includegraphics[width=\textwidth]{Wertedifferenz.png}
	\caption{Wirkliches D und x}
\end{figure}
