Die Kenntnis der Dichte lässt zu, die Materialien der Stäbe zu bestimmen.
Bei dem runden, golden glänzenden Stab, handelt es sich wahrscheinlich um einen Messingstab. Der eckige Stab, der matt silbern wirkt, scheint aus Aluminium zu sein. Die Dichten wurden mit großer Genauigkeit berechnet und stimmen mit den Literaturwerten überein (siehe Tabelle \ref{tab:vergleich_dichte}).

\begin{center}
	\captionof{table}{Vergleich mit Literaturwert der Dichten}
\begin{longtable}{c|c|c|c}
	Art & Literatur\footnote{W. Walcher: "Praktikum der Physik", Tabellen-Anhang} & Berechnet & Abweichung \\
	\hline
	Rund -- Messing & \SI{8500}{\kilo\gram\per\cubic\metre} & \SI{8455.90(10902)}{\kilo\gram\per\cubic\metre}  & 0\% \\
	Eckig -- Aluminium & \SI{2700}{\kilo\gram\per\cubic\metre} & \SI{2785.00(2785)}{\kilo\gram\per\cubic\metre}  & +3\% \\
\end{longtable}
\label{tab:vergleich_dichte}
\end{center}

Auch die Literaturwerte für die Elastizitätsmoduln von Messing und Aluminium können nun mit den berechneten Werten verglichen werden (siehe Tabelle \ref{tab:vergleich_elastizitatsmodul}). Die Abweichungen von den Literaturwerten sind bei einseitiger Einspannung relativ gering. Die Werte bei des beidseitig eingespannten Stabes dagegen sind ziemlich genau doppelt so groß, wie in der Literatur angegeben. Auch nach ausführlicher Fehlersuche in den Berechnungen konnte der fehlende Faktor $\frac{1}{2}$ nicht gefunden werden. \\
Ein systematischer Fehler könnte zu kurze Wartezeiten vor den Messungen sein. Sowohl Durchbiegung als auch Entspannung der Stäbe erfolgen zeitverzögert. Wobei das nicht die Abweichung um ca. 100\% beim beidseitig aufliegenden Stab erklärt. Weitere systematische Fehler konnten nicht gefunden werden.

\begin{center}
\captionof{table}{Vergleich mit Literaturwertwert der Elastizitätsmoduln}
\begin{longtable}{c | c | c | c}
	Art & Literatur\footnote{W. Walcher: "Praktikum der Physik", Tabellen-Anhang} & Berechnet & Abweichung \\
	\hline
	Rund & \SI{10e+10}{\newton\per\metre\squared} & \SI{5.29(4)e+10}{\newton\per\metre\squared} & -47.1\% \\
	Eckig & \SI{7.2e+10}{\newton\per\metre\squared} &\SI{7.37(5)e+10}{\newton\per\metre\squared} & +2.4\%\\
	Beidseitig links & \SI{7.2e+10}{\newton\per\metre\squared} & \SI{14.7(3)e+10}{\newton\per\metre\squared} & +104.1\% \\
	Beidseitig rechts & \SI{7.2e+10}{\newton\per\metre\squared} & \SI{13.8(6)e+10}{\newton\per\metre\squared} & +91.1\% \\
\end{longtable}
\label{tab:vergleich_elastizitatsmodul}
\end{center}
