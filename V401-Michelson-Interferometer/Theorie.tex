\subsection*{Interferenz bei Lichtwellen}
Ziel dieses Versuchs ist die Messung der Wellenlänge eines Lasers und die Bestimmung der Brechzahlen von Luft und $\ce{CO2}$. Ganz zentral ist hierbei die Interferenz von Licht. \\
Für die Erklärung der Interferenz eignet sich die Beschreibung des Lichts als ebene, elektromagnetische Welle
\begin{align}\label{E-Feld}
	\vec{E}(x,t) = \vec{E}_0\cos(kx-\omega t-\delta)
\end{align}
am besten. Man beschränkt sich hier aus Gründen der Einfachheit auf das elektrische Feld und eine Dimension, nämlich die Ausbreitungsrichtung $x$. Die restlichen Größen sind die Wellenzahl $k$, die (Kreis-)Frequenz $\omega$ und ein Phasenverschub $\delta$ bezogen auf einen festen Anfangspunkt. Treffen zwei Lichtwellen aufeinander können sie einfach addiert werden. Das führt bei einem Gangunterschied von $2\pi n$ ($n\in\mathbb{N}$), zwischen den beiden Einzelwellen, zu einer Feldstärke
\begin{align*}
	\vec{E}_{ges}(x,t) &= \vec{E}_{10}\cos(kx-\omega t+2\pi n) + \vec{E}_{20}\cos(kx-\omega t) \\
	&= \left(\vec{E}_{10}+\vec{E}_{20}\right)\cos(kx-wt) \ ,
\end{align*}
bei einem Gangunterschied von $(2n+1)\pi$ dagegen ist die resultierende Gesamtfeldstärke
\begin{align*}
	\vec{E}_{ges}(x,t) &= \vec{E}_{10}\cos(kx-\omega t+(2n+1)\pi) + \vec{E}_{20}\cos(kx-\omega t) \\
	&= \left(\vec{E}_{20}-\vec{E}_{10}\right)\cos(kx-wt) \ .
\end{align*}
Haben die beiden Lichtstrahlen die gleiche Amplitude wird die Feldstärke verdoppelt bzw. komplett ausgelöscht. \\
Da die Feldstärke keine leicht zu messende Größe ist, wird im Versuch die Intensität
\begin{align}
	I = \frac{1}{t_2-t_1}\int_{t_1}^{t_2}\left|\vec{E}(x,t)\right|^2 \text{d}t
\end{align}
verwendet. Der Integrationszeitraum $t_2-t_1$ sollte dabei groß gegenüber der Periodendauer $\frac{2\pi}{\omega}$ sein. Für zwei sich überlagernde elektromagnetische Wellen (hier in komplexer Schreibweise, um die Rechnung zu vereinfachen) aus derselben Quelle (gleiche Amplitude und Frequenz) ist die Intensität dann
\begin{align}
	I &= \frac{\vec{E}_0^2}{t_2-t_1}\int_{t_1}^{t_2}\left|\mathrm{e}^{i(kx-\omega_t-\delta_1)}+\mathrm{e}^{i(kx-\omega t-\delta_2)}\right|^2 \text{d}t \\
	&= \frac{\vec{E}_0^2}{t_2-t_1}\int_{t_1}^{t_2}
	\left(2+2\cos(\delta_2-\delta_1)\right)
	\text{d}t \\
	&= 2\vec{E}_0^2\left(1+\cos(\delta_2-\delta_1)\right) \ .
\end{align}
Demnach kann die Interferenz auch über die Intensität gemessen werden. Sie liegt, abhängig vom Gangunterschied $\delta_2-\delta_1$, zwischen $0$ und $4\vec{E}_0^2$.
\subsection*{Voraussetzungen für die Beobachtung von Interferenz}
Bei alltäglichen Lichtquellen, wie der Sonne oder einer Glühbirne, sind $\delta_1$ und $\delta_2$ keine Konstanten, sondern statistische Funktionen der Zeit, sodass das Integral des Cosinus-Terms über eine lange Zeit $t_2-t_1$ verschwindet und eine konstante Intensität beobachtet wird. Dieses Licht ist nicht interferenzfähig und wird als inkohärent bezeichnet. Kohärentes Licht dagegen kann durch \emph{einen} Ausdruck \ref{E-Feld} mit festen $k, \omega$ und $\delta$ beschrieben werden. \\
Üblicherweise wird ein Laser verwendet, wenn Licht mit hoher Kohärenz gebraucht wird. Durch geeignete Abschirmung können allerdings auch bei nichtkohärenten Lichtquellen Interferenz-Erscheinungen beobachtet werden.