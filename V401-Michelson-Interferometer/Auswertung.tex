\subsection{Bestimmung der Wellenlänge des Lasers}
\begin{table}
	\centering
	\begin{tabular}{cc|cc|c|c}
		Anfangspunkt & Endpunkt & Anfungspunkt in \si{\milli\meter} & Endpunkt \si{\milli\meter}  &  Impuls & Wellenlänge \si{\nano\meter} \\
		\hline
		10.00 & 5.03  & 1.98 & 1.00 & 3001 & 656.4 \\
5.00  & 9.97  & 0.99 & 1.98 & 3010 & 654.4 \\
10.00 & 5.04  & 1.98 & 1.00 & 3002 & 654.9 \\
5.00  & 9.96  & 0.99 & 1.97 & 3002 & 654.9 \\
10.00 & 5.04  & 1.98 & 1.00 & 3003 & 654.6 \\
5.00  & 10.14 & 0.99 & 2.01 & 3110 & 655.1 \\
10.00 & 5.00  & 1.98 & 0.99 & 3028 & 654.5 \\
5.00  & 9.96  & 0.99 & 1.97 & 3001 & 655.1 \\
10.00 & 4.45  & 1.98 & 0.88 & 3052 & 720.8 \\
5.00  & 9.95  & 0.99 & 1.97 & 2998 & 654.4 \\

	\end{tabular}
\end{table}


Die durchschnittliche Wellenlänge beträgt
\begin{align}
	\lambda = \SI{661.5+-6.6e-9}{\meter}

\end{align}

\subsection{Bestimmung des Brechungsindexes von Luft und CO$_2$}
\begin{table}[H]
	\centering
	\begin{tabular}{cccc}
		Druckunterschied in \si{\bar} & Impulse & Brechungsindex & Fehler \\
		\hline
		0.8 & 33 & 1.000297 & 0.000002 \\
0.8 & 33 & 1.000297 & 0.000002 \\
0.8 & 34 & 1.000306 & 0.000002 \\
0.8 & 33 & 1.000297 & 0.000002 \\
0.8 & 33 & 1.000297 & 0.000002 \\

	\end{tabular}
	\label{tab:BrechungsindexLuft}
	\captionof{table}{Brechungsindex Luft}
\end{table}
\todo[color=red, inline]{Anja, ich habe einen Knoten im Kopf oder eventuell auch sonst keine Ahnung, wie man das ausrechnet. In Tabelle \ref{tab:BrechungsindexLuft} sind die Bechungsindices und deren Fehler. Wie berechne ich den Mittelwert mit Standartabweichung? Haben Mittelwert und Fehler jeweils eine Standartabweichung?!}