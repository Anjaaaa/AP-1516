Mit Hilfe des Michelson-Interferometers können die Brechungsindizes von Luft und Kohlenstoffdioxid hier sehr genau bestimmt werden. Trotz anfänglicher Schwierigkeiten, den Laserstrahl zu justieren, sind die Abweichungen von den Literaturwerten\cite{Brechungsindex}
%\footnote{\url{http://www.kayelaby.npl.co.uk/general_physics/2_5/2_5_7.html}}
sehr gering. Es kann argumentiert werden, dass die eigentliche Aussagekraft der Brechungsindex erst ab der $\num{e-4}$-ten Stelle existiert. Auch dann sind die Abweichungen sehr klein (siehe Tabelle~\ref{tab:Vergleich}.)

\begin{table}[h!]
	\centering
	\captionof{table}{Vergleich mit Literaturwerten}	
	\begin{tabular}{c|c|c|c|c}
		& gemessen & Literaturwert & Abweichung & Abweichung ab \num{e-4} \\
		\hline
		$n_\text{Luft}$ & \num{1.000299} & \num{1,000292 } & \SI{7e-4}{\percent} & \SI{3.2}{\percent}\\
		$n_\text{CO$_2$}$ & \num{1.000475} &  \num{1.000 449} & \SI{2.6e-3}{\percent} &\SI{5.5}{\percent}
	\end{tabular}
	\label{tab:Vergleich}
\end{table}

Die gemessene Wellenlänge des Lasers weicht allerdings von dem Wert, der am Versuchsaufbau vermerkt ist ab.
\begin{align}
	\lambda_\text{gemessen} &= \SI{661.5+-6.6e-9}{\meter} \\
	\lambda_\text{gegeben} &= \SI{635e-9}{\meter}
\end{align}
Dies scheint zunächst verwunderlich, ist doch der Brechungsindez so genau gestimmt worden. Allerdings muss hier die Abweichung von der vierten Nachkommastelle berücksichtigt werden. Diese weicht zumindest bei der Berechnung mit Kohlenstoffdioxid ebenso wie die Wellenlänge um zirka fünf Prozent vom Erwartungswert ab.