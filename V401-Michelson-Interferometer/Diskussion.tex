Mit Hilfe des Michelson-Interferometers können die Brechungsindizes von Luft und Kohlenstoffdioxid hier sehr genau bestimmt werden. Trotz anfänglicher Schwierigkeiten, den Laserstrahl zu justieren, sind die Abweichungen von den Literaturwerten\cite{Brechungsindex}
%\footnote{\url{http://www.kayelaby.npl.co.uk/general_physics/2_5/2_5_7.html}}
sehr gering. Es kann argumentiert werden, dass die eigentliche Aussagekraft der Brechungsindex erst ab der $\num{e-4}$-ten Stelle existiert. Auch dann sind die Abweichungen sehr klein (siehe Tabelle~\ref{tab:Vergleich}.)

\begin{table}[h!]
	\centering
	\captionof{table}{Vergleich mit Literaturwerten}	
	\begin{tabular}{c|c|c|c|c}
		& gemessen & Literaturwert & Abweichung & Abweichung ab \num{e-4} \\
		\hline
		$n_\text{Luft}$ & \num{1.000298} & \num{1,000292 } & \SI{6e-6}{\percent} & \SI{2}{\percent}\\
		$n_\text{CO$_2$}$ & \num{1.000468} &  \num{1.000 449} & \SI{2e-3}{\percent} &\SI{4}{\percent}
	\end{tabular}
	\label{tab:Vergleich}
\end{table}

Für die Wellenlängenmessung des Lasers  ist es nicht mögliche einen Vergleichswert anzugeben ohne die Kenntnis technischer Daten.  Die Wellenlänge liegt im roten Bereich des Spektrums. Da jedoch die Wellenlänge bei der Berechnung des Brechungsindex genutzt wird, ist davon auszugehen, dass auch der Messwert der Wellenlänge sehr präzise ist.