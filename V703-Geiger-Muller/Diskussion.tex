In diesem Versuch ist es schwierig, die Genauigkeit der Messungen zu bestimmen, da sowohl die Totzeit und als auch die Nachentladungszeit gerätspezifische Größen sind. Sie können nicht mit Literaturwerten verglichen werden. Die in diesem Versuch berechnete Steigung des Plateaus des Geiger-Müller-Zählrohrs soll möglichst gering sein. Sie ist auch recht klein, was gute Zählrohre auszeichnet:
\begin{align}
	s = \input{build/s.txt} \, \frac{\si{\percent}}{\SI{100}{\volt}} \quad .
\end{align}
Die Nachentlagunszeit ist für $V = \SI{350}{\volt}$ ebenso wie für $V = \SI{700}{\volt}$ gerade
\begin{align}
t = \SI{70+-10}{\micro\second} \quad .
\end{align}
Wobei es sehr schwierig war, die Zeitdifferenz am Oszilloskop abzulesen. Die schwankte sehr stark. Vermutlich ist der Fehler hier sogar etwas zu gering gewählt.  \\
Die im nächsten Versuchsteil bestimmte Nachentladungszeit wird auf zwei verschiedene Wege ermittelt. Am Ozsilloskop abgelesen ist die Zeit $T_1$ und mit Hilfe zweier Proben berechnet ist die Zeit $T_2$. Die Werte unterschieden sich voneinander, haben aber zumindest dieselbe Größenordnung
\begin{align}
	T_1 = \SI{30+-10}{\micro\second} \quad , \\
	T_2 = \input{build/totzeit.txt} \quad .
\end{align}
Pro einreffendes Elektron der $\beta$-Strahlung werden zirka $10^{10}$ geladene Teilchen freigesetzt. Die Anzahl der freigesetzten Ladungen ist proportional zu Spannung (siehe Abbildung~\ref{fig:anzahl_elektronen}). Gerade bei dieser Messung liegt eine hohe Ungenauigkeit vor. Der gemessene Strom schwankte trotz konstanter angelegter Spannung sehr stark. Trotzdem ist ein qualitativer Zusammenhang gut erkennbar.
	
