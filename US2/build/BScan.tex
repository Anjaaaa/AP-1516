\begin{table}
    \centering
    \caption{Berechnete Abstände der Löcher zum oberen Rand $s_\text{oben}$ und zum unteren Rand $s_\text{unten}$, die daraus bestimmte Dicke der Löcher $d$ und die gemessene Dicke der Löcher. Alle Werte in \si{\milli\meter}.}
    \label{tab:ObenGanz}
    \sisetup{parse-numbers=false}
    \begin{tabular}{
	S[table-format=2.2]
	S[table-format=2.2]
	S[table-format=2.2]
	S[table-format=2.2]
	}
	\toprule
	{$s_\text{oben}$}		& {$s_\text{unten}$}		& 
	{$d$}		& {$\tilde{d}$}		\\ 
	\midrule
    20.11 & 59.83 & 0.26  & -0.35 \\
18.35 & 61.59 & 0.26  & 0.25  \\
62.09 & 13.57 & 4.53  & 5.80  \\
54.30 & 22.12 & 3.78  & 4.30  \\
46.76 & 30.92 & 2.52  & 3.55  \\
39.22 & 39.22 & 1.77  & 2.30  \\
31.42 & 47.51 & 1.27  & 2.90  \\
23.13 & 55.56 & 1.52  & 2.05  \\
15.08 & 63.60 & 1.52  & 1.90  \\
7.29  & 0.00  & 72.91 & 2.20  \\
56.31 & 15.84 & 8.05  & 9.80  \\

    \bottomrule
    \end{tabular}
    \end{table}
