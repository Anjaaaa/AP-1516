Tabelle \ref{tab:Disussion} zeigt die Abweichungen der berechneten Abstände der Löcher zum Rand zu den mit der Schieblehre gemessenen Werten. Diese Abweichungen sind alle sehr klein und gehen sowohl nach unten als auch nach oben, sodass die Berechnungsmethoden als gut bezeichnet werden können. Allerdings nur, wenn bei den B-Scans die Grenzschicht ignoriert wird. Wird beachtet, dass die Messung der Pixel eigentlich erst unterhalb des roten Balkens beginnen sollte, sind die Abstände viel zu klein. Woran das liegt ist unklar.
Ablesefehler
Warmeverluste beim warten
Messfehler bei den Temperaturen der Korper (Korper wird rausgeholt, kuhlt schon aus, wahrend das Thermometer noch steigt)
Atomwarme pro Mol bei Graphit bei konstantem Volumen 161.954
Atomwarme pro Mol bei Blei bei konstantem Volumen 106.5 +-  5.0
Atomwarme pro Mol nach Dulon-Petit 24.573

